Let $X/K$ be a variety over a number field $K \subset \mathbb C$.
Let $\bar K$ denote the algebraic closure of $K$ in~$\mathbb C$.
(In other words, we have chosen distinguished embeddings $K \subset \bar K \subset \mathbb C$.)

Then the singular cohomology $H_B = H^i(X(\mathbb C), \mathbb Q)$ naturally caries a Hodge structure,
and the $\ell$-adic \'etale cohomology $H_\ell = H_{\text{\'et}}^i(X_{\bar K}, \mathbb Q_\ell)$
naturally caries a continuous representation of $\text{Gal}(\bar K/K)$.
Artin's comparison isomorphism is a natural isomorphism of vector spaces
$H_B \otimes_{\mathbb Q} \mathbb Q_\ell \cong H_\ell$.

In both cases, this extra structure on the cohomology group can be described by the action of an algebraic group.
In the case of the $\mathbb Q$-Hodge structure $H_B$,
this is the so-called Mumford--Tate group $G_B$,
which can be described as the largest algebraic subgroup over~$\mathbb Q$ of $\text{GL}(H_B)$
that fixes all endomorphisms of all tensor powers of~$H_B$.
For the $\ell$-adic Galois representation $H_\ell$,
we let $G_\ell$ denote that Zariski closure of the image of $\text{Gal}(\bar K/K)$ in $\text{GL}(H_\ell)$.
This is an algebraic group over $\mathbb Q_\ell$.
Let $G_\ell^\circ \subset G_\ell$ denote the connected component of the identity.

The Mumford--Tate conjecture asserts that
Artin's comparison isomorphism induces an isomorphism of algebraic groups
\[
 G_B \times_{\mathbb Q} \text{Spec}(\mathbb Q_\ell) \cong G_\ell^\circ.
\]

In general, very little is known in support of this conjecture.
If $X$ is an abelian variety, then we can say a bit more.
In this case, the conjecture is known if $\dim(X) \le 3$,
if $X$ is simple of prime dimension, or if $X$ has trivial endomorphism ring and odd dimension.
In general, $G_\ell^\circ$ is contained in $G_B \times_{\mathbb Q} \text{Spec}(\mathbb Q_\ell)$.
More can be said under further techinical conditions.
My main contribution is the following result.

\begin{thm}[Commelin, 2019]
 Let $X_1$ and $X_2$ be abelian varieties over~$K$.
 If the Mumford--Tate conjecture is true for $X_1$ and~$X_2$,
 then it is also true for $X_1 \times X_2$.
\end{thm}

Let me sketch the main technical problems that the proof must overcome.
First, let us consider the case that $H_\ell(X_1)$ is isomorphic to $H_\ell(X_2)$ as Galois representation.
In that case, $G_\ell^\circ(X_1 \times X_2)$ is the diagonal in $G_\ell^\circ(X_1) \times G_\ell^\circ(X_2)$.
Now one must prove that the same thing happens for the Hodge structures.
This can be done, using Faltings's theorem that shows that $X_1$ and $X_2$ are isogenous
if the associated $\ell$-adic Galois representations are isomorphic.

But something a bit more subtle could happen:
it might be the case that a proper summand of $H_\ell(X_1)$ is isomorphic to a proper summand of $H_\ell(X_2)$.
And a priori these summands do not need to be defined over $\mathbb Q$ as summands of the $H_B(X_i)$.
Indeed, if $E_i$ is the algebra $\text{End}(X_i) \otimes \mathbb Q$,
then $E_i \otimes \mathbb Q_\ell$ can split into more factors than $E_i$.

This suggests that we should try to reduce to the case that $X_1$ and~$X_2$ are simple.
But it would be even better if we could reduce to a setting where $G_B(X_1)$ and $G_B(X_2)$ are simple.
This can be done, by using motivic techniques.

Motives are objects of cohomological nature, and the subject of a large network of conjectures.
Briefly speaking, there should be a cohomology functor from varieties to motives,
that is universal in the sense that every other suitable cohomology functor factors via this functor to motives.
At the time of writing, there is no definition of motive that is known to satisfy all desiderata,
but there are several proposed definitions that satisfy varying subsets of the desiderata.
In my work, I use an unconditional category of motives put forward by Yves Andr\'e.

I introduced the notion of \emph{adjoint motive},
and I showed that the Mumford--Tate conjecture for an abelian variety
is equivalent to the Mumford--Tate conjecture for the associated adjoint motive.
Using a Goursat-lemma argument,
we can then reduce to the case of two irreducible adjoint motives.

We still have to deal with the fact that $G_\ell^\circ$ might split into a product of simple groups,
even though the Mumford--Tate group is now simple by assumption.

For abelian varieties, it is known that the Mumford--Tate conjecture is independent of~$\ell$.
This is a consequence of the fact that
the $\ell$-adic Galois representations $H_\ell$ (for varying $\ell$)
form a \emph{compatible system of Galois representations};
a concept introduced by Serre.
So one might hope that there is always a prime~$\ell$ for which the above-mentioned problem doesn't occur.
This is indeed the correct general strategy,
but it requires yet again to overcome some problems:
there exist endomorphism algebras $E$ in which every prime splits.

Furthermore, the reduction step to adjoint motives comes with a cost:
a priori, we no longer know that the Mumford--Tate conjecture for these motives is independent of~$\ell$.
The adjoint motives that we consider are examples of so-called abelian motives,
which means that they are constructed from the motives of abelian varieties using tensor operations, quotients, duals, and direct sums.

The solution is to generalise the concept of compatible system of Galois representations.
Instead of considering a family of $mathbb Q_\ell$-representations,
we consider a family of $E_\lambda$-representations, where $E$ is a number field
(e.g., the endomorphism algebra of a simple abelian variety/motive).
We also need to relax certain other technical conditions
about the characteristic polynomial of Frobenius elements in the Galois group.
The resulting objects are called \emph{quasi-compatible} systems of Galois representations.
Many of the good properties of compatible systems also hold for quasi-compatible systems.

In addition, I obtained the following result,
which allows us to conclude
that the Mumford--Tate conjecture for abelian motives is independent of~$\ell$.

Let $M$ be an abelian motive, and let $E \subset \text{End}(M)$ be a number field.
Then the $\ell$-adic Galois representations $H_\ell(M)$
naturally split into direct sums $\bigoplus_{\lambda \mid \ell} H_\lambda(M)$,
where $\lambda$ runs over finite places of $E$ that divide $\ell$.

\begin{thm}[Commelin, 2019]
 The $\lambda$-adic realisations $H_\lambda(M)$
 form a quasi-compatible system of Galois representations.
\end{thm}

The proof of this theorem relies on a technical result of Kisin
about special lifts on integral models of certain Shimura varieties.

With this result in place,
the remainder of the proof of the first theorem
consists of a careful analysis of representations of semisimple algebraic groups over number fields of
classical Dynkin type.
The most tricky case is when $G_\ell^\circ(X_1)$ and $G_\ell^\circ(X_2)$ have type~$D_4$ because of the triality automorphisms of the Dynkin diagram.
But I will not go into further detail of this proof.
