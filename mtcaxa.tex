\documentclass[10pt,twoside,leqno]{article}

\usepackage[utf8]{inputenc}
\usepackage[T1]{fontenc}
\usepackage[full]{textcomp}

\usepackage{csquotes}
\usepackage[english]{babel}

% \usepackage[urw-garamond,expert,
% uppercase=upright,greeklowercase=upright]{mathdesign}
% \usepackage[osf,swashQ]{garamondx}
% \def\kappa{\varkappa}
\usepackage{mathtools}
\mathtoolsset{mathic} % Italic correction before mathmode, works with ~'s.
% \def\mathds{\mathbb}

\usepackage{cfr-lm}
\usepackage{dsfont}  % disable this when loading mathdesign

\usepackage{microtype}
\linespread{1.25}  % = 1.500 * fontheight
% \linespread{1.388} % = 1.666 * fontheight
\usepackage[
% paper=b5paper,
nohead,nomarginpar,
% bindingoffset=.3cm,
paper=a4paper,
]{geometry}

% \raggedbottom

\usepackage{lastpage}
\usepackage{fancyhdr}
\pagestyle{fancy}
\fancyhf{}
\renewcommand{\headrulewidth}{0pt}
\fancyfoot[LE,RO]{\thepage/\pageref{LastPage}}

\usepackage{longtable}
\usepackage{booktabs}
\usepackage{tabu}

\usepackage[inline]{enumitem}
\setlist{noitemsep,nosep,listparindent=\parindent}
\setlist[itemize]{label=\guillemotright}
\setlist[enumerate,1]{ref=\thesubsection.\arabic*}
\setlist[enumerate,2]{label=\alph*.,ref=\theenumi.\alph*}
% \setlist[enumerate*]{label=(\textit{\roman*}\thinspace)}

\usepackage{cjhebrew}

\usepackage[backend=biber,doi=false,url=false,isbn=false,%
sorting=nyt,safeinputenc]{biblatex}
\bibliography{\jobname.bib}
\defbibheading{bibliography}[\bibname]{\sectionstar{#1}}

%%% SECTION HEADINGS
\usepackage{ifthen}
\makeatletter
\renewcommand{\part}[1]{%
 \cleardoublepage%
 \vbox{\null\vskip90pt%
 \normalfont\fontsize{20pt}{30pt}\selectfont%
 \baselineskip=30pt%
 \scshape\noindent\textls*{#1}\par}%
 \addcontentsline{toc}{part}{#1}%
 \@afterindentfalse%
 \@afterheading%
}
\renewcommand{\section}[1]{%
 \vskip2\baselineskip\penalty-250%
 \refstepcounter{section}%
 \vbox{\normalfont\fontsize{12pt}{15pt}\selectfont%
  \centering\scshape\noindent\textls*{\thesection\quad#1}%
  \par}
 \nobreak
 \addcontentsline{toc}{section}{\protect\numberline{\thesection} #1}%
 \@afterindentfalse%
 \@afterheading%
} 
\newcommand{\sectionstar}[1]{%
 \bigskip\penalty-250
 \vbox{\normalfont\fontsize{12pt}{15pt}\selectfont%
  \centering\scshape\noindent\textls*{#1}%
  \par}
 \nobreak\vskip15pt
 \@afterindentfalse%
 \@afterheading%
} 
\renewcommand{\paragraph}[1]{\par\bigskip\refstepcounter{subsection}%
 {\normalfont\normalsize\scshape\noindent\thesubsection%
 \ifthenelse{\equal{#1}{}}%
 {}%
 {\ \textls{#1.}}%
 \ ---}%
}
\newcommand{\readme}{\par\vskip\baselineskip%
 {\normalfont\normalsize\scshape\noindent%
  \textls{Readme.}\ ---}
}
\renewcommand\tableofcontents{%
 \sectionstar{\contentsname}%
 \@starttoc{toc}%
}
\renewcommand*\l@part[2]{%
 \addvspace{15pt \@plus\p@}%
 \noindent{\leavevmode%
  \scshape\textls{#1\qquad#2}%
 }\par\nobreak%
}
\renewcommand*\l@section[2]{%
 \setlength\@tempdima{\parindent}%
 \noindent
 {\leavevmode%
  \hskip\parindent#1\qquad#2%
 }\par\nobreak%
}
\makeatother

%%% MATH PACKAGES
\usepackage{amsmath,amssymb}  % disable when using mathdesign
\usepackage{mathrsfs}         % disable when using mathdesign
\usepackage{mathabx}
\usepackage[all]{xy}

\usepackage[thmmarks,amsmath]{ntheorem}
\usepackage{thmtools}

\numberwithin{equation}{subsection}

\declaretheoremstyle[headformat=swapnumber,headpunct={.\ ---},%
headfont=\normalfont\scshape\lsstyle,bodyfont=\itshape,%
spaceabove=0pt,spacebelow=0pt,%
preheadhook={\bigskip}]{theorem}
\declaretheorem[style=theorem,sibling=subsection]{theorem}
\declaretheorem[style=theorem,sibling=subsection]{proposition}
\declaretheorem[style=theorem,sibling=subsection]{lemma}
\declaretheorem[style=theorem,sibling=subsection]{corollary}
\declaretheorem[style=theorem,sibling=subsection]{conjecture}

\declaretheoremstyle[headformat=swapnumber,headpunct={.\ ---},%
headfont=\normalfont\scshape\lsstyle,bodyfont=\normalfont,%
spaceabove=0pt,spacebelow=0pt,%
preheadhook={\bigskip}]{definition}
\declaretheorem[style=definition,sibling=subsection]{definition}
\declaretheorem[style=definition,sibling=subsection]{exercise}
\declaretheorem[style=definition,sibling=subsection]{example}
\declaretheorem[style=definition,sibling=subsection]{remark}
\declaretheorem[style=definition,sibling=subsection]{construction}

% \declaretheoremstyle[headpunct={\!.},headfont=\itshape,bodyfont=\normalfont,%
% qed=\ensuremath{\square},spaceabove=0pt,spacebelow=0pt]{proof}
% \declaretheoremstyle[headpunct={\!.},headfont=\itshape,bodyfont=\normalfont,%
% qed=\ensuremath{\square},spaceabove=0pt,spacebelow=0pt]{nonumberproof}
% \declaretheorem[style=proof,numbered=no]{proof}

\declaretheoremstyle[headformat=swapnumber,headpunct={.\ ---},%
headfont=\itshape,bodyfont=\normalfont,qed=\ensuremath{\square},%
spaceabove=0pt,spacebelow=0pt,%
preheadhook={\bigskip}]{nproof}
\declaretheorem[style=nproof,sibling=subsection,name=Proof]{nproof}

\let\qed\relax
\usepackage{pf2}
\pfkeywords{jmc}

\usepackage{cleveref}
\crefname{condition}{condition}{conditions}
\crefname{conjecture}{conjecture}{conjectures}
\crefname{construction}{construction}{constructions}
\crefname{corollary}{corollary}{corollaries}
\crefname{diagram}{diagram}{diagrams}
\crefformat{subsection}{\S#2#1#3}
\crefformat{enumi}{\S#2#1#3}
\crefformat{nproof}{\S#2#1#3}
\creflabelformat{equation}{#2#1#3}

%%% MATH MACROS
\newcommand{\id}{\textnormal{id}}

\newcommand{\into}{\hookrightarrow}
\newcommand{\onto}{\twoheadrightarrow}
\newcommand{\longto}{\longrightarrow}
\newcommand{\longinto}{\lhook\joinrel\longrightarrow}

\renewcommand{\Im}{\textnormal{Im}}

\newcommand{\colim}{\mathop{\textnormal{colim}}}

\newcommand{\Hom}{\textnormal{Hom}}
\newcommand{\End}{\textnormal{End}}
\newcommand{\Isom}{\textnormal{Isom}}
\newcommand{\Inn}{\textnormal{Inn}}
\newcommand{\Aut}{\textnormal{Aut}}
\newcommand{\Out}{\textnormal{Out}}
\newcommand{\iHom}{\underline{\Hom}}
\newcommand{\iEnd}{\underline{\End}}
\newcommand{\iIsom}{\underline{\Isom}}
\newcommand{\iInn}{\underline{\Inn}}
\newcommand{\iAut}{\underline{\Aut}}
\newcommand{\iOut}{\underline{\Out}}

\newcommand{\Mat}{\textnormal{Mat}}
\newcommand{\Sym}{\textnormal{Sym}}

\newcommand{\Fil}{\textnormal{Fil}}

\newcommand{\dual}[1]{\check{#1}}

\newcommand{\NN}{\mathbb{N}}
\newcommand{\ZZ}{\mathbb{Z}}
\newcommand{\QQ}{\mathbb{Q}}
\newcommand{\QQl}{\QQ_{\ell}}
\newcommand{\QQlbar}{\bar{\QQ}_{\ell}}
\newcommand{\QQp}{\QQ_{p}}
\newcommand{\QQpbar}{\bar{\QQ}_{p}}
\newcommand{\RR}{\mathbb{R}}
\newcommand{\CC}{\mathbb{C}}
\newcommand{\HQ}{\mathbb{H}}
\newcommand{\FF}{\mathbb{F}}
\newcommand{\FFp}{\FF_{p}}
\newcommand{\FFq}{\FF_{q}}
\newcommand{\FFqbar}{\bar{\FF}_{q}}
\newcommand{\Adele}{\mathbb{A}}
\newcommand{\fin}{\textnormal{f}}

\newcommand{\primes}{\mathscr{L}}

\newcommand{\Spec}{\textnormal{Spec}}

\newcommand{\DelS}{\mathbb{S}}
\newcommand{\Sh}{\textnormal{Sh}}
\newcommand{\mSh}{\mathscr{S}}
\newcommand{\DD}{\mathbb{D}}
\newcommand{\mcG}{\mathcal{G}}
\newcommand{\Kmpt}{\mathcal{K}}
\newcommand{\Sds}{\mathfrak{H}^{\pm}}
\newcommand{\AV}{\mathscr{A}}
\newcommand{\Ag}{\AV_{g}}

\newcommand{\Gal}{\textnormal{Gal}}

% \newcommand{\HH}{\textnormal{H}}
% \newcommand{\Hhom}{\HH_{\textnormal{hom}}}
\newcommand{\HdR}{\HH_{\dR}}
% \newcommand{\Hl}{\HH_{\ell}}
% \newcommand{\Hp}{\HH_{p}}
% \newcommand{\Hlambda}{\HH_{\lambda}}
% \newcommand{\HB}{\HH_{\textnormal{B}}}
% \newcommand{\Hsigma}{\HH_{\sigma}}

% \newcommand{\GG}{\textnormal{G}}
% \newcommand{\Gl}{\GG_{\ell}}
% \newcommand{\Glc}{\Gl^{\circ}}
% \newcommand{\GB}{\GG_{\textnormal{B}}}
% \newcommand{\Gsigma}{\GG_{\sigma}}
% \newcommand{\Gmot}[1]{\GG_{\textnormal{mot},#1}}
% \newcommand{\Gmots}{\Gmot{\sigma}}
% \newcommand{\Gmotl}{\Gmot{\ell}}
% \newcommand{\GmotB}{\Gmot{\textnormal{B}}}
% \newcommand{\Gp}{\GG_{p}}

% \newcommand{\Zl}{\textnormal{Z}_{\ell}}
% \newcommand{\Zlc}{\Zl^{\circ}}
% \newcommand{\ZB}{\textnormal{Z}_{\textnormal{B}}}
% \newcommand{\Zsigma}{\textnormal{Z}_{\sigma}}
% \newcommand{\Zmot}[1]{\textnormal{Z}_{\textnormal{mot},#1}}
% \newcommand{\Zmots}{\Zmot{\sigma}}
% \newcommand{\Zmotl}{\Zmot{\ell}}

\newcommand{\BdR}[1]{\textnormal{B}_{\dR,#1}}
\newcommand{\BHT}[1]{\textnormal{B}_{\textnormal{HT},#1}}
\newcommand{\gr}{\textnormal{gr}}

\newcommand{\Vect}{\textnormal{Vect}}
\newcommand{\Filt}{\textnormal{Filt}}
\newcommand{\Rep}{\textnormal{Rep}}
\newcommand{\QHS}{\QQ\textnormal{HS}}

\makeatletter
\def\cpwith[#1]#2{\textnormal{c.p.}_{#1}(#2)}
\def\cpwithout#1{\textnormal{c.p.}(#1)}
\def\cp{\@ifnextchar[{\cpwith}{\cpwithout}}
\makeatother
\makeatletter
\def\Gmwith[#1]{\mathbb{G}_{\textnormal{m},#1}}
\def\Gmwithout{\mathbb{G}_{\textnormal{m}}}
\def\Gm{\@ifnextchar[{\Gmwith}{\Gmwithout}}
\makeatother
\newcommand{\GL}{\textnormal{GL}}
\newcommand{\SL}{\textnormal{SL}}
\newcommand{\PGL}{\textnormal{PGL}}
\newcommand{\UU}{\textnormal{U}}
\newcommand{\SU}{\textnormal{SU}}
\newcommand{\GU}{\textnormal{GU}}
\newcommand{\PGU}{\textnormal{PGU}}
\newcommand{\OO}{\textnormal{O}}
\newcommand{\SO}{\textnormal{SO}}
\newcommand{\GO}{\textnormal{GO}}
\newcommand{\PGO}{\textnormal{PGO}}
\newcommand{\Spin}{\textnormal{Spin}}
\newcommand{\CSpin}{\textnormal{CSpin}}
\newcommand{\PSpin}{\textnormal{PSpin}}
\newcommand{\Sp}{\textnormal{Sp}}
\newcommand{\CSp}{\textnormal{CSp}}
\newcommand{\PCSp}{\textnormal{PCSp}}
\newcommand{\Lie}{\textnormal{Lie}}

\newcommand{\Char}{\textnormal{X}^{*}}
\newcommand{\Cochar}{\textnormal{X}_{*}}

\newcommand{\Dyn}{\textnormal{Dyn}}

% \newcommand{\mfsl}{\mathfrak{sl}}
% \newcommand{\mfso}{\mathfrak{so}}

\newcommand{\Cliff}{\textnormal{Cl}}
% \newcommand{\spin}{\textnormal{spin}}

% \newcommand{\St}{\textnormal{St}}

\newcommand{\ab}{\textnormal{ab}}
\newcommand{\der}{\textnormal{der}}
\newcommand{\ad}{\textnormal{ad}}

\newcommand{\SmPr}{\textnormal{SmPr}}

\newcommand{\an}{\textnormal{an}}
\newcommand{\cl}{\textnormal{cl}}

\newcommand{\dR}{\textnormal{dR}}
\newcommand{\et}{\textnormal{\'{e}t}}
\newcommand{\sing}{\textnormal{sing}}

\newcommand{\HH}{\textnormal{H}}
\newcommand{\Hl}{\HH_{\ell}}
\newcommand{\Hp}{\HH_{p}}
\newcommand{\Hlambda}{\HH_{\lambda}}
\newcommand{\HB}{\HH_{\textnormal{B}}}

\newcommand{\Zar}{\textnormal{Zar}}

\newcommand{\Mot}{\textnormal{Mot}}

\newcommand{\GG}{\textnormal{G}}
\newcommand{\GB}{\GG_{\textnormal{B}}}
\newcommand{\Gp}{\GG_{p}}
\newcommand{\Gl}{\GG_{\ell}}
\newcommand{\Glambda}{\GG_{\lambda}}

\newcommand{\alg}{\textnormal{alg}}
\newcommand{\tra}{\textnormal{tra}}

\newcommand{\Res}{\textnormal{Res}}
\newcommand{\Nm}{\textnormal{Nm}}
\newcommand{\trace}{\textnormal{tr}}
\newcommand{\rk}{\textnormal{rk}}
\renewcommand{\det}{\textnormal{det}}
\newcommand{\res}{\textnormal{res}}

\newcommand{\chrc}{\textnormal{char}}

\newcommand{\Tangen}[1]{\langle #1 \rangle^{\otimes}}
\newcommand{\Val}{\textnormal{Val}}

\newcommand{\tr}{\textsc{tr}}
\newcommand{\cm}{\textsc{cm}}

\newcommand{\MTC}{\textnormal{MTC}}

\newcommand{\rotatesim}{\rotatebox{90}{$\sim$}}

\usepackage{tikz}
\usetikzlibrary{cd,positioning}
\usetikzlibrary{decorations.pathmorphing}
\usetikzlibrary{decorations.markings}

%%%%%%%%%%%%%% Macros for Dynkin diagrams %%%%%%%%%%%%%%
\newcommand{\dynkinradius}{.04cm}
\newcommand{\dynkinstep}{.35cm}
\newcommand{\dynkinnode}[2]{\fill (\dynkinstep*#1,\dynkinstep*#2) circle (\dynkinradius);}
\newcommand{\dynkinXsize}{1.5}
\newcommand{\dynkinnodespecial}[2]{
 \draw[thick] (#1*\dynkinstep-\dynkinXsize,#2*\dynkinstep-\dynkinXsize) -- (#1*\dynkinstep+\dynkinXsize,#2*\dynkinstep+\dynkinXsize);
 \draw[thick] (#1*\dynkinstep-\dynkinXsize,#2*\dynkinstep+\dynkinXsize) -- (#1*\dynkinstep+\dynkinXsize,#2*\dynkinstep-\dynkinXsize);
}
\newcommand{\dynkinedge}[4]{\draw[thin] (\dynkinstep*#1,\dynkinstep*#2) -- (\dynkinstep*#3,\dynkinstep*#4);}
\newcommand{\dynkinnodes}[4]{\draw[dotted] (\dynkinstep*#1,\dynkinstep*#2) -- (\dynkinstep*#3,\dynkinstep*#4);}
\newcommand{\dynkindoubleedge}[4]{\draw[double,postaction={decorate}] (\dynkinstep*#1,\dynkinstep*#2) -- (\dynkinstep*#3,\dynkinstep*#4);}

\newenvironment{dynkin}{\begin{tikzpicture}[decoration={markings,mark=at position 0.7 with {\arrow{>}}}]}
 {\end{tikzpicture}}
%%%%%%%%%%%%%% End of macros for Dynkin diagrams %%%%%%%%%%%%%%


\def\title{The Mumford--Tate conjecture for products of abelian varieties}
\def\author{Johan Commelin}

\usepackage{datetime}
\def\date{\dayofweekname{\day}{\month}{\year},
 the \ordinaldate{\day} of \monthname, \number\year}

\begin{document}
\begin{center}\Large\scshape
\textls*{\title}
\end{center}

\medskip

\noindent\textit{by} \quad \author \hfill \date

\vskip3\baselineskip

\tableofcontents

\section{Introduction}

The strategy:
\begin{enumerate}
 \item Reduce to the case of hyperadjoint motives, \(M_{1}\) and \(M_{2}\).
 \item Show that if \(\Gl(M_{1} \oplus M_{2}) \subsetneq \Gl(M_{1}) \times \Gl(M_{2})\),
  then we have \(\HH_{\lambda_{1}}(M_{1}) \cong \HH_{\lambda_{2}}(M_{2})\)
  and therefore \(\End(M_{1}) = E = \End(M_{2})\)
  and also \(\HH_{\Lambda}(M_{1}) \cong \HH_{\Lambda}(M_{2})\).
 \item Deduce that \(\GB(M_{1})\) and \(\GB(M_{2})\) are isomorphic
  at all finite and infinite places of~\(E\).
 \item Conclude that \(\GB(M_{1}) \cong G \cong \GB(M_{2})\).
 \item Let \((G,X_{i})\) be the Shimura datum attached to \(M_{i}\).
 \item Fix one infinite place \(v_{0}\) of \(E\),
  such that \(G_{v_{0}}(\RR)\) is non-compact.
 \item Act on \(X_{2}\) by \(\Out(G)\) so that
  special node on the Dynkin diagram of~\(G_{v_{0}}\)
  corresponding to the factor \((G_{v_{0}},X_{v_{0},2})\)
  is the same as the special node
  corresponding to the factor \((G_{v_{0}},X_{v_{0},1})\).
 \item Show (TODO) that the reflex field \(E(G,X_{1})\)
  is the same as \(E(G,X_{2})\).
  \begin{enumerate}
   \item Idea: look locally to see how \(\Gal(\QQlbar/\QQl)\)
    acts on the set of special nodes.
   \item Combine all this \(\ell\)-adic data to determine
    the reflex field.
  \end{enumerate}
 \item Conclude that the special nodes on all components of the Dynkin diagram agree.
 \item Conclude that \((G,X_{1}) = (G,X) = (G,X_{2})\).
 \item Left-over: prove the MTC for the product of two fibres on a connected Shimura variety,
  given that MTC is true for the two factors.
 \item Do this by lifting to a Shimura variety of Hodge type.
 \item Use that MTC is known on centres, and reduce to Faltings.
\end{enumerate}

TODO %TODO
% \hfill\cjRL{hw/s`nh brwK hb' b/sM yhwh}

\section{On algebras with involutions}

We recall the classification results of~\cite{BoI}.
For definitions we give references to~\cite{BoI}.

\section{Natural lifts of adjoint Shimura data of abelian type
 to Shimura data of Hodge type}

\paragraph{Setup} \label{setup}
Let \(G\) be a simple adjoint algebraic group over~\(\QQ\).
Let \(h \colon \DelS \to G_{\RR}\) be a homomorphism,
where \(\DelS = \Res_{\CC/\RR} \Gm\) is the Deligne torus.
The morphism \(h\) endows \(\Lie(G)\) with a Hodge structure of weight~\(0\).
Assume that this Hodge structure is polarisable.
Also assume that it has level~\(2\);
in other words,
assume that \(h^{p,-p}(\Lie(G)) = 0\) if \(p > 1\),
and assume that \(h^{1,-1}(\Lie(G)) \ne 0\).

We also assume that there are diagrams of the following form:
\[
 \begin{tikzcd}
  G^{\sharp} \ar[d,two heads] \ar[r,hook]
  & \Aut(V) \\
  G
 \end{tikzcd}
 \quad \stackrel{\otimes_{\RR}}{\tikz \draw [->,
 line join=round,
 decorate, decoration={
  zigzag,
  segment length=4,
  amplitude=.9,post=lineto,
  post length=2pt
 }]  (0,0) -- (1,0);} \quad
 \begin{tikzcd}
  & G^{\sharp}_{\RR} \ar[d,two heads] \ar[r,hook]
  & \Aut(V)_{\RR} \\
  \DelS \ar[r,swap,"h"] \ar[ru,dashed,"\tilde{h}"]
  & G_{\RR}
 \end{tikzcd}
\]
where
\begin{enumerate*}[label=(\textit{\roman*})]
 \item \(G^{\sharp}\) is a reductive algebraic group over~\(\QQ\)
  such that the projection \(G^{\sharp} \onto G\) has central kernel;
 \item \(G^{\sharp} \into \Aut(V)\) is a finite-dimensional faithful representation
  of~\(G^{\sharp}\) on a vector space~\(V\) over~\(\QQ\);
 \item \(\tilde{h}\) is a homomorphism that lifts~\(h\); and
 \item \(\tilde{h}\) endows~\(V\) with a polarisable Hodge structure
  of type \((0,1) + (1,0)\).
\end{enumerate*}

\paragraph{Goal} In this section we propose natural candidates
for \(G^{\sharp}\),~\(V\), and~\(\tilde{h}\).
We describe these in \cref{natlift}.

\paragraph{}
Recall that \(\DelS(\CC)\) acts on \(\Lie(G)_{\CC}\)
via \(h(z)v = z^{-p}\bar{z}^{-q}v\) for \(v \in \Lie(G)_{\CC}\).
Let \(\mu \colon \Gm \to G_{\CC}\) be the Hodge cocharacter
associated with~\(h\);
it is defined via \(\mu(z)v = z^{-p}v\) for \(v \in \Lie(G)_{\CC}\).


\paragraph{}
The situation in \cref{setup} has been studied extensively by Deligne
in \S1.2~and~\S1.3 of~\cite{Del_ShimVar}.
We give an overview of known implications of the above assumptions.
\begin{enumerate}
 \item The group \(G\) must be
  of type~\(A_{n}\),~\(B_{n}\), \(C_{n}\), or~\(D_{n}\).
  (See \S1.3.8 of~\cite{Del_ShimVar}.)
  The type \(D_{n}\) has a refinement into subtypes
  \(D_{n}^{\RR}\) and~\(D_{n}^{\HQ}\).
  The distinction between
  \(D_{4}^{\RR}\) and~\(D_{4}^{\HQ}\)
  is subtle (see \S2.3.8 of~\cite{Del_ShimVar}).
 \item The endomorphism algebra \(E = \End_{G}(\Lie(G))\)
  is a totally real field.
\end{enumerate}
In particular \(G = \Res_{E/\QQ} \mathcal{G}\),
where \(\mathcal{G}\) is some
absolutely simple adjoint algebraic group over~\(E\).
Write \(\Sigma = \Sigma(E)\) for the set of
embeddings \(E \into \RR\).
We write \(\mathcal{G}_{\sigma}\) for the
algebraic group \(\mathcal{G} \otimes_{\sigma} \RR\),
and we write \(\mathcal{G}_{\sigma,\CC}\)
for \(\mathcal{G}_{\sigma} \otimes_{\RR} \CC\).
Write \(\mu_{\sigma}\) for the projection
of \(\mu \colon \Gm \to G_{\CC}\) onto
the factor \(\mathcal{G} \otimes_{\sigma} \CC\).

Once and for all, fix a maximal torus
\(T = \prod_{\sigma} T_{\sigma}\)
inside \(G_{\CC} = \prod_{\sigma} \mathcal{G}_{\sigma,\CC}\)
such that \(T_{\sigma}\) contains the image of~\(\mu_{\sigma}\).
Write \(\Char(T)\) for the character lattice \(\Hom(T,\Gm)\)
and \(\Cochar(T)\) for the cocharacter lattice \(\Hom(\Gm,T)\).
Composition gives a perfect pairing
\[
 \Char(T) \times \Cochar(T) \stackrel{\circ}{\longrightarrow}
 \Hom(\Gm,\Gm) = \ZZ.
\]
Choose a base~\(\Delta = \bigsqcup_{\sigma} \Delta_{\sigma}\)
of the root system
\(R(G_{\CC},T) = \prod_{\sigma} R(\mathcal{G}_{\sigma,\CC},T_{\sigma})\).
For each \(\sigma \in \Sigma\),
let \(\Delta_{\sigma}^{+}\) denote
\(\Delta_{\sigma} \cup \{\alpha_{\sigma}\}\),
where \(\alpha_{\sigma}\) is
the shortest root with respect to~\(\Delta_{\sigma}\).
One may identify \(\Delta_{\sigma}\) with
the nodes of a connected Dynkin diagram,
and \(\Delta_{\sigma}^{+}\) with
the nodes of the corresponding extended Dynkin diagram.
We say that \(\nu \in \Delta_{\sigma}\) is \emph{special}
if it is in the orbit of \(\alpha_{\sigma}\)
under the action of \(\Aut(\Delta_{\sigma}^{+})\).
We say that an element of \(\Cochar(T_{\sigma})\)
(\textit{i.e.}, a cocharacter)
is \emph{special} if it is the indicator function
on a special element of \(\Delta_{\sigma} \subset \Char(T)\)
under the perfect pairing mentioned above.
\begin{enumerate}[resume]
 \item For each embedding \(\sigma \in \Sigma\),
  the group \(\mathcal{G} \otimes_{\sigma} \RR\)
  is an inner form of its compact form.
 \item For each embedding \(\sigma \in \Sigma\),
 the cocharacter \(\mu_{\sigma}\) is special.
 (See \S1.2.5 of~\cite{Del_ShimVar}.)
\end{enumerate}

\paragraph{Candidates} \label{candidates}
In this paragraph we list candidates for~\(G^{\sharp}\) and~\(V\).
In \cref{natlift} we prove that \(h\) lifts uniquely
to a homomorphism \(\tilde{h} \colon \DelS \to G^{\sharp}_{\RR}\)
such that \(G^{\sharp}\),~\(V\), and~\(h\) satisfy the four conditions
listed in~\cref{setup}.
To describe the candidates we need to distinguish six cases:
\(A_{1}\),~\(A_{n}\)~(\(n \ge 2\)),
\(B_{n}\),
\(C_{n}\),
\(D_{n}^{\RR}\), and~\(D_{n}^{\HQ}\)~(\(n \ge 4\)).
In each case we define a group \(\mathcal{G}^{\sharp}\) over~\(E\)
together with a representation of \(\mathcal{G}^{\sharp}\).
At the end of the paragraph we wrap up all the different cases,
and define \(G^{\sharp}\) and~\(V\).

\begin{enumerate}[label=\thesubsection.\arabic*,align=left,%
  itemsep=\baselineskip,topsep=\baselineskip,%
  leftmargin=0pt,labelindent=0pt,labelsep=1ex,labelwidth=6ex,itemindent=!]
 \item \emph{Type \(A_{1}\).}
  The group \(\mathcal{G}\) is isomorphic to \(\PGL_{1}(Q)\),
  where \(Q\) is a (possibly split) quaternion algebra over~\(E\).
  The quaternion algebra~\(Q\) is unique up to isomorphism.
  Put \(\mathcal{G}^{\sharp} = \GL_{1}(Q)\) and \(V = Q \oplus \dual{Q}\).

 \item \emph{Type \(A_{n}\)~(\(n \ge 2\)).}
  The group \(\mathcal{G}\) over~\(E\) is isomorphic to~\(\PGU(B,\tau)\),
  where \(B\)~is a central simple algebra over
  an \'{e}tale quadratic extension \(F/E\),
  and \(\tau\)~is a unitary involution on~\(B\)
  that restricts to the canonical involution of~\(F\) over~\(E\).
  The pair \((B,\tau)\) is unique up to isomorphism.
  Put \(\mathcal{G}^{\sharp} = \UU(B,\tau)\), and \(V = B\).

 \item \emph{Type \(B_{n}\).}
  The group \(\mathcal{G}\) over~\(E\) is isomorphic to~\(\OO^{+}(W,q)\),
  where \((W,q)\) is a regular quadratic space of dimension~\(2n+1\)
  with trivial discriminant.
  The pair \((W,q)\) is unique up to isomorphism.
  Put \(\mathcal{G}^{\sharp} = \Spin(W,q)\),
  and let \(V = \Cliff^{+}(W,q)\) be the even Clifford algebra.

 \item \emph{Type \(C_{n}\).}
  The group \(\mathcal{G}\) over~\(E\) is isomorphic to~\(\PCSp(A,\iota)\),
  where \(A\) is a central simple algebra over~\(E\) of degree~\(2n\),
  and \(\iota\) is a symplectic involution.
  The pair \((A,\iota)\) is unique up to isomorphism.
  Put \(\mathcal{G}^{\sharp} = \Sp(A,\iota)\), and \(V = A\).

 \item \emph{Type \(D_{n}^{\RR}\)~(\(n \ge 5\)).}
  The group \(\mathcal{G}\) over~\(E\) is isomorphic to~\(\PGO^{+}(A,\iota,f)\),
  where \((A,\iota,f)\) is a central simple algebra over~\(E\)
  of degree~\(2n\) with quadratic pair.
  The triple \((A,\iota,f)\) is unique up to isomorphism.
  Put \(\mathcal{G}^{\sharp} = \Spin(A,\iota,f)\),
  and let \(V = \Cliff^{+}(A,\iota,f)\) be the even Clifford algebra.
  TODO %TODO
  Bull crap, need to explain what we mean with \(\Cliff^{+}(A,\iota,f)\).
  Know what this is on a quadratic space,
  but not on a CSA of dimension \(O(n^2)\).

 \item \emph{Type \(D_{n}^{\HQ}\)~(\(n \ge 5\)).}
  Once again,
  the group \(\mathcal{G}\) over~\(E\) is isomorphic to~\(\PGO^{+}(A,\iota,f)\),
  where \((A,\iota,f)\) is a central simple algebra over~\(E\)
  of degree~\(2n\) with quadratic pair.
  The triple \((A,\iota,f)\) is unique up to isomorphism.
  This time put \(\mathcal{G}^{\sharp} = \OO^{+}(A,\iota,f)\), and \(V = A\).

 \item \emph{Type \(D_{4}^{\RR}\).} TODO %TODO
  trialitarian algebra, spin rep.
 \item \emph{Type \(D_{4}^{\HQ}\).} TODO %TODO
  trialitarian algebra, orthogonal rep.

 \item Observe that in all cases \(V\) is naturally a
  faithful representation of \(\Res_{E/\QQ} \mathcal{G}^{\sharp}\).
  We set
  \[
   G^{\sharp} = (\Res_{E/\QQ} \mathcal{G}^{\sharp}) \cdot \Gm \subset \Aut(V).
  \]
\end{enumerate}

\begin{theorem} \label{natlift}
 Let \(G\) and~\(h\) be as in \cref{setup}.
 Let \(G^{\sharp}\) and~\(V\) be the corresponding candidates
 described in \cref{candidates}.
 Then there is a unique homomorphism
 \(\tilde{h} \colon \DelS \to G^{\sharp}_{\RR}\) that lifts~\(h\)
 and that endows \(V\)
 with a polarisable Hodge structure of type \((0,1) + (1,0)\).
\end{theorem}



\paragraph{}
Suppose that we have an algebraic group
\(G^{\sharp} \onto G\) as in~\cref{setup}.
To verify that \(h \colon \DelS \to G_{\RR}\)
lifts to a homomorphism \(\tilde{h} \colon \DelS \to G^{\sharp}_{\RR}\)
it suffices to verify that
\(\mu \colon \Gm \to G_{\CC}\) lifts to a homomorphism
\(\tilde{\mu} \colon \Gm \to G^{\sharp}_{\CC}\).

% ------ ------ ------ ------ ------ ------

% \paragraph{}
% Let \(G\) be an absolutely simple adjoint real algebraic group,
% and let \(X\) be a conjugacy class of homomorphisms
% \(\DelS \to G\) satisfying the conditions of~\cite{Del_ShimVar}.

% Deligne considers diagrams of the form
% \[
%  (G,X) \longleftarrow (G^{\sharp},X_{1})
%  \lhook\joinrel\longrightarrow (\CSp(V),\Sds),
% \]
% such that \(G\) is the adjoint group of~\(G^{\sharp}\),
% and \(X_{1}\) is a conjugacy class of homomorphisms
% \(\DelS \to G^{\sharp}\).
% Assume that such a diagram exists; in other words,
% assume that \(G\) is of type~\(A_{n}\),~\(B_{n}\), \(C_{n}\), or~\(D_{n}\).
% In this section we propose a natural candidate for \((G^{\sharp}, X_{1})\).

\paragraph{The case \(A_{1}\)\relax} \label{caseA1}
The group \(\mathcal{G}\) over~\(E\) is isomorphic to \(\PGL_{1}(Q)\),
where \(Q\) is a (possibly split) quaternion algebra over~\(E\).
Take \(\mathcal{G}^{\sharp} = \GL_{1}(Q)\) and \(V = Q \oplus \dual{Q}\).
Observe that \(V\) is naturally a faithful representation of
\(\Res_{E/\QQ} \mathcal{G}^{\sharp}\).
Put \(G^{\sharp} = (\Res_{E/\QQ} \mathcal{G}^{\sharp}) \cdot \Gm \subset \Aut(V)\).

We claim that there is a unique
\(\tilde{h} \colon \DelS \to G^{\sharp}_{\RR}\) that lifts~\(h\)
and endows \(V\) with a polarisable Hodge structure of type \((0,1) + (1,0)\).
To see this, it suffices to verify that
\(\mu_{\sigma} \colon \Gm \to \PGL_{1}(Q \otimes_{E,\sigma} \CC)\)
lifts uniquely to some
\(\tilde{\mu}_{\sigma} \colon \Gm \to
 \GL_{1}(Q \otimes_{E,\sigma} \CC) \cdot \Gm
 \subset \Aut(V \otimes_{E,\sigma} \CC)\)
such that \(V\) has precisely weights \(0\) and~\(1\).

Note that \(Q \otimes_{E,\sigma} \CC \cong \Mat_{2}(\CC)\).
Consider the following diagrams:
\[
 \begin{tikzcd}
  (\GL_{2} \times \Gm)/\langle \pm 1 \rangle
  \ar[d,two heads]
  \ar[r,hook,"\phi"]
  & \GL_{4} \\
  \PGL_{2}
 \end{tikzcd}
 \quad \stackrel{\Cochar(\_)}{\tikz \draw [->,
 line join=round,
 decorate, decoration={
  zigzag,
  segment length=4,
  amplitude=.9,post=lineto,
  post length=2pt
 }]  (0,0) -- (1,0);} \quad
 \begin{tikzcd}
  \bigl\{(x_{1},x_{2},k) \in (\frac{1}{2}\ZZ)^{3} \bigm|
  x_{i} \equiv k \mod \ZZ \bigr\}
  \ar[d,"\psi"] \ar[r,"\phi_{*}"]
  & \ZZ^{4} \\
  \ZZ^{2}/D
 \end{tikzcd}
\]
where \(\phi\) is the map \((g,\lambda) \mapsto \lambda(g \oplus g^{\dagger})\);
and \(g \mapsto g^{\dagger}\) is the opposition involution on \(\GL_{2}\)
which maps a matrix to its inverse-transpose;
and \(\phi_{*}\) is the map
\((x_{1},x_{2},k) \mapsto (x_{1} + k, x_{2} + k, -x_{1} + k, -x_{2} + k)\);
and \(D \subset \ZZ^{2}\) is the diagonal;
and \(\psi\) is the map \((x_{1},x_{2},k) \mapsto
 (x_{1} + k, x_{2} + k) + D\).

The special cocharacter in \(\ZZ^{2}/\Delta\) is \((1,0) + D\).
It lifts to \((\frac12, -\frac12, \frac12)\),
which maps to \((1,0,0,1)\) under~\(\phi_{*}\).

\paragraph{The case \(A_{n}\), \(n \ge 2\)\relax}
The group \(\mathcal{G}\) over~\(E\) is isomorphic to~\(\PGU(B,\tau)\),
where \(B\)~is a central simple algebra over
an \'{e}tale quadratic extension \(F/E\),
and \(\tau\)~is a unitary involution on~\(B\)
that restricts to the canonical involution of~\(F\) over~\(E\).
Since \(\mathcal{G}_{\sigma}\) is an absolutely simple group over~\(\RR\),
we see that \(F\) is a \cm~field,
and \(E \subset F\) is the maximal \tr~subfield.

Put \(\mathcal{G}^{\sharp} = \UU(B,\tau)\), and \(V = B\).
Observe that \(V\) is naturally a
faithful representation of \(\Res_{E/\QQ} \mathcal{G}^{\sharp}\).
Put \(G^{\sharp} = (\Res_{E/\QQ} \mathcal{G}^{\sharp}) \cdot \Gm \subset \Aut(V)\).

With similar reasoning as in \cref{caseA1}
we may consider the diagrams:
\[
 \begin{tikzcd}
  (\GL_{n+1} \times \Gm)/\langle \pm 1 \rangle
  \ar[d,two heads]
  \ar[r,hook,"\phi"]
  & \GL_{2n+2} \\
  \PGL_{n+1}
 \end{tikzcd}
 \quad \stackrel{\Cochar(\_)}{\tikz \draw [->,
 line join=round,
 decorate, decoration={
  zigzag,
  segment length=4,
  amplitude=.9,post=lineto,
  post length=2pt
 }]  (0,0) -- (1,0);} \quad
 \begin{tikzcd}
  \bigl\{(\vec x,k) \in (\frac{1}{2}\ZZ)^{n+2} \bigm|
  x_{i} \equiv k \mod \ZZ \bigr\}
  \ar[d,"\psi"] \ar[r,"\phi_{*}"]
  & \ZZ^{2n+2} \\
  \ZZ^{n+1}/D
 \end{tikzcd}
\]
where \(\phi\) is the map \((g,\lambda) \mapsto \lambda(g \oplus g^{\dagger})\);
and \(g \mapsto g^{\dagger}\) is the opposition involution on \(\GL_{n}\)
which maps a matrix to its inverse-transpose;
and \(\phi_{*}\) is the map \((\vec x,k) \mapsto
 (x_{1} + k, \ldots, x_{n+1} + k, -x_{1} + k, \ldots -x_{n+1} + k)\);
and \(D \subset \ZZ^{n+1}\) is the diagonal;
and \(\psi\) is the map \((\vec x,k) \mapsto
 (x_{1} + k, \ldots, x_{n+1} + k) + D\).

There are now \(n\) special cocharacters.
For \(i = 1,\ldots,n\), they are
\[
 \mu_{i} = (\underbrace{1,\ldots,1}_{i},
 \underbrace{0,\ldots,0}_{n+1-i}) + D.
\]
It is evident that one can lift~\(\mu_{i}\) to 
\[
 \big(\underbrace{\tfrac12,\ldots,\tfrac12}_{i},
 \underbrace{-\tfrac12,\ldots,-\tfrac12}_{n+1-i},\tfrac12\big)
 \in (\tfrac12\ZZ)^{n+2},
\]
and that this lift has only weights~\(0\) and~\(1\) under~\(\phi_{*}\).

\paragraph{The case \(B_{n}\)\relax}
This is basically the familiar Kuga--Satake construction.
The group \(\mathcal{G}\) over~\(E\) is isomorphic to~\(\OO^{+}(W,q)\),
where \((W,q)\) is a regular quadratic space of dimension~\(2n+1\)
with trivial discriminant.
Put \(\mathcal{G}^{\sharp} = \Spin(W,q)\),
and let \(V = \Cliff^{+}(W,q)\) be the even Clifford algebra.
Observe that \(V\) is naturally a
faithful representation of \(\Res_{E/\QQ} \mathcal{G}^{\sharp}\).
Put \(G^{\sharp} = (\Res_{E/\QQ} \mathcal{G}^{\sharp}) \cdot \Gm \subset \Aut(V)\).

With similar reasoning as in \cref{caseA1}
we may consider the diagrams:
\[
 \begin{tikzcd}
  \CSpin_{2n+1}
  \ar[d,two heads] \\
  \OO_{2n+1}^{+}
 \end{tikzcd}
 \quad \stackrel{\Cochar(\_)}{\tikz \draw [->,
 line join=round,
 decorate, decoration={
  zigzag,
  segment length=4,
  amplitude=.9,post=lineto,
  post length=2pt
 }]  (0,0) -- (1,0);} \quad
 \begin{tikzcd}
  \bigl\{(\vec x,k) \in \ZZ^{n} \oplus \frac12\ZZ \bigm|
  2k + \sum x_{i} \equiv 0 \mod 2 \bigr\}
  \ar[d,"\psi"] \\
  \ZZ^{n}
 \end{tikzcd}
\]
where \(\psi\) is the map \((\vec x,k) \mapsto \vec x\).

The only special cocharacter is \(\mu = (1,0,\ldots,0) \in \ZZ^{n}\).
It is evident that one can lift~\(\mu\)
to \(\tilde{\mu} = (1,0,\ldots,0,\frac12)\).
The weights of the spin representation are
\[
 (\underbrace{\pm\tfrac12,\ldots,\pm\tfrac12}_{n},1),
\]
and we conclude that the spin representation
only has weights~\(0\) and~\(1\) after composition with~\(\tilde{\mu}\).

\paragraph{The case \(C_{n}\)\relax}
The group \(\mathcal{G}\) over~\(E\) is isomorphic to~\(\PCSp(A,\iota)\),
where \(A\) is a central simple algebra over~\(E\) of degree~\(2n\),
and \(\iota\) is a symplectic involution.
Put \(\mathcal{G}^{\sharp} = \Sp(A,\iota)\), and \(V = A\).
Observe that \(V\) is naturally a
faithful representation of \(\Res_{E/\QQ} \mathcal{G}^{\sharp}\).
Put \(G^{\sharp} = (\Res_{E/\QQ} \mathcal{G}^{\sharp}) \cdot \Gm \subset \Aut(V)\).

With similar reasoning as in \cref{caseA1}
we may consider the diagrams:
\[
 \begin{tikzcd}
  \CSp_{2n}
  \ar[d,two heads]
  \ar[r,hook,"\phi"]
  & \GL_{2n} \\
  \PCSp_{2n}
 \end{tikzcd}
 \quad \stackrel{\Cochar(\_)}{\tikz \draw [->,
 line join=round,
 decorate, decoration={
  zigzag,
  segment length=4,
  amplitude=.9,post=lineto,
  post length=2pt
 }]  (0,0) -- (1,0);} \quad
 \begin{tikzcd}
  \bigl\{(\vec x,k) \in (\frac{1}{2}\ZZ)^{n+1} \bigm|
  x_{i} \equiv k \mod \ZZ \bigr\}
  \ar[d,"\psi"] \ar[r,"\phi_{*}"]
  & \ZZ^{2n} \\
  \bigl\{\vec x \in (\frac{1}{2}\ZZ)^{n} \bigm|
  x_{i} \equiv x_{1} \mod \ZZ \bigr\}
 \end{tikzcd}
\]
where \(\phi_{*}\) is the map
\((\vec x,k) \mapsto (x_{1}+ k, \ldots, x_{n}+k, -x_{1}+k, \ldots, -x_{n}+k)\);
and \(\psi\) is the projection onto the first~\(n\) coordinates.

The only special cocharacter is \(\mu = (\frac12,\ldots,\frac12)\).
It is evident that it can be lifted to
\(\tilde{\mu} = (\frac12,\ldots,\frac12)\),
and we see that \(\phi_{*}(\tilde{\mu})\)
only has weight~\(0\) and~\(1\).

\paragraph{The case \(D_{n}^{\RR}\), \(n \ge 5\)\relax}
This is basically the familiar Kuga--Satake construction.
The group \(\mathcal{G}\) over~\(E\) is isomorphic to~\(\PGO^{+}(A,\iota,f)\),
where \((A,\iota,f)\) is a central simple algebra over~\(E\)
of degree~\(2n\) with quadratic pair.
Put \(\mathcal{G}^{\sharp} = \Spin(A,\iota,f)\),
and let \(V = \Cliff^{+}(A,\iota,f)\) be the even Clifford algebra.
Observe that \(V\) is naturally a
faithful representation of \(\Res_{E/\QQ} \mathcal{G}^{\sharp}\).
Put \(G^{\sharp} = (\Res_{E/\QQ} \mathcal{G}^{\sharp}) \cdot \Gm \subset \Aut(V)\).

With similar reasoning as in \cref{caseA1}
we may consider the diagrams:
\[
 \begin{tikzcd}
  \CSpin_{2n}
  \ar[d,two heads] \\
  \PGO_{2n}^{+}
 \end{tikzcd}
 \quad \stackrel{\Cochar(\_)}{\tikz \draw [->,
 line join=round,
 decorate, decoration={
  zigzag,
  segment length=4,
  amplitude=.9,post=lineto,
  post length=2pt
 }]  (0,0) -- (1,0);} \quad
 \begin{tikzcd}
  \bigl\{(\vec x,k) \in \ZZ^{n} \oplus \frac12\ZZ \bigm|
  2k + \sum x_{i} \equiv 0 \mod 2 \bigr\}
  \ar[d,"\psi"] \\
  \bigl\{\vec x \in (\frac{1}{2}\ZZ)^{n} \bigm|
  x_{i} \equiv x_{1} \mod \ZZ \bigr\}
 \end{tikzcd}
\]
where \(\psi\) is the map \((\vec x,k) \mapsto \vec x\).

The only special cocharacter is \(\mu = (1,0,\ldots,0)\).
It is evident that one can lift~\(\mu\)
to \(\tilde{\mu} = (1,0,\ldots,0,\frac12)\).
The weights of the spin representation are
\[
 (\underbrace{\pm\tfrac12,\ldots,\pm\tfrac12}_{n},1),
\]
and we conclude that the spin representation
only has weights~\(0\) and~\(1\) after composition with~\(\tilde{\mu}\).


\paragraph{The case \(D_{n}^{\HQ}\), \(n \ge 5\)\relax}
Once again,
the group \(\mathcal{G}\) over~\(E\) is isomorphic to~\(\PGO^{+}(A,\iota,f)\),
where \((A,\iota,f)\) is a central simple algebra over~\(E\)
of degree~\(2n\) with quadratic pair.
This time put \(\mathcal{G}^{\sharp} = \OO^{+}(A,\iota,f)\), and \(V = A\).
Observe that \(V\) is naturally a
faithful representation of \(\Res_{E/\QQ} \mathcal{G}^{\sharp}\).
Put \(G^{\sharp} = (\Res_{E/\QQ} \mathcal{G}^{\sharp}) \cdot \Gm \subset \Aut(V)\).

With similar reasoning as in \cref{caseA1}
we may consider the diagrams:
\[
 \begin{tikzcd}
  \GO_{2n}^{+}
  \ar[d,two heads] \\
  \PGO_{2n+1}^{+}
 \end{tikzcd}
 \quad \stackrel{\Cochar(\_)}{\tikz \draw [->,
 line join=round,
 decorate, decoration={
  zigzag,
  segment length=4,
  amplitude=.9,post=lineto,
  post length=2pt
 }]  (0,0) -- (1,0);} \quad
 \begin{tikzcd}
  \bigl\{(\vec x,k) \in (\frac12\ZZ)^{n+1} \bigm|
  x_{i} \equiv k \mod \ZZ \bigr\}
  \ar[d,"\psi"] \\
  \bigl\{\vec x \in (\frac{1}{2}\ZZ)^{n} \bigm|
  x_{i} \equiv x_{1} \mod \ZZ \bigr\}
 \end{tikzcd}
\]
where \(\psi\) is the projection map \((\vec x,k) \mapsto \vec x\).

The only special cocharacter is \(\mu = (\frac12,\ldots,\frac12)\).
It is evident that one can lift~\(\mu\)
to \(\tilde{\mu} = (\frac12,\ldots,\frac12,\frac12)\).
This leaves us with weights
\((1,\ldots,1,0,\ldots,0)\) for the standard representation.

\paragraph{The case \(D_{4}^{\RR}\)\relax}

\paragraph{The case \(D_{4}^{\HQ}\)\relax}

% \section{A local-global principle for forms of adjoint algebraic groups}

% \begin{theorem} \label{locglob}
%  Let \(E\) be a number field.
%  Let \(G\) be an absolutely simple classical group over~\(E\).
%  Assume that \(G\) is either simply connected or adjoint.
%  Then the natural morphism
%  \[
%   \HH^{1}(E, \iAut_{G}) \stackrel{\phi_{G}}{\longto}
%   \prod_{v} \HH^{1}(E_{v}, \iAut_{G})
%  \]
%  is injective.
%  (Here \(v\) runs over all the places of~\(E\), both finite and infinite.)
% \end{theorem}

% \paragraph{First reduction} \label{firstred}
% The proof of \cref{locglob} takes the rest of this section.
% We first show that we may assume that \(G\) is split and simply connected.
% Let \(G'\) be a form of~\(G\) over~\(E\),
% corresponding to a class \([G'] \in \HH^{1}(E, \iAut_{G})\).
% Then we obtain a natural commutative diagram
% \[
%  \begin{tikzcd}
%   \HH^{1}(E, \iAut_{G}) \ar[r,"\phi_{G}"] \ar[d,<->,"\rotatesim"]
%   & \prod_{v} \HH^{1}(E_{v}, \iAut_{G}) \ar[d,<->,"\rotatesim"] \\
%   \HH^{1}(E, \iAut_{G'}) \ar[r,"\phi_{G'}"]
%   & \prod_{v} \HH^{1}(E_{v}, \iAut_{G'}) \\
%  \end{tikzcd}
% \]
% where the downward map on the left sends \([G']\)
% to the distinguished element \(*\) of \(\HH^{1}(E, \iAut_{G})\);
% and the downward map on the right sends \(\phi_{G}([G'])\)
% to \(* \in \prod_{v} \HH^{1}(E_{v}, \iAut_{G})\).
% It is therefore sufficient to prove \cref{locglob}
% in the case that \(G\) is a split group.
% If \(G\) is simply connected, then \(\iAut_{G} \cong \Aut_{G^{\ad}}\),
% and thus the adjoint case follows from the simply connected case.

% \paragraph{} \label{exactEpts}
% For the moment, let \(E\) be any field of characteristic~\(0\)
% (for example a \(p\)-adic local field); and
% let \(G\) be a classical group over~\(E\) that is split and simply connected.
% Let \(\Delta\) denote the Dynkin diagram of~\(G_{\bar{E}}\),
% and let \(\Delta_{\bar{E}}\) denote the Dynkin diagram of~\(G_{\bar{E}}\).
% Recall that \(\iOut_{G}(\bar{E}) \cong \Aut(\Delta)\).
% The assumption that \(G\) is split and simply connected
% implies that
% \begin{enumerate*}[label=(\textit{\roman*})]
%  \item \(\iOut_{G}(E) = \iOut_{G}(\bar{E})\); and
%  \item the sequence \(0 \to \iInn_{G}(E) \to \iAut_{G}(E) \to \iOut_{G}(E) \to 0\)
%   is exact.
% \end{enumerate*}
% See for example proposition~1.5.1 and example~1.5.2 of~\cite{Conrad_RedGrp}.

% \paragraph{}
% For ease of mind we recall that
% \[
%  \Aut(\Delta) =
%  \begin{cases}
%   1 & \text{if \(G\) is of type \(A_{1}\)} \\
%   \ZZ/2\ZZ & \text{if \(G\) is of type \(A_{n}\), with \(n > 1\)} \\
%   1 & \text{if \(G\) is of type \(B_{n}\) or~\(C_{n}\)} \\
%   \mathfrak{S}_{3} & \text{if \(G\) is of type \(D_{4}\)} \\
%   \ZZ/2\ZZ & \text{if \(G\) is of type \(D_{n}\), with \(n > 4\).}
%  \end{cases}
% \]

% \paragraph{} \label{phiDelta}
% We now return to the setting of \cref{locglob} and~\cref{firstred}:
% \(E\) is a number field,
% and \(G\) is a classical group over~\(E\) that is split and simply connected.
% Recall from \cref{exactEpts} that
% the action of \(\Gal(\bar{E}/E)\) on~\(\Aut(\Delta)\) is trivial.
% Therefore \(\HH^{1}(E, \Aut(\Delta)) = \Hom(\Gal(\bar{E}/E), \Aut(\Delta))\),
% which classifies Galois extensions of~\(E\)
% whose Galois group is a subgroup of~\(\Aut(\Delta)\).
% Let \(F/E\) be a Galois extension, and let~\(v\) be any place of~\(E\).
% Recall that \(F \otimes E_{v} \cong \prod_{w|v} F_{w}\),
% and the Galois group acts transitively on the factors~\(F_{w}\).
% Consider the map
% \(\phi_{\Delta} \colon
%  \HH^{1}(E, \Aut(\Delta)) \longto \prod_{v} \HH^{1}(E_{v}, \Aut(\Delta))\),
% which maps a Galois extension \(F/E\) to \((F_{w}/E_{v})_{v}\).
% Hence the map \(\phi_{\Delta}\) is injective,
% since a Galois extension \(F/E\) is trivial
% if and only if it is totally split at all places of~\(E\).

% \paragraph{}
% Recall the short exact sequence
% \( 0 \to \iInn_{G} \to \iAut_{G} \to \iOut_{G} \to 0 \).
% Now consider the following commutative diagram of exact sequences
% \[
%  \begin{tikzcd}[column sep=1em]
%     \HH^{0}(E,\iOut_{G}) \ar[r] \ar[d]
%   & \HH^{1}(E,\iInn_{G}) \ar[r,"\beta"]  \ar[d,"\psi"]
%   & \HH^{1}(E,\iAut_{G}) \ar[r,"\gamma"] \ar[d,"\phi_{G}"]
%   & \HH^{1}(E,\iOut_{G}) \ar[d,"\phi_{\Delta}",hook] \\
%     \prod_{v} \HH^{0}(E_{v},\iOut_{G}) \ar[r,"\delta"]
%   & \prod_{v} \HH^{1}(E_{v},\iInn_{G}) \ar[r,"\varepsilon"]
%   & \prod_{v} \HH^{1}(E_{v},\iAut_{G}) \ar[r]
%   & \prod_{v} \HH^{1}(E_{v},\iOut_{G}) \\
%  \end{tikzcd}
% \]
% Recall that \cref{phiDelta} shows that the map \(\phi_{\Delta}\) is injective.
% We claim that
% the map \(\delta\) is trivial; and
% the map \(\psi\) is injective.
% Recall from \cref{exactEpts} that the sequences
% \( 0 \to \iInn_{G}(E_{v}) \to \iAut_{G}(E_{v}) \to \iOut_{G}(E_{v}) \to 0 \)
% are exact. This shows that the map \(\delta\) is trivial.
% The map \(\psi\) is injective by theorem~6.22 on page~336 of~\cite{PlaRap},
% since \(\iInn_{G} = G^{\ad}\).

% We can now finish the proof of \cref{locglob} by a diagram chase.
% Let \(x \in \HH^{1}(E, \iAut_{G})\) be a class such that
% \(\phi_{G}(x) = *\).
% Then \(\phi_{\Delta} \circ \gamma(x) = *\),
% and since \(\phi_{\Delta}\) is injective we conclude that \(\gamma(x) = *\).
% Therefore \(x = \beta(y)\) for some class \(y \in \HH^{1}(E, \iInn_{G})\).
% Since \(\delta\) is trivial and \(\psi\) is injective,
% we find that \(\varepsilon \circ \psi\) is injecitve.
% Because \(\varepsilon \circ \psi(y) = \phi_{G} \circ \beta(y) = *\),
% we conclude that \(y = *\), and hence \(x = *\).
% This completes the proof of \cref{locglob}.

% \section{Recap: classification of real adjoint Shimura data}

% \readme
% Deligne completely describes all pairs~\((G,X)\), where
% \(G\) is a simple adjoint real algebraic group, and
% \(X\) is a conjugacy class of homomorphisms
% \(h \colon \DelS \to G\) satisfying the following conditions:
% \begin{enumerate*}[label=(\kern-.1pt\textit{\roman*}\kern.6pt)]
%  \item the adjoint representation~\(\Lie(G)\) is of type
%   \(\{(-1,1),(0,0),(1,-1)\}\);
%  \item conjugation by \(h(i)\) is a Cartan involution; and
%  \item \(h\) is non-trivial.
% \end{enumerate*}
% What follows is a partial translation of~\S0 and~\S1 of~\cite{Del_ShimVar}
% to recapitulate concepts and notation.
% Each paragraph indicates to which paragraph of~\cite{Del_ShimVar} it corresponds
% and intermezzos are clearly specified as such.

% \paragraph{Terminology and notation (parts of~\S0 of~\cite{Del_ShimVar})}
% In this section \emph{reductive group}
% always means \emph{connected reductive group}.
% A \emph{covering} of a reductive group is a \emph{connected} covering.
% \emph{Adjoint} group means \emph{reductive adjoint} group.
% If \(G\) is a reductive group, we denote with \(G^{\ad}\) its adjoint group,
% with \(G^{\der}\) its derived subgroup, and with
% \(\rho \colon \tilde{G} \to G^{\der}\) the universal covering of~\(G^{\der}\).
% We always denote with~\(Z\) (or~\(Z(G)\)) the centre of~\(G\),
% and (conflict of notation) with \(\tilde{Z}\) that of~\(\tilde{G}\).

% With a superscript~\({}^{\circ}\) we denote a
% \emph{algebraic connected component}
% (for example: \(Z^{\circ}\) is
% the connected component of the identity of the centre~\(Z\) of~\(G\)).
% The superscript~\({}^{+}\) denotes a
% \emph{topological connected component}
% (for example: \(G(\RR)^{+}\) is the topological connected component of the
% identity of the real points of a group~\(G\)).
% We also denote with \(G(\QQ)^{+}\) the trace of~\(G(\RR)^{+}\) on~\(G(\QQ)\).
% For \(G\) real reductive, we denote with a subscript \({}_{+}\)
% the inverse image of \(G^{\ad}(\RR)^{+}\) in~\(G(\RR)\).
% The same notation \({}_{+}\) for the trace on the rational points of a group.

% If \(X\) is a topological space,
% then we denote with \(\pi_{0}(X)\) the set of its connected components,
% endowed with the quotient topology of that on~\(X\).

% A \emph{hermitian symmetric domain} is a hermitian symmetric space
% with curvature \(< 0\) (that is, without euclidean or compact factors).

% If not expressly mentioned otherwise,
% a \emph{vector space} is supposed to be finite-dimensional,
% and a \emph{number field} is supposed to be of finite degree over~\(\QQ\).

% \paragraph{Moduli spaces of Hodge structures
%  (parts of~\S1.1 of~\cite{Del_ShimVar})}
% Recall that a Hodge structure on a real vector space~\(V\)
% is a bigrading \(V_{\CC} = \bigoplus V^{pq}\) of the complexification of~\(V\),
% such that \(V^{pq}\) is the complex conjugate of~\(V^{qp}\).

% Define an action \(h\) of~\(\CC^{*}\) on~\(V_{\CC}\) by the formula
% \[
%  h(z)v = z^{-p}\bar{z}^{-q}v \qquad \text{for \(v \in V^{pq}\).}
% \]
% The \(h(z)\) commute with complex conjugation on~\(V_{\CC}\),
% and thus are deduced from the extension of scalars of an action,
% again denoted~\(h\), of~\(\CC^{*}\) on~\(V\).
% Regard \(\CC\) as an extension of~\(\RR\),
% and let~\(\DelS\) be its multiplicative group,
% regarded as real algebraic group
% (in other words, \(\DelS = \Res_{\CC/\RR}\Gm\)
% (restriction of scalars in the sense of Weil)).
% One has \(\DelS(\RR) = \CC^{*}\),
% and \(h\) is an action of the algebraic group~\(\DelS\).
% One verifies that this construction defines an equivalence of categories:
% \(\{\)real vector spaces endowed with a Hodge structure
% \(\} \to \{\)
% real vector spaces endowed with an action of the algebraic group~\(\DelS\}\).

% The inclusion \(\RR^{*} \subset \CC^{*}\) corresponds to an inclusion
% of real algebraic groups \(\Gm \subset \DelS\).
% We denote with~\(w_{h}\) (or simply~\(w\))
% the restriction of~\(h^{-1}\) to~\(\Gm\),
% and call \(w \colon \Gm \to \GL(V)\) the \emph{weights}.
% One says that~\(V\) is pure of weight~\(n\)
% if \(V^{pq} = 0\) for \(p + q \ne n\),
% that is, \(w(\lambda)\) is multiplication by the scalar~\(\lambda^{n}\).

% We denote with \(\mu_{h}\) (or simply~\(\mu\))
% the action of~\(\Gm\) on~\(V_{\CC}\)
% defined by \(\mu(z)v = z^{-p}v\) for \(v \in V^{pq}\).
% It is a composition \(\Gm \to \DelS_{\CC} \stackrel{h}{\to} \GL(V_{\CC})\).

% The \emph{Hodge filtration} \(F_{h}\) (or simply~\(F\)) is defined by
% \(F^{p} = \bigoplus_{r \ge p} V^{rs}\).
% One says that \(V\) is of type \(\mathscr{E} \subset \ZZ \times \ZZ\)
% if \(V^{pq} = 0\) for \((p,q) \notin \mathscr{E}\).

% More generally, if \(A\) is a subring of~\(\RR\)
% such that \(A \otimes \QQ\) is a field
% (in practice, \(A = \ZZ\),~\(\QQ\), or~\(\RR\)),
% an \emph{\(A\)-Hodge structure}
% is an \(A\)-module~\(V\) of finite type,
% endowed with a Hodge structure on \(V \otimes_{A} \RR\).

% \paragraph{(\S1.1.10 of~\cite{Del_ShimVar})}
% A \emph{polarisation} of a real Hodge structure~\(V\) of weight~\(n\)
% is a morphism \(\Psi \colon V \otimes V \to \RR(-n)\)
% such that the form \((2\pi i)^{n} \Psi(x, h(i)y)\)
% is symmetric and positive definite.
% The same for \(\ZZ\)-Hodge structures,
% upon replacing \(\RR(-n)\) with \(\ZZ(-n)\), \dots.
% Since \(\Psi(h(i)x, y) = \Psi(x, h(-i)y)\)
% (indeed, \(h(i)\) is trivial on~\(\RR(-n)\)),
% and \(h(-i)y = (-1)^{n}h(i)y\),
% the condition of symmetry is equivalent to:
% \(\Psi\) is symmetric if \(n\) is even,
% alternating if \(n\) is odd.

% Hodge structure that come from algebraic geometry
% are \(\ZZ\)-Hodge structures that are pure and polarisable.
% Fundamental example:
% the Hodge positivity theorems assure that \(\HH^{n}(X, \ZZ)\),
% for \(X\) a smooth projective variety, is polarisable.

% \paragraph{(From the proof of proposition~1.1.14 of~\cite{Del_ShimVar})}
% Recall that a \emph{Cartan involution}
% of a (not necessarily connected) real linear algebraic group~\(G\)
% is an involution~\(\sigma\) of~\(G\)
% such that the real form~\(G^{\sigma}\) of~\(G\)
% (with complex conjugation \(g \mapsto \sigma(\bar{g})\))
% is compact:
% \(G^{\sigma}(\RR)\) is compact
% and intersects all the connected components of \(G^{\sigma}(\CC) = G(\CC)\).

% \paragraph{Classification (parts of~\S1.2 of~\cite{Del_ShimVar})} \label{3cond}
% Consider systems \((G,X)\) of an adjoint simple real algebraic group~\(G\)
% and a \(G(\RR)\)-conjugacy class~\(X\) of
% real algebraic morphisms \(h \colon \DelS \to G\) satisfying:
% \begin{enumerate*}[label=(\kern-.1pt\textit{\roman*}\kern.6pt)]
%  \item the adjoint representation~\(\Lie(G)\) is of type
%   \(\{(-1,1),(0,0),(1,-1)\}\)
%   (in particular, \(h\) is trivial on \(\Gm \subset \DelS\));
%  \item conjugation by \(h(i)\) is a Cartan involution; and
%  \item \(h\) is non-trivial.
% \end{enumerate*}
% By \S1.1.17 of \cite{Del_ShimVar}, the connected components of the spaces~\(X\)
% are irreducible hermitian symmetric domains.
% Hypothesis (\textit{ii}) assures that the Cartan involutions of~\(G\)
% are inner automorphisms, and hence \(G\) is an inner form of its compact form.
% In particular, \(G\) being simple, is absolutely simple.

% The \(G(\CC\)-conjugacy class of \(\mu_{h} \colon \Gm \to G_{\CC}\)
% does not depend on the choice of \(h \in X\).
% We denote it with~\(M_{X}\).

% \begin{proposition}[1.2.2 of~\cite{Del_ShimVar}] \label{Del122}
%  Let \(G_{\CC}\) be a simple adjoint complex algebraic group.
%  To each system \((G,X)\) consisting of a
%  real form~\(G\) of~\(G_{\CC}\)
%  and an~\(X\) as in~\cref{3cond}
%  we attach~\(M_{X}\).
%  This gives a bijection
%  between \(G_{\CC}(\CC)\)-conjugacy classes of pairs \((G,X)\),
%  and \(G_{\CC}(\CC)\)-conjugacy classes of
%  non-trivial morphisms \(\mu \colon \Gm \to G_{\CC}\)
%  that satisfy the following condition:

%  \((\star)\)
%  only the characters \(z^{-1}\),~\(1\), and~\(z\)
%  occur in the representation
%  \(\ad \circ \mu\) of~\(\Gm\) on~\(\Lie(G_{\CC})\).
% \end{proposition}
% \begin{proof}
%  \pf\ Omitted; see~\cite{Del_ShimVar}. \qed
% \end{proof}

% \paragraph{\S1.2.5 of~\cite{Del_ShimVar}}
% Let \(G\) be an adjoint simple complex algebraic group.
% We enumerate the conjugacy classes of non-trivial morphisms
% \(\mu \colon \Gm \to G\) that satisfy condition~\((\star)\) of \cref{Del122},
% in termes of the Dynkin diagram~\(\Delta\) of~\(G\).
% Recall that the latter is canonically attached to~\(G\)---in particular,
% the automorphisms of~\(G\) act on~\(\Delta\)---one can identify nodes
% of~\(\Delta\) with conjugacy classes of maximal parabolic subgroups of~\(G\).

% Let \(T\) be a maximal torus,
% \(X(T) = \Hom(T,\Gm)\),
% \(Y(T) = \Hom(\Gm,T)\)
% (the dual of~\(X(T)\) under the pairing
% \(X(T) \times Y(T) \stackrel{\circ}{\to} \Hom(\Gm,\Gm) = \ZZ\)),
% \(R \subset X(T)\) the set of roots,
% \(B\) a system of simple roots,
% \(\alpha_{0}\) the opposite of the largest root,
% and \(B^{+} = B \cup \{\alpha_{0}i\}\).
% The nodes of~\(\Delta\) are parameterised by~\(B\),
% and those of the extended Dynkin diagram~\(\Delta^{+}\) by~\(B^{+}\).

% A conjugacy class of morphisms from \(\Gm\) to \(G\)
% has a unique representative \(\mu \in Y(T)\)
% in the fundamental Weyl chamber
% \(\langle \alpha, \mu \rangle \ge 0\) for \(\alpha \in B\).
% It is uniquely determined by the positive integers
% \(\langle \alpha, \mu \rangle_{\alpha \in B}\) and,
% \(G\) being adjoint,
% the may be prescribed arbitrarily.
% The condition~\((\star)\) of \cref{Del122}, for \(\mu\) non-trivial,
% translates to
% \[
%  (\star)' \qquad \langle \alpha_{0}, \mu \rangle = -1.
% \]

% Write the largest root as linear combination of the simple roots,
% \(\sum_{\alpha \in B^{+}} n(\alpha)\alpha = 0\), with \(n(\alpha_{0}) = 1\),
% and call a node of~\(\Delta^{+}\) \emph{special}
% if one has \(n(\alpha) = 1\) for the corresponding root \(\alpha \in B^{+}\).
% The quotient of the lattice of coweights by that of the coroots
% acts on \(\Delta^{+}\),
% and this action is simply transitive on the set of special nodes.
% The special nodes are the conjugates under \(\Aut(\Delta^{+})\)
% of the node corresponding to~\(\alpha_{0}\).

% The condition \((\star)'\) translates to

% \((\star)''\) For a simple root \(\alpha \in B\) corresponding to
% a special node of~\(\Delta\), one has \(\langle \alpha, \mu \rangle = 1\).
% For the other simple roots, \(\langle \alpha, \mu \rangle = 0\).

% \paragraph{Intermezzo: the opposition involution}
% We continue the notation of the previous paragraph.
% Let \(w_{0}\) be the longest element of the Weyl group (with respect to~\(B\)).
% Then \(w_{0}(B) = -B\)
% and \(-w_{0}\) defines an element of \(\Aut(\Delta) = \Out(G)\):
% the \emph{opposition involution}.
% This involution is non-trivial if and only if \(\Delta\)
% has type \(A_{k}\) with \(k \ne 1\), \(D_{k}\) with \(k\)~odd, or \(E_{6}\).

% \paragraph{(\S1.2.6 of~\cite{Del_ShimVar})} \label{specialnode}
% In total, the \(G_{\CC}(\CC)\)-conjugacy classes \((G,X)\) as in \cref{Del122}
% are parameterised by the special nodes
% of the Dynkin diagram~\(\Delta\) of~\(G_{\CC}\).
% In particular, for~\(G\) a given real form of~\(G_{\CC}\),
% \(X\) is determined by the corresponding special node~\(s(X)\).
% The node corresponding to \(X^{-1} = \{h^{-1} \mid h \in X\}\)
% is the transform of~\(s(X)\) under the opposition involution.

% \begin{proposition}[1.2.7 of~\cite{Del_ShimVar}]
%  Let \(G\) be an adjoint simple real algebraic group,
%  an suppose that there exist morphisms \(h \colon \CC^{*}/\RR^{*} \to G\)
%  satisfying the conditions of~\cref{3cond}.
%  Then the set of such morphisms has two connected components,
%  exchanged by \(h \mapsto h^{-1}\).
%  Either component has the neutral component~\(G(\RR)^{+}\) of~\(G\)
%  as its stabiliser.
% \end{proposition}
% \begin{proof}
%  \pf\ Omitted; see~\cite{Del_ShimVar}. \qed
% \end{proof}

% \begin{corollary}[1.2.8 of~\cite{Del_ShimVar}]
%  Let \((G,X)\) be as in~\cref{3cond},
%  and \(s\) the corresponding node of the Dynkin diagram of~\(G_{\CC}\).
% \begin{enumerate*}[label=(\kern-.1pt\textit{\roman*}\kern.6pt)]
% \item If \(s\) is not fixed by the opposition involution,
%  then \(G(\RR)\) and~\(X\) are connected.
% \item If \(s\) is fixed by the opposition involution,
%  then \(G(\RR)\) and~\(X\) have two connected components;
%  the components of~\(X\) are exchanged by \(h \mapsto h^{-1}\),
%  and by \(g \in G(\RR) - G(\RR)^{+}\).
% \end{enumerate*}
% \end{corollary}

% \paragraph{Symplectic embeddings (\S1.3 of~\cite{Del_ShimVar})}
% Let \(V\) be a real vector space,
% endowed with a non-degenerate alternating form~\(\Psi\).
% The corresponding \emph{Siegel half space}~\(\mathfrak{H}^{+}\)
% admits the following description:
% it is the space of complex structures~\(h\) on~\(V\),
% such that~\(\Psi\) is of type~\(1,1\)
% (under the identification of complex structures
% and Hodge structures of type \(\{(-1,0),(0,-1)\}\),
% \S1.1.3~of~\cite{Del_ShimVar})
% and such that the form \(\Psi(x,h(i)y)\) is symmetric and positive definite.
% If one replaces ``positive definite'' by ``definite'',
% the obtained \emph{double Siegle half space}~\(\Sds\)
% is a conjugation class of morphisms \(h \colon \DelS \to \CSp(V)\).

% \paragraph{(\S1.3.2 of~\cite{Del_ShimVar})} \label{Del132}
% Let \(G\) be an adjoint real algebraic group
% and let \(X\) be a conjugacy class of morphisms \(h \colon \DelS \to G\).
% Assume that \((G,X)\) satisfies conditions~(\textit{i}) and~(\textit{ii})
% of~\cref{3cond}, and replace (\textit{iii}) by

% (\textit{iii})' \(G\) does not have a compact factor.

% The system \((G,X)\) is then a product of systems
% \((G_{\iota},X_{\iota})\) as in~\cref{3cond},
% and \(X_{\iota}\) corresponds with a special node
% of the Dynkin diagram of~\(G_{\iota\CC}\) (\cref{specialnode}).

% Consider diagrams
% \[
%  (G,X) \longleftarrow (G^{\sharp},X_{1}) \rightarrow (\CSp(V),\Sds),
% \]
% where \(G\) is the adjoint group of the reductive group~\(G^{\sharp}\),
% and where \(X_{1}\) is a \(G^{\sharp}(\RR)\)-conjugacy class of
% morphisms from \(\DelS\) to~\(G^{\sharp}\).
% There is a section \(\tilde{G} \to G^{\sharp}\),
% so that \(V\) is a representation of~\(\tilde{G}\).
% The goal is to determine the
% non-trivial irreducible complex representations of~\(\tilde{G}\),
% which is essentially equivalent to occuring in the complexification of
% one of the representations just obtained.

% Replacing \(G^{\sharp}\) by the subgroup generated
% by the derived group~\(G^{\sharp}^{\der}\) and the image of~\(h\),
% one can assume that conjugation by \(h(i)\)
% is a Cartan involution of~\(G^{\sharp}/w(\Gm)\).
% Hence there exists a polarisation~\(\Psi\) on~\(V\)
% (\S1.18(a)~of~\cite{Del_ShimVar}),
% such that \(\rho\) is a morphism from \((G^{\sharp},X_{1})\) to \((\CSp(V),\Sds)\).

% \paragraph{(\S1.3.4 of~\cite{Del_ShimVar})}
% Consider the following projective system~\(H_{n})_{n \in \NN}\):
% order~\(\NN\) by divisibility, \(H_{n} = \Gm\),
% and the transition morphism from \(H_{nd}\) to \(H_{n}\) is \(x \mapsto x^{d}\).
% (Then \(\lim H_{n}\) is
% the universal covering---in the algebraic sense---of~\(\Gm\).)
% A \emph{fractional morphism} of~\(\Gm\) to a group~\(H\)
% is an element of \(\colim \Hom(H_{n}, H)\).
% The same for the group~\(\DelS\).
% For a fractional morphism \(\mu \colon \Gm \to H\),
% defined by \(\mu_{n} \colon H_{n} = \Gm \to H\),
% every linear representation~\(V\) of~\(H\)
% is the sum of subspaces \(V_{a}\) (\(a \in (1/n)\ZZ\))
% such that \(\Gm\) acts on~\(V_{a}\) via~\(\mu_{n}\)
% as multiplication by~\(x^{na}\).
% The \(a\) such that \(V_{a} \ne 0\) are the \emph{weights} of~\(\mu\) on~\(V\).
% Similarly, a fractional morphism \(h \colon \DelS \to H\)
% determines a fractional Hodge decomposition~\(V_{r,s}\) of~\(V\)
% (\(r,s \in \QQ\)).

% \begin{lemma}[1.3.5 of~\cite{Del_ShimVar}] \label{Del135}
%  For \(h \in X\), let \(\tilde{\mu}_{h}\) be
%  the fractional lift of~\(\mu_{h}\) to~\(\tilde{G}_{\CC}\).
%  The representations~\(W\) of~\cref{Del132} are those
%  for which \(\tilde{\mu}_{h}\) has only two weights \(a\)~and~\(a+1\).
% \end{lemma}
% \begin{proof}
%  \pf\ Omitted; see~\cite{Del_ShimVar}. \qed
% \end{proof}

% \paragraph{(\S1.3.6 of~\cite{Del_ShimVar})}
% Let us translate the condition of~\cref{Del135} in terms of roots.
% Let \(T\) be a maximal torus of~\(G_{\CC}\),
% \(\tilde{T}\) its inverse image in \(\tilde{G}_{\CC}\),
% \(B\) a system of simple roots of~\(T\),
% and \(\mu \in Y(T)\) the representative in the fundamental Weyl chamber
% of the conjugacy class of~\(\mu_{h}\)~(\(h \in X\)).
% If \(\alpha\) is the dominant weight of~\(W\),
% the smallest weight is \(-\tau(\alpha)\),
% where \(\tau\) is the opposition involution.
% For \(\beta\) a weight of~\(W\), the \(\langle \mu, \beta \rangle\)
% may only take the two values \(a\)~and~\(a + 1\).
% The weights are all of the form
% \((\alpha + \text{a \(\ZZ\)-linear combination of roots})\),
% and the \(\langle \mu, r \rangle\), with \(r\) a root, are integers.
% The condition is therefore
% \(\langle \mu, -\tau(\alpha) \rangle = \langle \mu, \alpha \rangle - 1\),
% that is to say
% \begin{equation} \label{Del1361}
%  \langle \mu, \alpha + \tau(\alpha) \rangle = 1.
% \end{equation}

% Let us determine the solutions to this condition.
% For all dominant weights~\(\alpha\)
% we have \(\langle \mu, \alpha + \tau(\alpha) \rangle \in \ZZ\),
% because \(\alpha + \tau(\alpha)\) is a \(\ZZ\)-linear combination of roots.
% If \(\alpha \ne 0\), then \(\alpha > 0\),
% for otherwise \(\mu\) kills all the weights of the corresponding representation.
% A dominant weight~\(\alpha\) that satisfies~\cref{Del1361}
% can not be the sum of two weights.

% \begin{lemma}[1.3.7 of~\cite{Del_ShimVar}] \label{Del137}
%  Only the fundamental weights can satisfy~\cref{Del1361}.
% \end{lemma}

% \paragraph{(\S1.3.8 of~\cite{Del_ShimVar})}
% By \cref{Del137}, the desired representations~\(W\)
% factor via a simple factor~\(G_{\iota}\) of~\(G\),
% and their dominant weight is a fundamental weight; it corresponds
% to a node of the Dynkin diagram~\(\Delta_{\iota}\) of~\(G_{\iota\CC}\).
% The necessary and sufficient condition~\cref{Del1361}
% only depends on the projection of~\(\mu\) to~\(G_{\iota\CC}\);
% it corresponds with a special node~\(s\) of~\(\Delta_{\iota}\),
% and \(s\) with a simple root~\(\alpha_{s}\).
% The number \(\langle \mu, \varpi \rangle\), for \(\varpi\) a weight,
% is the coefficient of~\(\alpha_{s}\)
% in the expression of~\(\varpi\)
% as \(\QQ\)-linear combination of the simple roots.

% \paragraph{Table (1.3.9 of~\cite{Del_ShimVar})} \label[table]{DelTable}
% We reproduce the diagrams depicted in~\cite{Del_ShimVar};
% and we add one column of information.

% \begin{tabular}{lm{2.5cm}m{2cm}m{3cm}l}
%  D.t. & Diagram & \(h^{-1,1}\) & \(G_{\iota}\) & Note \\
%  \(A_{n}\) &
%  % \begin{tikzpicture}
%  %  \node (pi1) at (0,0) {};
%  %  \node (pi2) [right=of pi1] {};
%  % \end{tikzpicture}
%  \begin{dynkin}
%   \dynkinedge{1}{0}{2}{0};
%   \dynkinnodes{2}{0}{3}{0};
%   \dynkinedge{3}{0}{5}{0};
%   \dynkinnodes{5}{0}{6}{0};
%   \dynkinedge{6}{0}{7}{0};
%   \foreach \x in {1,...,7}
%   {
%    \ifnum \x=4 {\dynkinnodespecial{\x}{0}}
%    \else {\dynkinnode{\x}{0}}
%    \fi
%   }
%  \end{dynkin}
%  & \(pq\) & \(\PGU(p,q)\)? & \(n = p+q-1\) \\
%  \(B_{n}\) &
%  \begin{dynkin}
%   \dynkinedge{1}{0}{2}{0};
%   \dynkinnodes{2}{0}{3}{0};
%   \dynkinedge{3}{0}{4}{0};
%   \dynkindoubleedge{4}{0}{5}{0};
%   \dynkinnodespecial{1}{0};
%   \foreach \x in {2,...,5}
%   {
%    \dynkinnode{\x}{0}
%   }
%  \end{dynkin}
%  & TODO & \(\PSpin(2,n-2)\)? \\
%  \(C_{n}\)
%  &
%  \begin{dynkin}
%   \dynkinedge{1}{0}{2}{0};
%   \dynkinnodes{2}{0}{3}{0};
%   \dynkinedge{3}{0}{4}{0};
%   \dynkindoubleedge{5}{0}{4}{0};
%   \dynkinnodespecial{5}{0};
%   \foreach \x in {1,...,4}
%   {
%    \dynkinnode{\x}{0}
%   }
%  \end{dynkin}
%  & TODO \\
%  \(D_{n}\)
%  &
%  \begin{dynkin}
%   \foreach \x in {2,...,4}
%   {
%    \dynkinnode{\x}{0}
%   }
%   \dynkinnode{4.5}{.9}
%   \dynkinnode{4.5}{-.9}
%   \dynkinnodespecial{1}{0}
%   \dynkinedge{1}{0}{2}{0}
%   \dynkinnodes{2}{0}{3}{0}
%   \dynkinedge{3}{0}{4}{0}
%   \dynkinedge{4}{0}{4.5}{.9}
%   \dynkinedge{4}{0}{4.5}{-.9}
%  \end{dynkin}
%  & TODO \\
%  \(D_{n}\)
%  &
%  \begin{dynkin}
%   \foreach \x in {1,...,4}
%   {
%    \dynkinnode{\x}{0}
%   }
%   \dynkinnodespecial{4.5}{.9}
%   \dynkinnode{4.5}{-.9}
%   \dynkinedge{1}{0}{2}{0}
%   \dynkinnodes{2}{0}{3}{0}
%   \dynkinedge{3}{0}{4}{0}
%   \dynkinedge{4}{0}{4.5}{.9}
%   \dynkinedge{4}{0}{4.5}{-.9}
%  \end{dynkin}
%  & TODO \\
%  \(E_{6}\)
%  &
%  \begin{dynkin}
%   \foreach \x in {2,...,5}
%   {
%    \dynkinnode{\x}{0}
%   }
%   \dynkinnodespecial{1}{0}
%   \dynkinnode{3}{1}
%   \dynkinedge{1}{0}{5}{0}
%   \dynkinedge{3}{0}{3}{1}
%  \end{dynkin}
%  & TODO \\
%  \(E_{7}\)
%  &
%  \begin{dynkin}
%   \foreach \x in {1,...,5}
%   {
%    \dynkinnode{\x}{0}
%   }
%   \dynkinnodespecial{6}{0}
%   \dynkinnode{3}{1}
%   \dynkinedge{1}{0}{6}{0}
%   \dynkinedge{3}{0}{3}{1}
%  \end{dynkin}
%  &
%  TODO
% \end{tabular}


\section{Hyperadjoint motives}

TODO %TODO

\section{The Hodge numbers of a motive with endomorphisms}
\label{padicHodgeEnd}

\paragraph{Setup}
Let \(K \subset \CC\) be a finitely generated field.
Let \(\bar{K}\) be the algebraic closure of~\(K\) in~\(\CC\).
Let \(M\) be a motive over~\(K\).
Let \(E\) be subfield of~\(\End(M)\).
Let \(\tilde{E}\) be the normal closure of~\(E\) in~\(\CC\).
After passing to a suitable finite extension of~\(K\)
we may and do assume that \(\tilde{E} \subset K\).

Let \(p\) be a prime number.
Fix an embedding \(K \into K_{v}\) into some \(p\)-adic field~\(K_{v}\).
Let \(\BdR{K_{v}}\) be the ring of \(p\)-adic periods associated with~\(K_{v}\),
in the sense of Fontaine~\cite{Fo94}.
Fix an embedding \(\bar{K} \into \BdR{K_{v}}\)
that extends the embedding \(K \into K_{v}\).
Write \(\bar{K}_{v}\) for the algebraic closure of~\(K_{v}\) in~\(\BdR{K_{v}}\).

Let \(\Sigma = \Sigma(E)\) be the set of embeddings \(E \into \tilde{E}\).
Depending on the context,
we may also view an element \(\sigma \in \Sigma\)
as embedding \(E \into K\) or \(E \into \CC\) or \(E \into \BdR{K_{v}}\).

\paragraph{}
Observe that \(\HB(M) \otimes_{\QQ} \CC\) is a module over
\(E \otimes_{\QQ} \CC = \prod_{\sigma \in \Sigma} \CC\).
Thus we have a decomposition
\(\HB(M) \otimes_{\QQ} \CC \cong
 \bigoplus_{\sigma \in \Sigma} \HB(M) \otimes_{E,\sigma} \CC\).
Since the comparison isomorphism
\(\HB(M) \otimes_{\QQ} \CC \cong \HdR(M) \otimes_{K} \CC\)
is compatible with the action of~\(E\),
we find that \(\HdR(M)\) is a filtered module over
\(E \otimes_{\QQ} K \cong \prod_{\sigma \in \Sigma} K\).
This gives a decomposition
\(\HdR(M) \cong \bigoplus_{\sigma \in \Sigma} \HdR(M)^{(\sigma)}\)
in such a way that
\[
 \HdR(M)^{\sigma} \otimes_{K} \CC \cong \HB(M) \otimes_{E,\sigma} \CC
\]
is a natural isomorphism of filtered vector spaces.

\paragraph{}
We have a natural isomorphism of vector spaces
\(\HB(M) \otimes \QQp \cong \Hp(M)\).
This isomorphism is compatible with the action of~\(E\),
and \(\Hp(M)\) is a module over \(E \otimes \QQp\).

We have a natural isomorphism
\(\HdR(M) \otimes_{K} \BdR{K_{v}} \cong \Hp(M) \otimes_{\QQp} \BdR{K_{v}}\)
of filtered modules with an action by \(\Gal(\bar{K}_{v}/K_{v})\).
This isomorphism is compatible with the action of~\(E\),
so that \(\Hp(M) \otimes_{\QQp} \BdR{K_{v}}\)
is a module over
\(E \otimes_{\QQ} \BdR{K_{v}} = \prod_{\sigma \in \Sigma} \BdR{K_{v}}\).
Thus we get a natural isomorphism
\[
 \HdR(M)^{(\sigma)} \otimes_{K} \BdR{K_{v}} \cong
 \Hp(M) \otimes_{E \otimes \QQp, \sigma \otimes \id} \BdR{K_{v}}
\]
of filtered modules with an action by \(\Gal(\bar{K}_{v}/K_{v})\),
for each \(\sigma \in \Sigma\).

\paragraph{}
Assume that \(p\) is totally split in~\(E\).
Then we get a simplified description of
\(\Hp(M) \otimes_{E \otimes \QQp, \sigma \otimes \id} \BdR{K_{v}}\),
as follows.
For each \(\sigma \in \Sigma\)
the composite embedding \(\sigma \colon E \into K \into K_{v}\)
factors via \(\QQp \subset K_{v}\).
This gives a bijection between \(\Sigma\)~and
the places \(\pi\) of~\(E\) that lie above~\(p\).

Since \(\Hp(M)\) is a module over \(E \otimes \QQp = \prod_{\pi|p} E_{\pi}\)
we get a decomposition \(\Hp(M) \cong \bigoplus_{\pi|p} \HH_{\pi}(M)\)
of Galois representations.
The above bijection between \(\Sigma\) and the places \(\pi|p\)
allows us to unambigously write \(\Hp(M)^{(\sigma)}\) for~\(\HH_{\pi}(M)\).
Then
\(\Hp(M) \otimes_{E \otimes \QQp, \sigma \otimes \id} \BdR{K_{v}}
 = \Hp(M)^{(\sigma)} \otimes_{\QQp} \BdR{K_{v}}\).

\paragraph{}
The upshot of all these comparison isomorphisms
and their decompositions induced by the action of~\(E\)
is that we get a refined view on the Hodge numbers of~\(M\).
It is classical that the comparison isomorphisms
equate the Hodge numbers of~\(\HB(M)\)
with the Hodge--Tate numbers of the Galois representation~\(\Hp(M)\).
Now we also have the following result.

\begin{theorem} \label{Elin_padicHodge}
 Let \(M\) be a motive over a finitely generated field \(K \subset \CC\).
 Let \(E\) be subfield of~\(\End(M)\), and
 let \(\tilde{E}\) be the normal closure of~\(E\) in~\(\CC\).
 Assume that \(\tilde{E} \subset K\).
 Let \(\Sigma = \Sigma(E)\) be the set of embeddings \(E \into \tilde{E}\).
 For each \(\sigma \in \Sigma\) we then have
 \[
  \bigl\{\,
  \text{Hodge numbers of \(\HB(M) \otimes_{E,\sigma} \CC\)}
  \,\bigr\}
  =
  \bigl\{\,
  \text{Hodge--Tate numbers of
   \(\Hp(M) \otimes_{E \otimes \QQp, \sigma \otimes \id} \BdR{K_{v}}\)}
  \,\bigr\}.
 \]
\end{theorem}
\begin{proof}
 \pf\ By the above computations we see that both the left-hand side
 and the right-hand side are equal to
 \(
  \bigl\{\,
  \text{Hodge numbers of \(\HdR(M)^{(\sigma)}\)}
  \,\bigr\}
 \). \qed
\end{proof}

\section{Compatible systems}
TODO %TODO
Give a summary of the other article.
First finish that article,
then copy/paste the relevant definitions and the main theorem.

\section{Geometric lifts of Galois representations}

Give a summary of the relevant result of~\cite{No06}.

\section{Structure of hyperadjoint abelian motives}
\label{hypadMstruct}

\begin{lemma} \label{EGstruct}
 Let \(K \subset \CC\) be a finitely generated field.
 Let \(M\) be an irreducible hyperadjoint abelian motive over~\(K\).
 Let \(E\) be the endomorphism algebra of~\(M\).
 Assume that \(E = \End(M_{\CC})\).
 Then \(E\) is a totally real field,
 and \(\GB(M) = \Res_{E/\QQ} \mathcal{G}\)
 for some absolutely simple adjoint algebraic group \(\mathcal{G}/E\).
\end{lemma}

\paragraph{}
We follow the conventions of table~2.3.8 of~\cite{Del_ShimVar}.
Let \(K \subset \CC\) be a finitely generated field.
Let \(M\) be an irreducible hyperadjoint abelian motive over~\(K\).
\begin{enumerate}
 \item All the simple factors of~\(\GB(M)_{\CC}\)
  have the same Dynkin type;
  the only types that can occur are the classical types
  \(A_{k}\),~\(B_{k}\), \(C_{k}\), and~\(D_{k}\).
 \item For the non-compact simple factors of~\(\GB(M)_{\RR}\)
  there is a refined type:
  \(A_{k}\),~\(B_{k}\), \(C_{k}\), \(D_{k}^{\RR}\), and~\(D_{k}^{\HQ}\).
  For \(k \ge 4\) there is a unique (up to isomorphism)
  non-compact adjoint simple real algebraic group of type~\(D_{k}^{\RR}\)
  and a unique (up to isomorphism)
  non-compact adjoint simple real algebraic group of type~\(D_{k}^{\HQ}\).
  For \(k = 4\) these two isomorphism classes coincide,
  for \(k \ge 5\) they are distinct.
 \item All the non-compact simple factors of~\(\GB(M)_{\RR}\)
  have the same refined type.
 \item In conclusion, it makes sense to say that
  the motive~\(M\) has type
  \(A_{k}\),~\(B_{k}\), \(C_{k}\), \(D_{k}^{\RR}\), or~\(D_{k}^{\HQ}\).
  For \(k = 4\) the distinction between 
  \(D_{k}^{\RR}\), and~\(D_{k}^{\HQ}\)
  is subtle, and depends on \(\GB(M)\) over~\(\QQ\).
  We refer to the final paragraphs of table~2.3.8 of~\cite{Del_ShimVar}
  for more details.
\end{enumerate}

\paragraph{}
Let \(E\) be the endomorphism algebra of~\(M\).
Let \(\tilde{E}\) be the normal closure of~\(E\) in~\(\CC\).
After passing to a suitable finite extension of~\(K\)
we may and do assume that \(E = \End(M_{\CC})\) and \(\tilde{E} \subset K\).
Let \(\mu \colon \Gm[\CC] \to \GB(M)_{\CC}\)
be the cocharacter that determines the Hodge structure~\(\HB(M)\).
Let \(\mathcal{G}/E\) be an absolutely simple adjoint algebraic group
such that \(\GB(M) = \Res_{E/\QQ} \mathcal{G}\).
Let \(\sigma \colon E \into \tilde{E}\) be an embedding.
Write \(G_{\sigma}\) for \(\mathcal{G} \times_{\sigma} \RR\),
and write \(\mu_{\sigma}\) for the projection of~\(\mu\)
onto \(G_{\sigma,\CC}\).
It follows from \S1 of~\cite{Del_ShimVar}
that if \(G_{\sigma}\) is non-compact, then
\begin{enumerate*}[label=(\textit{\roman*})]
 \item the \(G_{\sigma}(\CC\)-conjucacy class of~\(\mu_{\sigma}\)
  is determined by a special node of the Dynkin diagram of
  \(\mathcal{G} \times_{\sigma} \CC\);
 \item the opposition involution exchanges the
  \(G_{\sigma}(\CC)\)-conjugacy class of~\(\mu_{\sigma}\)
  with the
  \(G_{\sigma}(\CC)\)-conjugacy class of~\(\mu_{\sigma}^{-1}\); and
 \item the \(G_{\sigma}(\CC)\)-conjucacy class of~\(\mu_{\sigma}\)
  determines the isomorphism class of~\(G_{\sigma}\)
  (as real algebraic group).
\end{enumerate*}
We refer to \S\S1.2.6--1.2.8 of~\cite{Del_ShimVar} for more details and proofs.
See table~1.3.9 of~\cite{Del_ShimVar} for an overview of which Dynkin diagrams
and the nodes that may occur as special nodes.

\paragraph{}
Let us now discuss the relationship between the type of~\(M\)
and the \(p\)-adic realisations of~\(M\).
Since \(M\) is abelian and hyperadjoint,
we know that \(\HB(M) \otimes_{E,\sigma} \CC\)
has type \(\{(-1,1), (0,0), (1,-1)\}\) with multiplicities
\((h^{-1,1},h^{0,0},h^{1,-1})\).
Since \(E\) is totally real, we know that \(h^{-1,1} = h^{1,-1}\),
and therefore \(h^{-1,1}\) determines the others.
Write \(\Hp(M)^{(\sigma)}\) for
\(\Hp(M) \otimes_{E \otimes \QQp, \sigma \otimes \id} \BdR[K_{v}]\).
It follows from \cref{Elin_padicHodge}
that \(\Hp(M)^{(\sigma)}\) determines~\(h^{-1,1}\).
Note that \(G_{\sigma}\) is compact if and only if \(h^{-1,1} = 0\).
In particular \(\Hp(M)^{(\sigma)}\) can distinguish whether
\(G_{\sigma}\) is compact or not.
Assume that \(G_{\sigma}\) is non-compact.
\begin{enumerate}
 \item Suppose that the type of~\(M\) is~\(A_{k}\).
  Then every node of the Dynkin diagram may occur as special node.
  If the \(i\)th node is the special node associated with~\(\mu_{\sigma}\)
  then \(h^{-1,1} = i\cdot(k+1-i)\).
  We conclude that \(\Hp(M)^{(\sigma)}\)
  determines the special node up to the opposition involution.
  Thus we recover the isomorphism class of~\(G_{\sigma}\)
  and we recover the \(G_{\sigma}(\CC)\)-conjucay class of~\(\mu_{\sigma}\)
  up to a sign (in other words, we do not (yet)
  see the difference between \(\mu_{\sigma}\) and~\(\mu_{\sigma}^{-1}\)).
 \item Suppose that the type of~\(M\) is \(B_{k}\) or~\(C_{k}\).
  Then there is only one node of the Dynkin diagram
  that may occur as special node.
  We conclude that \(\Hp(M)^{(\sigma)}\) determines
  the isomorphism class of~\(G_{\sigma}\),
  and since the opposition involution acts trivially,
  \(\Hp(M)^{(\sigma)}\) also determines
  the \(G_{\sigma}(\CC)\)-conjucay class of~\(\mu_{\sigma}\).
 \item Suppose that the Dynkin type of~\(G_{\sigma,\CC}\) is~\(D_{k}\).
  Then the three extremal nodes may occur as special node of the Dynkin diagram.
  In all cases we have \(h^{-1,1} = 2k-2\),
  and hence \(\Hp(M)^{(\sigma)}\) is not (yet) able to distinguish
  between the types \(D_{k}^{\RR}\) and~\(D_{k}^{\HQ}\).
\end{enumerate}

\begin{proposition} \label{inputNoot}
 Let \(M\) be an irreducible hyperadjoint abelian motive of type~\(D_{k}\).
 One can recognize from \(\HH_{\Lambda}(M)\)
 whether \(M\) is of type \(D_{k}^{\RR}\) or of type~\(D_{k}^{\HQ}\).
\end{proposition}
\begin{proof}
 \pf\ Use theorem~6.2 till remark~6.6 of~\cite{No06}. \qed
\end{proof}

\paragraph{}
Recall that up to isomorphism
there is a unique real algebraic group of type~\(D_{k}^{\RR}\)
and a unique real algebraic group of type~\(D_{k}^{\HQ}\).
\Cref{inputNoot} shows that one may determine the isomorphism class of the
non-compact simple factors of~\(\GB(M)_{\RR}\)
even in the case that \(M\) is of type~\(D_{k}\).

\section{An intermediate result}

\begin{theorem} \label{intermed}
 Let \(M_{1}\) and~\(M_{2}\) be two irreducible hyperadjoint abelian motives
 over a finitely generated field \(K \subset \CC\).
 Assume that \(\MTC(M_{1})\) and \(\MTC(M_{2})\) are true.
 Then
 \[
  M_{1} \cong M_{2} \iff \exists \ell :
  \Gl(M_{1} \oplus M_{2}) \subsetneq \Gl(M_{1}) \times \Gl(M_{2}).
 \]
\end{theorem}

\begin{corollary} \label{intermed_sums}
 Let \(M_{i}\), with \(i \in I\),
 be a finite collection of hyperadjoint abelian motives.
 Write \(M = \bigoplus M_{i}\).
 If \(\MTC(M_{i}\) is true for all \(i \in I\),
 then \(\MTC(M)\) is true.
\end{corollary}
\begin{proof}
 \pf\ We may and do assume that for \(i,j \in I\)
 we have \(M_{i} \cong M_{j} \iff i = j\).
 By \cref{intermed},
 we know that if \(i,j \in I\) are two different indices,
 then \(\Gl(M_{i} \oplus M_{j}) \cong \Gl(M_{i}) \times \Gl(M_{j})\).
 Recall that there is a natural injection
 \(\Gl(M) \into \prod_{i \in I} \Gl(M_{i})\)
 and the image projects surjectively onto the factors \(\Gl(M_{i})\).
 By the lemma in step~3 on pages~790--791 of~\cite{Ribet}
 we conclude that
 \(\Gl(\bigoplus_{i \in I} M_{i}) \cong \prod_{i \in I} \Gl(M_{i})\).
 Since we know \(\MTC(M_{i})\) for all \(i \in I\)
 we conclude that \(\MTC(M)\) is true. \qed
\end{proof}

\paragraph{Strategy}
The proof of \cref{intermed} will consist of several distinct steps.
Let \(h_{i} \colon \DelS \to \GB(M_{i})_{\RR}\)
be the morphism that defines the Hodge structure on~\(\HB(M_{i})\).
\begin{enumerate}
 \item We first show that \(\End(M_{1}) = E = \End(M_{2})\).
 \item We then show that \(\GB(M_{1}) \cong \GB(M_{2})\).
 \item We choose the isomorphism \(\GB(M_{1}) \cong \GB(M_{2})\)
  in such a way that \(h_{1}\) and~\(h_{2}\) are conjugate.
 \item In the final step we deduce \(M_{1} \cong M_{2}\).
\end{enumerate}

\paragraph{}
Let us embark on the proof of \cref{intermed}.
Let \(E_{i}\) be the endomorphism algebra of~\(M_{i}\).
Let \(\tilde{E_{i}}\) be the normal closure of~\(E_{i}\) in~\(\CC\).
After passing to a suitable finitely generated extension of~\(K\)
we may and do assume that
\begin{enumerate*}[label=(\textit{\roman*})]
 \item \(E_{i} = \End(M_{i,\CC})\);
 \item \(\tilde{E_{i}} \subset K\); and
 \item \(\Gl(M_{1} \oplus M_{2})\) is connected for all prime numbers~\(\ell\).
\end{enumerate*}
(Actually, assumption~(\textit{iii}) implies assumption~(\textit{i}),
but that is not important right now.)
Let \(\bar{K}\) be the algebraic closure of~\(K\) in~\(\CC\).

\paragraph{Step 0}
The forward implication in \cref{intermed} is trivial.
Therefore we assume that there exists a prime number~\(\ell\)
such that \(\Gl(M_{1} \oplus M_{2}) \subsetneq \Gl(M_{1}) \times \Gl(M_{2})\).

\paragraph{Step 1}
Observe that \(\Hl(M_{i}) \cong \Lie(\Gl(M_{i}))\).
By Goursat's lemma for Lie algebras,
our assumption implies that there is an irreducible summand
of~\(\Hl(M_{1})\) that is isomorphic as \(\ell\)-adic Galois representation
to an irreducible summand of~\(\Hl(M_{2})\).
The irreducible summands of~\(\Hl(M_{i})\)
correspond to the places~\(\lambda_{i}\) of~\(E_{i}\) above~\(\ell\),
via the decomposition
\(\Hl(M_{i}) = \prod_{\lambda_{i}|\ell} \HH_{\lambda_{i}}(M_{i})\).

Hence there exists a place~\(\lambda_{1}\) of~\(E_{1}\),
a place~\(\lambda_{2}\) of~\(E_{2}\),
and an isomorphism
\(\phi \colon \HH_{\lambda_{1}}(M_{1}) \to \HH_{\lambda_{2}}(M_{2})\)
in \(\Rep_{\QQl}(\Gal(\bar{K}/K))\).
The isomorphism~\(\phi\) induces an isomorphism
\(\psi \colon \End_{\Gal}(\HH_{\lambda_{1}}(M_{1}))
 \to \End_{\Gal}(\HH_{\lambda_{2}}(M_{2}))\).
Observe that \(\End_{\Gal}(\HH_{\lambda_{i}}(M_{i})) \cong E_{i,\lambda_{i}}\),
is commutative, and therefore the isomorphism~\(\psi\)
does not depend on the choice of~\(\phi\).
By foobar?? we recover
\(\lambda_{i} \colon E_{i} \into E_{i,\lambda_{i}}
\cong \End_{\Gal}(\HH_{\lambda_{i}}(M_{i}))\).
Therefore \(\psi\) induces a canonical isomorphism \(E_{1} = E_{2}\)
that identifies~\(\lambda_{1}\) with~\(\lambda_{2}\).
From now on we write \(E\) for \(E_{1} = E_{2}\),
\(\tilde{E}\) for \(\tilde{E_{1}} = \tilde{E_{2}}\),
and \(\Lambda\) for the set of finite places of~\(E\).

\paragraph{Step 2}
Note that by foobar?? we find
\(\HH_{\Lambda}(M_{1}) \cong \HH_{\Lambda}(M_{2})\).
Therefore \(\Glambda(M_{1}) \cong \Glambda(M_{2})\)
for all \(\lambda \in \Lambda\).
Let \(G_{i}\) be an absolutely simple adjoint algebraic group over~\(E\)
such that \(\GB(M) \cong \Res_{E/\QQ}G_{i}\).
Since we know \(\MTC(M_{i})\)
we know that \(G_{i,\lambda} \cong \Glambda(M_{i})\),
for all \(\lambda \in \Lambda\).

By the discussions and calculations in \cref{hypadMstruct}
we also know the isomorphism class of \(G_{i,\sigma}\)
for all \(\sigma \colon E \into \tilde{E} \subset \RR\);
and in particular \(G_{1,\sigma} \cong G_{2,\sigma}\).
By \cref{locglob} we conclude that \(G_{1} \cong G_{2}\)
and therefore \(\GB(M_{1}) \cong \GB(M_{2})\).

\paragraph{Step 3}
Denote by \(X_{i}\) the \(\GB(M_{i})(\RR)\)-conjugacy class of
\(h_{i} \colon \DelS \to \GB(M_{i})_{\RR}\).
Remark that \((\GB(M_{i})_{\RR}, X_{i})\)
decomposes as a product of factors \((G_{i,\sigma}, X_{i,\sigma})\)
where \(\sigma\) runs over the embeddings \(E \into \tilde{E}\).

Choose an isomorphism \(\phi \colon G_{1} \to G_{2}\).
If \(\phi\) identifies \(X_{1}\) with \(X_{2}\)
then we are done with this step, and we continue with step~4.
Assume that this is not the case.
Let \(\sigma_{0}\) be an embedding for which
\(\phi\) does not identify \(X_{1,\sigma_{0}}\) with \(X_{2,\sigma_{0}}\).
Then \(G_{1,\sigma_{0}} \cong G_{2,\sigma_{0}}\) is non-compact.
Let \(\Delta_{\sigma_{0}}\) be the Dynkin diagram of~\(G_{1,\sigma_{0}}\).
Note that \(X_{1,\sigma_{0}}\) corresponds
to a special node~$v_{1}$ of~\(\Delta_{\sigma_{0}}\),
and (via~\(\phi\)) \(X_{2,\sigma_{0}}\) corresponds
to a different special node~$v_{2}$ of~\(\Delta_{\sigma_{0}}\).
Let \(\psi\) be an automorphism of \(\Delta_{\sigma_{0}}\)
that maps~$v_{2}$ to~$v_{1}$.
(Such an automorphism exists!
Indeed, we have distinguished between type~\(D_{k}^{\RR}\) and~\(D_{k}^{\HQ}\).)
Since \(G_{1}\) is adjoint, we have isomorphism
\(\Out(G_{1}) = \Out(G_{1,\sigma_{0}}) = \Aut(\Delta_{\sigma_{0}})\)
and we may view \(\psi\) as automorphism of~\(G_{1}\).
Replace \(\phi\) with \(\phi \circ \psi\).
This has the effect that \(\phi\) identifies
\(X_{1,\sigma_{0}}\) with~\(X_{2,\sigma_{0}}\).

Let \(\tau \colon E \into \tilde{E}\) be any embedding.
Our task is to show that \(\phi\) identifies
\(X_{1,\tau}\) with~\(X_{2,\tau}\).
To this end, let \(\Sigma(E)\) be the set of embeddings \(E \into \tilde{E}\)
and recall that \(\Gal(\tilde{E}/\QQ)\)
acts faithfully and transitively on~\(\Sigma(E)\).
Let \(\gamma \in \Gal(\tilde{E}/\QQ)\) be an automorphism of~\(\tilde{E}\)
that sends \(\sigma_{0}\) to~\(\tau\).
By Chebotaryov's density theorem,
there exists a prime~\(p\),
and a place~\(\pi\) of~\(\tilde{E}\) lying above~\(p\)
such that
\(\tilde{E}_{\pi}/\QQp\) is unramified, and
\(\gamma\) is the Frobenius element in
\(\Gal(\tilde{E}_{\pi}/\QQp) \subset \Gal(\tilde{E}/\QQ)\).
We denote with \(E_{\pi}\) the completion of~\(E\) inside \(\tilde{E}_{\pi}\).

Let \(K \into K_{v}\) be an embedding of~\(K\) into a \(p\)-adic field
that extends the embedding \(\tilde{E} \into \tilde{E}_{\pi}\).
Now observe that \cref{padicHodgeEnd} shows that
for all \(\sigma \in \Sigma(E)\), and \(i = 1,2\),
the cocharacter \(\mu_{i,\sigma}\) is defined over~\(K\);
after extending scalars to \(\BdR[K_{v}]\)
this cocharacter agrees with the cocharacter~\(\nu_{i}\)
that defines the grading on
\(\gr(\Hp(M_{i}) \otimes_{E \otimes \QQp, \sigma \otimes \id} \BdR{K_{v}})\).
Let \(\psi\) be an isomorphism \(\HH_{\pi}(M_{1}) \to \HH_{\pi}(M_{2})\)
of \(E_{\pi}\)-adic Galois representations.
This induces an isomorphism
\(\psi' \colon \GG_{\pi}(M_{1}) \to \GG_{\pi}(M_{2})\)
of linear algebraic groups over~\(E_{\pi}\).
After extending scalars to \(\BdR[K_{v}]\),
the isomorphism \(\psi'_{\BdR[K_{v}]}\) identifies \(\nu_{1}\) with~\(\nu_{2}\).

Let \(\Sigma_{0} \subset \Sigma(E)\) be the orbit of \(\sigma_{0}\)
under the action of~\(\gamma\).
In other words, \(\Sigma_{0} = \Gal(\tilde{E}_{\pi}/\QQp) \cdot \sigma_{0}\).
By assumption, \(\tau \in \Sigma_{0}\).

Observe that we get decompositions
\[
 \HH_{\pi}(M_{i}) \otimes_{E_{\pi}} \BdR[K_{v}] \cong
 \bigoplus_{\sigma \in \Sigma_{0}}
 \Hp(M_{i}) \otimes_{E \otimes \QQp, \sigma \otimes \id} \BdR{K_{v}},
\]
and \(\GG_{\pi}(M_{i}) \times_{E_{\pi}} \BdR[K_{v}] \cong
 \prod_{\sigma \in \Sigma_{0}} \GG_{\pi,\sigma}(M_{i})\)
and \(\nu_{i} = (\nu_{i,\sigma})_{\sigma \in \Sigma_{0}}\).


Let us recapitulate what we have got.
We have an isomorphism \(\phi \colon G_{1} \to G_{2}\)
of algebraic groups over~\(E\).
This isomorphism satisfies the assumption that \(\phi\) identifies
the conjugacy class \(X_{1,\sigma_{0}}\) with~\(X_{2,\sigma_{0}}\).
We also have an isomorphism
\(\psi' \colon \GG_{\pi}(M_{1}) \to \GG_{\pi}(M_{2})\)
of algebraic groups over~\(E_{\pi}\),
and over~\(\BdR[K_{v}]\) it identifies \(\nu_{1}\) with~\(\nu_{2}\).
This last statement may be decomposed into:
we have an isomorphism
\(\psi'_{\sigma} \colon \GG_{\pi,\sigma}(M_{1}) \to \GG_{\pi,\sigma}(M_{2})\)
of algebraic groups over~\(\BdR[K_{v}]\)
that identifies \(\nu_{1,\sigma}\) with~\(\nu_{2,\sigma}\).

Recall that \(G_{i} \times_{E} E_{\pi} = \GG_{\pi}(M_{i})\).
So we also get isomorphisms
\(\phi_{\pi} \colon \GG_{\pi}(M_{1}) \to \GG_{\pi}(M_{2})\) and
\(\phi_{\pi,\sigma} \colon
 \GG_{\pi,\sigma}(M_{1}) \to \GG_{\pi,\sigma}(M_{2})\).
Since \(\phi\) identifies the conjugacy class of~\(\mu_{1,\sigma_{0}}\)
with the conjugacy class of~\(\mu_{2,\sigma_{0}}\),
we conclude that \(\phi_{\pi,\sigma_{0}}\)
identifies the conjugacy class of~\(\nu_{1,\sigma_{0}}\),
with the conjugacy class of~\(\nu_{2,\sigma_{0}}\).
In other words,
\(\phi_{\pi,\sigma_{0}}\) and \(\psi'_{\sigma_{0}}\)
induce isomorphisms
\(\Dyn(\GG_{\pi,\sigma_{0}}(M_{1})) \to \Dyn(\GG_{\pi,\sigma_{0}}(M_{2}))\)
that both preserve the decorated special node.
Hence they are the same isomorphism (TODO: the case \(D_{4}\)). %TODO
Therefore the isomorphisms \(\phi_{\pi}\) and~\(\psi'\)
(from \(\GG_{\pi}(M_{1})\) to \(\GG_{\pi}(M_{2})\))
differ by an inner automorphism.
We conclude that \(\Dyn(\phi_{\pi,\tau})\)
also preserves the decorated special node,
and thus \(\phi_{\CC}\) identifies \(X_{1,\tau}\) with~\(X_{2,\tau}\).

\paragraph{Step 4}
Lift to a Shimura datum of Hodge type, and use the following result.

\begin{proposition} \label{twofib}
 Let \(K \subset \CC\) be a finitely generated field.
 Let \(S\) be a scheme of finite type over~\(K\).
 Let \(f \colon \mathcal{A} \to S\) be an abelian scheme.
 Let \(x\) and~\(y\) be two points in~\(S(K)\).
 Let \(\ell\) be a prime number.
 If \(\Lie(\Gl(\mathcal{A}_{x}))\) and \(\Lie(\Gl(\mathcal{A}_{y}))\)
 are isomorphic as Galois representations,
 then the abelian varieties \(\mathcal{A}_{x}\) and~\(\mathcal{A}_{y}\)
 are isogenous over a finite extension of~\(K\).
\end{proposition}
\begin{proof}
 \pf\ View the lisse \(\ell\)-adic sheaf \(\mathrm{R}^{1}f_{*}\QQl\)
 as representation
 \(\pi(S,\bar{x}) \to \GL(\Hl^{1}(\mathcal{A}_{x})\).
 Let \(\rho_{x}\) denote the representation
 \(\Gal(\bar{K}/K) \to \GL(\Hl^{1}(\mathcal{A}_{x})\).
 Via parallel transport,
 the Galois representation on~\(\Hl^{1}(\mathcal{A}_{y})\)
 corresponds with a representation
 \(\rho_{y} \colon \Gal(\bar{K}/K) \to \GL(\Hl^{1}(\mathcal{A}_{x})\).
 Let \(G \subset \GL(\Hl^{1}(\mathcal{A}_{x})\) be the smallest
 subgroup such that \(G(\QQl)\) contains the image
 of both \(\rho_{x}\) and~\(\rho_{y}\).

 Consider the maps \(\ab \colon G \to G^{\ab}\)
 and \(\ad \colon G \to G^{\ad}\).
 It follows from \cref{MTCcentre} that
 the map \(\ab \circ \rho_{s}\) does not depend on~\(s \in \{x,y\}\).
 By assumption
 \(\Lie(\Gl(\mathcal{A}_{x}))\) and \(\Lie(\Gl(\mathcal{A}_{y}))\)
 are isomorphic as Galois representations,
 and therefore \(\ad \circ \rho_{s}\) does not depend on~\(s \in \{x,y\}\).

 Write \(Z\) for the finite intersection \(\ker(\ab) \cap \ker(\ad)\).
 Define the morphism \(\phi \colon \Gal(\bar{K}/K) \to G(\QQl)\)
 via \(g \mapsto \rho_{x}(g)\rho_{y}(g)^{-1}\).
 We claim that \(\phi\) is a homomorphism.
 Note that \(\ab \circ \phi\) and \(\ad \circ \phi\)
 are trivial, and therefore \(\phi\) factors via~\(Z(\QQl)\).
 We need to show that
 \[
  \rho_{x}(g_{1})\rho_{y}(g_{1})^{-1} \cdot \rho_{x}(g_{2})\rho_{y}(g_{2})^{-1}
  \stackrel{?}{=}
  \rho_{x}(g_{1}g_{2}) \cdot \rho_{y}(g_{1}g_{2})^{-1}
  =
  \rho_{x}(g_{1})\rho_{x}(g_{2}) \cdot
  \rho_{y}(g_{2})^{-1}\rho_{y}(g_{1})^{-1}.
 \]
 This amounts to showing that
 \(\rho_{y}(g_{1})^{-1} \cdot \phi(g_{2}) =
  \phi(g_{2}) \cdot \rho_{y}(g_{1})^{-1}\).
 Since \(Z\) is contained in the centre of~\(G\),
 we know that \(\phi(g_{2})\) commutes with \(\rho_{y}(g_{1})^{-1}\).
 This proves the claim that \(\phi\) is a homomorphism.

 Because \(Z\) is finite,
 the homomorphism \(\phi\) becomes trivial
 after replacing \(K\) by suitable a finite extension;
 which means \(\rho_{x} = \rho_{y}\).
 The \namecref{twofib} now follows from Faltings' results
 on the Tate conjecture for endomorphisms of abelian varieties
 (Korollar~1 of Satz~4 of~\cite{Fal83}, see also~\cite{Fal84}). \qed
\end{proof}

\section{Main theorem}

\paragraph{}
Let \(K \subset \CC\) be a finitely generated field.
Let \(\mathcal{M}_{K} \subset \Mot_{K}\) be the subcategory
of \emph{abelian} motives for which the Mumford--Tate conjecture is true.

\begin{theorem}
 Let \(K \subset \CC\) be a finitely generated field.
 The category~\(M_{K}\) is a Tannakian subcategory of \(\Mot_{K}\).
\end{theorem}
\begin{proof}
 \pf\
 The category~\(\Mot_{K}\) is semisimple,
 so subquotients are direct summands.
 It is clear that the subcategory~\(\mathcal{M}_{K}\)
 is closed under duals, tensor powers, and direct summands.
 Let \(M_{1}\) and~\(M_{2}\) be two objects in~\(\mathcal{M}_{K}\).
 We need to show that \(M_{1} \oplus M_{2}\)
 and \(M_{1} \otimes M_{2}\) are objects in \(\mathcal{M}_{K}\).
 Observe that \(M_{1} \otimes M_{2}\) is a direct summand of
 \((M_{1} \oplus M_{2})^{\otimes 2}\).
 Thus we are done if we show that the Mumford--Tate conjecture is true for
 \(M = M_{1} \oplus M_{2}\).
 By foobar?? we may assume that \(M\) is hyperadjoint.
 Write \(M = \bigoplus_{i \in I} M'_{i}\),
 with the \(M'_{i}\) hyperadjoint.
 Now the result follows from \cref{intermed_sums}.
 \qed
\end{proof}

\begin{corollary}
 Let \(A_{1}\),~\(A_{2}\), \dots, \(A_{n}\) be abelian varieties over~\(K\).
 If the Mumford--Tate conjecture is true for \(A_{i}\), for \(i = 1,\dots,n\),
 then the Mumford--Tate conjecture is true for
 \(A_{1} \times A_{2} \times \cdots \times A_{n}\).
\end{corollary}


\printbibliography

\end{document}
