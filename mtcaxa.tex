\documentclass[10pt,twoside,leqno]{article}

\usepackage[utf8]{inputenc}
\usepackage[T1]{fontenc}
\usepackage[full]{textcomp}

\usepackage{csquotes}
\usepackage[english]{babel}

% \usepackage[urw-garamond,expert,
% uppercase=upright,greeklowercase=upright]{mathdesign}
% \usepackage[osf,swashQ]{garamondx}
% \def\kappa{\varkappa}
\usepackage{mathtools}
\mathtoolsset{mathic} % Italic correction before mathmode, works with ~'s.
% \def\mathds{\mathbb}

\usepackage{cfr-lm}
\usepackage{dsfont}  % disable this when loading mathdesign

\usepackage{microtype}
\linespread{1.25}  % = 1.500 * fontheight
% \linespread{1.388} % = 1.666 * fontheight
\usepackage[
% paper=b5paper,
nohead,nomarginpar,
% bindingoffset=.3cm,
paper=a4paper,
]{geometry}

% \raggedbottom

\usepackage{lastpage}
\usepackage{fancyhdr}
\pagestyle{fancy}
\fancyhf{}
\renewcommand{\headrulewidth}{0pt}
\fancyfoot[LE,RO]{\thepage/\pageref{LastPage}}

\usepackage{longtable}
\usepackage{booktabs}
\usepackage{tabu}

\usepackage[inline]{enumitem}
\setlist{noitemsep,nosep,listparindent=\parindent}
\setlist[itemize]{label=\guillemotright}
\setlist[enumerate,1]{ref=\thesubsection.\arabic*}
\setlist[enumerate,2]{label=\alph*.,ref=\theenumi.\alph*}
% \setlist[enumerate*]{label=(\textit{\roman*}\thinspace)}

\usepackage{cjhebrew}

\usepackage[backend=biber,doi=false,url=false,isbn=false,%
sorting=nyt,safeinputenc]{biblatex}
\bibliography{\jobname.bib}
\defbibheading{bibliography}[\bibname]{\sectionstar{#1}}

%%% SECTION HEADINGS
\usepackage{ifthen}
\makeatletter
\renewcommand{\part}[1]{%
 \cleardoublepage%
 \vbox{\null\vskip90pt%
 \normalfont\fontsize{20pt}{30pt}\selectfont%
 \baselineskip=30pt%
 \scshape\noindent\textls*{#1}\par}%
 \addcontentsline{toc}{part}{#1}%
 \@afterindentfalse%
 \@afterheading%
}
\renewcommand{\section}[1]{%
 \refstepcounter{section}%
 \bigskip\penalty-250
 \vbox{\normalfont\fontsize{12pt}{15pt}\selectfont%
  \centering\scshape\noindent\textls*{\thesection\quad#1}%
  \par}
 \nobreak
 \addcontentsline{toc}{section}{\protect\numberline{\thesection} #1}%
 \@afterindentfalse%
 \@afterheading%
} 
\newcommand{\sectionstar}[1]{%
 \bigskip\penalty-250
 \vbox{\normalfont\fontsize{12pt}{15pt}\selectfont%
  \centering\scshape\noindent\textls*{#1}%
  \par}
 \nobreak\vskip15pt
 \@afterindentfalse%
 \@afterheading%
} 
\renewcommand{\paragraph}[1]{\par\bigskip\refstepcounter{subsection}%
 {\normalfont\normalsize\scshape\noindent\thesubsection%
 \ifthenelse{\equal{#1}{}}%
 {}%
 {\ \textls{#1.}}%
 \ ---}%
}
\newcommand{\readme}{\par\vskip\baselineskip%
 {\normalfont\normalsize\scshape\noindent%
  \textls{Readme.}\ ---}
}
\renewcommand\tableofcontents{%
 \sectionstar{\contentsname}%
 \@starttoc{toc}%
}
\renewcommand*\l@part[2]{%
 \addvspace{15pt \@plus\p@}%
 \noindent{\leavevmode%
  \scshape\textls{#1\qquad#2}%
 }\par\nobreak%
}
\renewcommand*\l@section[2]{%
 \setlength\@tempdima{\parindent}%
 \noindent
 {\leavevmode%
  \hskip\parindent#1\qquad#2%
 }\par\nobreak%
}
\makeatother

%%% MATH PACKAGES
\usepackage{amsmath,amssymb}  % disable when using mathdesign
\usepackage{mathrsfs}         % disable when using mathdesign
\usepackage{mathabx}
\usepackage[all]{xy}

\usepackage[thmmarks,amsmath]{ntheorem}
\usepackage{thmtools}

\numberwithin{equation}{subsection}

\declaretheoremstyle[headformat=swapnumber,headpunct={.\ ---},%
headfont=\normalfont\scshape\lsstyle,bodyfont=\itshape,%
spaceabove=0pt,spacebelow=0pt,%
preheadhook={\bigskip}]{theorem}
\declaretheorem[style=theorem,sibling=subsection]{theorem}
\declaretheorem[style=theorem,sibling=subsection]{proposition}
\declaretheorem[style=theorem,sibling=subsection]{lemma}
\declaretheorem[style=theorem,sibling=subsection]{corollary}
\declaretheorem[style=theorem,sibling=subsection]{conjecture}

\declaretheoremstyle[headformat=swapnumber,headpunct={.\ ---},%
headfont=\normalfont\scshape\lsstyle,bodyfont=\normalfont,%
spaceabove=0pt,spacebelow=0pt,%
preheadhook={\bigskip}]{definition}
\declaretheorem[style=definition,sibling=subsection]{definition}
\declaretheorem[style=definition,sibling=subsection]{exercise}
\declaretheorem[style=definition,sibling=subsection]{example}
\declaretheorem[style=definition,sibling=subsection]{remark}
\declaretheorem[style=definition,sibling=subsection]{construction}

% \declaretheoremstyle[headpunct={\!.},headfont=\itshape,bodyfont=\normalfont,%
% qed=\ensuremath{\square},spaceabove=0pt,spacebelow=0pt]{proof}
% \declaretheoremstyle[headpunct={\!.},headfont=\itshape,bodyfont=\normalfont,%
% qed=\ensuremath{\square},spaceabove=0pt,spacebelow=0pt]{nonumberproof}
% \declaretheorem[style=proof,numbered=no]{proof}

\declaretheoremstyle[headformat=swapnumber,headpunct={.\ ---},%
headfont=\itshape,bodyfont=\normalfont,qed=\ensuremath{\square},%
spaceabove=0pt,spacebelow=0pt,%
preheadhook={\bigskip}]{nproof}
\declaretheorem[style=nproof,sibling=subsection,name=Proof]{nproof}

\let\qed\relax
\usepackage{pf2}
\pfkeywords{jmc}

\usepackage{cleveref}
\crefname{condition}{condition}{conditions}
\crefname{conjecture}{conjecture}{conjectures}
\crefname{construction}{construction}{constructions}
\crefname{corollary}{corollary}{corollaries}
\crefname{diagram}{diagram}{diagrams}
\crefformat{subsection}{\S#2#1#3}
\crefformat{enumi}{\S#2#1#3}
\crefformat{nproof}{\S#2#1#3}
\creflabelformat{equation}{#2#1#3}

%%% MATH MACROS
\newcommand{\id}{\textnormal{id}}

\newcommand{\into}{\hookrightarrow}
\newcommand{\onto}{\twoheadrightarrow}
\newcommand{\longto}{\longrightarrow}
\newcommand{\longinto}{\lhook\joinrel\longrightarrow}

\renewcommand{\Im}{\textnormal{Im}}

\newcommand{\colim}{\mathop{\textnormal{colim}}}

\newcommand{\Hom}{\textnormal{Hom}}
\newcommand{\End}{\textnormal{End}}
\newcommand{\Isom}{\textnormal{Isom}}
\newcommand{\Inn}{\textnormal{Inn}}
\newcommand{\Aut}{\textnormal{Aut}}
\newcommand{\Out}{\textnormal{Out}}
\newcommand{\iHom}{\underline{\Hom}}
\newcommand{\iEnd}{\underline{\End}}
\newcommand{\iIsom}{\underline{\Isom}}
\newcommand{\iInn}{\underline{\Inn}}
\newcommand{\iAut}{\underline{\Aut}}
\newcommand{\iOut}{\underline{\Out}}

\newcommand{\Mat}{\textnormal{Mat}}
\newcommand{\Sym}{\textnormal{Sym}}

\newcommand{\Fil}{\textnormal{Fil}}

\newcommand{\dual}[1]{\check{#1}}

\newcommand{\NN}{\mathbb{N}}
\newcommand{\ZZ}{\mathbb{Z}}
\newcommand{\QQ}{\mathbb{Q}}
\newcommand{\QQl}{\QQ_{\ell}}
\newcommand{\QQlbar}{\bar{\QQ}_{\ell}}
\newcommand{\QQp}{\QQ_{p}}
\newcommand{\QQpbar}{\bar{\QQ}_{p}}
\newcommand{\RR}{\mathbb{R}}
\newcommand{\CC}{\mathbb{C}}
\newcommand{\HQ}{\mathbb{H}}
\newcommand{\FF}{\mathbb{F}}
\newcommand{\FFp}{\FF_{p}}
\newcommand{\FFq}{\FF_{q}}
\newcommand{\FFqbar}{\bar{\FF}_{q}}
\newcommand{\Adele}{\mathbb{A}}
\newcommand{\fin}{\textnormal{f}}

\newcommand{\primes}{\mathscr{L}}

\newcommand{\Spec}{\textnormal{Spec}}

\newcommand{\DelS}{\mathbb{S}}
\newcommand{\Sh}{\textnormal{Sh}}
\newcommand{\mSh}{\mathscr{S}}
\newcommand{\DD}{\mathbb{D}}
\newcommand{\mcG}{\mathcal{G}}
\newcommand{\Kmpt}{\mathcal{K}}
\newcommand{\Sds}{\mathfrak{H}^{\pm}}
\newcommand{\AV}{\mathscr{A}}
\newcommand{\Ag}{\AV_{g}}

\newcommand{\Gal}{\textnormal{Gal}}

% \newcommand{\HH}{\textnormal{H}}
% \newcommand{\Hhom}{\HH_{\textnormal{hom}}}
\newcommand{\HdR}{\HH_{\dR}}
% \newcommand{\Hl}{\HH_{\ell}}
% \newcommand{\Hp}{\HH_{p}}
% \newcommand{\Hlambda}{\HH_{\lambda}}
% \newcommand{\HB}{\HH_{\textnormal{B}}}
% \newcommand{\Hsigma}{\HH_{\sigma}}

% \newcommand{\GG}{\textnormal{G}}
% \newcommand{\Gl}{\GG_{\ell}}
% \newcommand{\Glc}{\Gl^{\circ}}
% \newcommand{\GB}{\GG_{\textnormal{B}}}
% \newcommand{\Gsigma}{\GG_{\sigma}}
% \newcommand{\Gmot}[1]{\GG_{\textnormal{mot},#1}}
% \newcommand{\Gmots}{\Gmot{\sigma}}
% \newcommand{\Gmotl}{\Gmot{\ell}}
% \newcommand{\GmotB}{\Gmot{\textnormal{B}}}
% \newcommand{\Gp}{\GG_{p}}

% \newcommand{\Zl}{\textnormal{Z}_{\ell}}
% \newcommand{\Zlc}{\Zl^{\circ}}
% \newcommand{\ZB}{\textnormal{Z}_{\textnormal{B}}}
% \newcommand{\Zsigma}{\textnormal{Z}_{\sigma}}
% \newcommand{\Zmot}[1]{\textnormal{Z}_{\textnormal{mot},#1}}
% \newcommand{\Zmots}{\Zmot{\sigma}}
% \newcommand{\Zmotl}{\Zmot{\ell}}

\newcommand{\BdR}[1]{\textnormal{B}_{\dR,#1}}
\newcommand{\BHT}[1]{\textnormal{B}_{\textnormal{HT},#1}}
\newcommand{\gr}{\textnormal{gr}}

\newcommand{\Vect}{\textnormal{Vect}}
\newcommand{\Filt}{\textnormal{Filt}}
\newcommand{\Rep}{\textnormal{Rep}}
\newcommand{\QHS}{\QQ\textnormal{HS}}

\makeatletter
\def\cpwith[#1]#2{\textnormal{c.p.}_{#1}(#2)}
\def\cpwithout#1{\textnormal{c.p.}(#1)}
\def\cp{\@ifnextchar[{\cpwith}{\cpwithout}}
\makeatother
\makeatletter
\def\Gmwith[#1]{\mathbb{G}_{\textnormal{m},#1}}
\def\Gmwithout{\mathbb{G}_{\textnormal{m}}}
\def\Gm{\@ifnextchar[{\Gmwith}{\Gmwithout}}
\makeatother
\newcommand{\GL}{\textnormal{GL}}
\newcommand{\SL}{\textnormal{SL}}
\newcommand{\PGL}{\textnormal{PGL}}
\newcommand{\UU}{\textnormal{U}}
\newcommand{\SU}{\textnormal{SU}}
\newcommand{\PU}{\textnormal{PU}}
\newcommand{\OO}{\textnormal{O}}
\newcommand{\SO}{\textnormal{SO}}
\newcommand{\Spin}{\textnormal{Spin}}
\newcommand{\CSpin}{\textnormal{CSpin}}
\newcommand{\PSpin}{\textnormal{PSpin}}
\newcommand{\CSp}{\textnormal{CSp}}
\newcommand{\Lie}{\textnormal{Lie}}

% \newcommand{\mfsl}{\mathfrak{sl}}
% \newcommand{\mfso}{\mathfrak{so}}

% \newcommand{\Cliff}{\textnormal{Cl}}
% \newcommand{\spin}{\textnormal{spin}}

% \newcommand{\St}{\textnormal{St}}

\newcommand{\ab}{\textnormal{ab}}
\newcommand{\der}{\textnormal{der}}
\newcommand{\ad}{\textnormal{ad}}

\newcommand{\SmPr}{\textnormal{SmPr}}

\newcommand{\an}{\textnormal{an}}
\newcommand{\cl}{\textnormal{cl}}

\newcommand{\dR}{\textnormal{dR}}
\newcommand{\et}{\textnormal{\'{e}t}}
\newcommand{\sing}{\textnormal{sing}}

\newcommand{\HH}{\textnormal{H}}
\newcommand{\Hl}{\HH_{\ell}}
\newcommand{\Hp}{\HH_{p}}
\newcommand{\Hlambda}{\HH_{\lambda}}
\newcommand{\HB}{\HH_{\textnormal{B}}}

\newcommand{\Zar}{\textnormal{Zar}}

\newcommand{\Mot}{\textnormal{Mot}}

\newcommand{\GG}{\textnormal{G}}
\newcommand{\GB}{\GG_{\textnormal{B}}}
\newcommand{\Gp}{\GG_{p}}
\newcommand{\Gl}{\GG_{\ell}}

\newcommand{\alg}{\textnormal{alg}}
\newcommand{\tra}{\textnormal{tra}}

\newcommand{\Res}{\textnormal{Res}}
\newcommand{\Nm}{\textnormal{Nm}}
\newcommand{\trace}{\textnormal{tr}}
\newcommand{\rk}{\textnormal{rk}}
\renewcommand{\det}{\textnormal{det}}
\newcommand{\res}{\textnormal{res}}

\newcommand{\chrc}{\textnormal{char}}

\newcommand{\Tangen}[1]{\langle #1 \rangle^{\otimes}}
\newcommand{\Val}{\textnormal{Val}}

\newcommand{\tr}{\textsc{tr}}
\newcommand{\cm}{\textsc{cm}}

\newcommand{\MTC}{\textnormal{MTC}}

\newcommand{\rotatesim}{\rotatebox{90}{$\sim$}}

\usepackage{tikz}
\usetikzlibrary{cd,positioning}
\usetikzlibrary{decorations.pathmorphing}
\usetikzlibrary{decorations.markings}

%%%%%%%%%%%%%% Macros for Dynkin diagrams %%%%%%%%%%%%%%
\newcommand{\dynkinradius}{.04cm}
\newcommand{\dynkinstep}{.35cm}
\newcommand{\dynkinnode}[2]{\fill (\dynkinstep*#1,\dynkinstep*#2) circle (\dynkinradius);}
\newcommand{\dynkinXsize}{1.5}
\newcommand{\dynkinnodespecial}[2]{
 \draw[thick] (#1*\dynkinstep-\dynkinXsize,#2*\dynkinstep-\dynkinXsize) -- (#1*\dynkinstep+\dynkinXsize,#2*\dynkinstep+\dynkinXsize);
 \draw[thick] (#1*\dynkinstep-\dynkinXsize,#2*\dynkinstep+\dynkinXsize) -- (#1*\dynkinstep+\dynkinXsize,#2*\dynkinstep-\dynkinXsize);
}
\newcommand{\dynkinedge}[4]{\draw[thin] (\dynkinstep*#1,\dynkinstep*#2) -- (\dynkinstep*#3,\dynkinstep*#4);}
\newcommand{\dynkinnodes}[4]{\draw[dotted] (\dynkinstep*#1,\dynkinstep*#2) -- (\dynkinstep*#3,\dynkinstep*#4);}
\newcommand{\dynkindoubleedge}[4]{\draw[double,postaction={decorate}] (\dynkinstep*#1,\dynkinstep*#2) -- (\dynkinstep*#3,\dynkinstep*#4);}

\newenvironment{dynkin}{\begin{tikzpicture}[decoration={markings,mark=at position 0.7 with {\arrow{>}}}]}
 {\end{tikzpicture}}
%%%%%%%%%%%%%% End of macros for Dynkin diagrams %%%%%%%%%%%%%%


\def\title{The Mumford--Tate conjecture for products of abelian varieties}
\def\author{Johan Commelin}

\usepackage{datetime}
\def\date{\dayofweekname{\day}{\month}{\year},
 the \ordinaldate{\day} of \monthname, \number\year}

\begin{document}
\begin{center}\Large\scshape
\textls*{\title}
\end{center}

\medskip

\noindent\textit{by} \quad \author \hfill \date

\vskip3\baselineskip

\tableofcontents

\section{Introduction}

The strategy:
\begin{enumerate}
 \item Reduce to the case of hyperadjoint motives, \(M_{1}\) and \(M_{2}\).
 \item Show that if \(\Gl(M_{1} \oplus M_{2}) \subsetneq \Gl(M_{1}) \times \Gl(M_{2})\),
  then we have \(\HH_{\lambda_{1}}(M_{1}) \cong \HH_{\lambda_{2}}(M_{2})\)
  and therefore \(\End(M_{1}) = E = \End(M_{2})\)
  and also \(\HH_{\Lambda}(M_{1}) \cong \HH_{\Lambda}(M_{2})\).
 \item Deduce that \(\GB(M_{1})\) and \(\GB(M_{2})\) are isomorphic
  at all finite and infinite places of~\(E\).
 \item Conclude that \(\GB(M_{1}) \cong G \cong \GB(M_{2})\).
 \item Let \((G,X_{i})\) be the Shimura datum attached to \(M_{i}\).
 \item Fix one infinite place \(v_{0}\) of \(E\),
  such that \(G_{v_{0}}(\RR)\) is non-compact.
 \item Act on \(X_{2}\) by \(\Out(G)\) so that
  special node on the Dynkin diagram of~\(G_{v_{0}}\)
  corresponding to the factor \((G_{v_{0}},X_{v_{0},2}\)
  is the same as the special node
  corresponding to the factor \((G_{v_{0}},X_{v_{0},1}\).
 \item Show (TODO) that the reflex field \(E(G,X_{1})\)
  is the same as \(E(G,X_{2})\).
  \begin{enumerate}
   \item Idea: look locally to see how \(\Gal(\QQlbar/\QQl)\)
    acts on the set of special nodes.
   \item Combine all this \(\ell\)-adic data to determine
    the reflex field.
  \end{enumerate}
 \item Conclude that the special nodes on all components of the Dynkin diagram agree.
 \item Conclude that \((G,X_{1}) = (G,X) = (G,X_{2})\).
 \item Left-over: prove the MTC for the product of two fibres on a connected Shimura variety,
  given that MTC is true for the two factors.
 \item Do this by lifting to a Shimura variety of Hodge type.
 \item Use that MTC is known on centres, and reduce to Faltings.
\end{enumerate}

TODO %TODO
\hfill\cjRL{hw/s`nh brwK hb' b/sM yhwh}

\paragraph{Conventions}
We say that a reductive group~\(G\) over a field~\(K\) of characteristic~\(0\)
is \emph{classical} if all its simple factors
are of type \(A_{n}\),~\(B_{n}\), \(C_{n}\), or~\(D_{n}\).
(In other words, we exclude the exceptional groups.)

\section{A local-global principle for forms of adjoint algebraic groups}

\begin{theorem} \label{locglob}
 Let \(E\) be a number field.
 Let \(G\) be an absolutely simple classical group over~\(E\).
 Assume that \(G\) is either simply connected or adjoint.
 Then the natural morphism
 \[
  \HH^{1}(E, \iAut_{G}) \stackrel{\phi_{G}}{\longto}
  \prod_{v} \HH^{1}(E_{v}, \iAut_{G})
 \]
 is injective.
 (Here \(v\) runs over all the places of~\(E\), both finite and infinite.)
\end{theorem}

\paragraph{First reduction} \label{firstred}
The proof of \cref{locglob} takes the rest of this section.
We first show that we may assume that \(G\) is split and simply connected.
Let \(G'\) be a form of~\(G\) over~\(E\),
corresponding to a class \([G'] \in \HH^{1}(E, \iAut_{G})\).
Then we obtain a natural commutative diagram
\[
 \begin{tikzcd}
  \HH^{1}(E, \iAut_{G}) \ar[r,"\phi_{G}"] \ar[d,<->,"\rotatesim"]
  & \prod_{v} \HH^{1}(E_{v}, \iAut_{G}) \ar[d,<->,"\rotatesim"] \\
  \HH^{1}(E, \iAut_{G'}) \ar[r,"\phi_{G'}"]
  & \prod_{v} \HH^{1}(E_{v}, \iAut_{G'}) \\
 \end{tikzcd}
\]
where the downward map on the left sends \([G']\)
to the distinguished element \(*\) of \(\HH^{1}(E, \iAut_{G})\);
and the downward map on the right sends \(\phi_{G}([G'])\)
to \(* \in \prod_{v} \HH^{1}(E_{v}, \iAut_{G})\).
It is therefore sufficient to prove \cref{locglob}
in the case that \(G\) is a split group.
If \(G\) is simply connected, then \(\iAut_{G} \cong \Aut_{G^{\ad}}\),
and thus the adjoint case follows from the simply connected case.

\paragraph{} \label{exactEpts}
For the moment, let \(E\) be any field of characteristic~\(0\)
(for example a \(p\)-adic local field); and
let \(G\) be a classical group over~\(E\) that is split and simply connected.
Let \(\Delta\) denote the Dynkin diagram of~\(G_{\bar{E}}\),
and let \(\Delta_{\bar{E}}\) denote the Dynkin diagram of~\(G_{\bar{E}}\).
Recall that \(\iOut_{G}(\bar{E}) \cong \Aut(\Delta)\).
The assumption that \(G\) is split and simply connected
implies that
\begin{enumerate*}[label=(\textit{\roman*})]
 \item \(\iOut_{G}(E) = \iOut_{G}(\bar{E})\); and
 \item the sequence \(0 \to \iInn_{G}(E) \to \iAut_{G}(E) \to \iOut_{G}(E) \to 0\)
  is exact.
\end{enumerate*}
See for example proposition~1.5.1 and example~1.5.2 of~\cite{Conrad_RedGrp}.

\paragraph{}
For ease of mind we recall that
\[
 \Aut(\Delta) =
 \begin{cases}
  1 & \text{if \(G\) is of type \(A_{1}\)} \\
  \ZZ/2\ZZ & \text{if \(G\) is of type \(A_{n}\), with \(n > 1\)} \\
  1 & \text{if \(G\) is of type \(B_{n}\) or~\(C_{n}\)} \\
  \mathfrak{S}_{3} & \text{if \(G\) is of type \(D_{4}\)} \\
  \ZZ/2\ZZ & \text{if \(G\) is of type \(D_{n}\), with \(n > 4\).}
 \end{cases}
\]

\paragraph{} \label{phiDelta}
We now return to the setting of \cref{locglob} and~\cref{firstred}:
\(E\) is a number field,
and \(G\) is a classical group over~\(E\) that is split and simply connected.
Recall from \cref{exactEpts} that
the action of \(\Gal(\bar{E}/E)\) on~\(\Aut(\Delta)\) is trivial.
Therefore \(\HH^{1}(E, \Aut(\Delta)) = \Hom(\Gal(\bar{E}/E), \Aut(\Delta))\),
which classifies Galois extensions of~\(E\)
whose Galois group is a subgroup of~\(\Aut(\Delta)\).
Let \(F/E\) be a Galois extension, and let~\(v\) be any place of~\(E\).
Recall that \(F \otimes E_{v} \cong \prod_{w|v} F_{w}\),
and the Galois group acts transitively on the factors~\(F_{w}\).
Consider the map
\(\phi_{\Delta} \colon
 \HH^{1}(E, \Aut(\Delta)) \longto \prod_{v} \HH^{1}(E_{v}, \Aut(\Delta))\),
which maps a Galois extension \(F/E\) to \((F_{w}/E_{v})_{v}\).
Hence the map \(\phi_{\Delta}\) is injective,
since a Galois extension \(F/E\) is trivial
if and only if it is totally split at all places of~\(E\).

\paragraph{}
Recall the short exact sequence
\( 0 \to \iInn_{G} \to \iAut_{G} \to \iOut_{G} \to 0 \).
Now consider the following commutative diagram of exact sequences
\[
 \begin{tikzcd}[column sep=1em]
    \HH^{0}(E,\iOut_{G}) \ar[r] \ar[d]
  & \HH^{1}(E,\iInn_{G}) \ar[r,"\beta"]  \ar[d,"\psi"]
  & \HH^{1}(E,\iAut_{G}) \ar[r,"\gamma"] \ar[d,"\phi_{G}"]
  & \HH^{1}(E,\iOut_{G}) \ar[d,"\phi_{\Delta}",hook] \\
    \prod_{v} \HH^{0}(E_{v},\iOut_{G}) \ar[r,"\delta"]
  & \prod_{v} \HH^{1}(E_{v},\iInn_{G}) \ar[r,"\varepsilon"]
  & \prod_{v} \HH^{1}(E_{v},\iAut_{G}) \ar[r]
  & \prod_{v} \HH^{1}(E_{v},\iOut_{G}) \\
 \end{tikzcd}
\]
Recall that \cref{phiDelta} shows that the map \(\phi_{\Delta}\) is injective.
We claim that
the map \(\delta\) is trivial; and
the map \(\psi\) is injective.
Recall from \cref{exactEpts} that the sequences
\( 0 \to \iInn_{G}(E_{v}) \to \iAut_{G}(E_{v}) \to \iOut_{G}(E_{v}) \to 0 \)
are exact. This shows that the map \(\delta\) is trivial.
The map \(\psi\) is injective by theorem~6.22 on page~336 of~\cite{PlaRap},
since \(\iInn_{G} = G^{\ad}\).

We can now finish the proof of \cref{locglob} by a diagram chase.
Let \(x \in \HH^{1}(E, \iAut_{G})\) be a class such that
\(\phi_{G}(x) = *\).
Then \(\phi_{\Delta} \circ \gamma(x) = *\),
and since \(\phi_{\Delta}\) is injective we conclude that \(\gamma(x) = *\).
Therefore \(x = \beta(y)\) for some class \(y \in \HH^{1}(E, \iInn_{G})\).
Since \(\delta\) is trivial and \(\psi\) is injective,
we find that \(\varepsilon \circ \psi\) is injecitve.
Because \(\varepsilon \circ \psi(y) = \phi_{G} \circ \beta(y) = *\),
we conclude that \(y = *\), and hence \(x = *\).
This completes the proof of \cref{locglob}.

\section{Recap: classification of real adjoint Shimura data}

\readme
Deligne completely describes all pairs~\((G,X)\), where
\(G\) is a simple adjoint real algebraic group, and
\(X\) is a conjugacy class of homomorphisms
\(h \colon \DelS \to G\) satisfying the following conditions:
\begin{enumerate*}[label=(\kern-.1pt\textit{\roman*}\kern.6pt)]
 \item the adjoint representation~\(\Lie(G)\) is of type
  \(\{(-1,1),(0,0),(1,-1)\}\);
 \item conjugation by \(h(i)\) is a Cartan involution; and
 \item \(h\) is non-trivial.
\end{enumerate*}
What follows is a partial translation of~\S0 and~\S1 of~\cite{Del_ShimVar}
to recapitulate concepts and notation.
Each paragraph indicates to which paragraph of~\cite{Del_ShimVar} it corresponds
and intermezzos are clearly specified as such.

\paragraph{Terminology and notation (parts of~\S0 of~\cite{Del_ShimVar})}
In this section \emph{reductive group}
always means \emph{connected reductive group}.
A \emph{covering} of a reductive group is a \emph{connected} covering.
\emph{Adjoint} group means \emph{reductive adjoint} group.
If \(G\) is a reductive group, we denote with \(G^{\ad}\) its adjoint group,
with \(G^{\der}\) its derived subgroup, and with
\(\rho \colon \tilde{G} \to G^{\der}\) the universal covering of~\(G^{\der}\).
We always denote with~\(Z\) (or~\(Z(G)\)) the centre of~\(G\),
and (conflict of notation) with \(\tilde{Z}\) that of~\(\tilde{G}\).

With a superscript~\({}^{\circ}\) we denote a
\emph{algebraic connected component}
(for example: \(Z^{\circ}\) is
the connected component of the identity of the centre~\(Z\) of~\(G\)).
The superscript~\({}^{+}\) denotes a
\emph{topological connected component}
(for example: \(G(\RR)^{+}\) is the topological connected component of the
identity of the real points of a group~\(G\)).
We also denote with \(G(\QQ)^{+}\) the trace of~\(G(\RR)^{+}\) on~\(G(\QQ)\).
For \(G\) real reductive, we denote with a subscript \({}_{+}\)
the inverse image of \(G^{\ad}(\RR)^{+}\) in~\(G(\RR)\).
The same notation \({}_{+}\) for the trace on the rational points of a group.

If \(X\) is a topological space,
then we denote with \(\pi_{0}(X)\) the set of its connected components,
endowed with the quotient topology of that on~\(X\).

A \emph{hermitian symmetric domain} is a hermitian symmetric space
with curvature \(< 0\) (that is, without euclidean or compact factors).

If not expressly mentioned otherwise,
a \emph{vector space} is supposed to be finite-dimensional,
and a \emph{number field} is supposed to be of finite degree over~\(\QQ\).

\paragraph{Moduli spaces of Hodge structures
 (parts of~\S1.1 of~\cite{Del_ShimVar})}
Recall that a Hodge structure on a real vector space~\(V\)
is a bigrading \(V_{\CC} = \bigoplus V^{pq}\) of the complexification of~\(V\),
such that \(V^{pq}\) is the complex conjugate of~\(V^{qp}\).

Define an action \(h\) of~\(\CC^{*}\) on~\(V_{\CC}\) by the formula
\[
 h(z)v = z^{-p}\bar{z}^{-q}v \qquad \text{for \(v \in V^{pq}\).}
\]
The \(h(z)\) commute with complex conjugation on~\(V_{\CC}\),
and thus are deduced from the extension of scalars of an action,
again denoted~\(h\), of~\(\CC^{*}\) on~\(V\).
Regard \(\CC\) as an extension of~\(\RR\),
and let~\(\DelS\) be its multiplicative group,
regarded as real algebraic group
(in other words, \(\DelS = \Res_{\CC/\RR}\Gm\)
(restriction of scalars in the sense of Weil)).
One has \(\DelS(\RR) = \CC^{*}\),
and \(h\) is an action of the algebraic group~\(\DelS\).
One verifies that this construction defines an equivalence of categories:
\(\{\)real vector spaces endowed with a Hodge structure
\(\} \to \{\)
real vector spaces endowed with an action of the algebraic group~\(\DelS\}\).

The inclusion \(\RR^{*} \subset \CC^{*}\) corresponds to an inclusion
of real algebraic groups \(\Gm \subset \DelS\).
We denote with~\(w_{h}\) (or simply~\(w\))
the restriction of~\(h^{-1}\) to~\(\Gm\),
and call \(w \colon \Gm \to \GL(V)\) the \emph{weights}.
One says that~\(V\) is pure of weight~\(n\)
if \(V^{pq} = 0\) for \(p + q \ne n\),
that is, \(w(\lambda)\) is multiplication by the scalar~\(\lambda^{n}\).

We denote with \(\mu_{h}\) (or simply~\(\mu\))
the action of~\(\Gm\) on~\(V_{\CC}\)
defined by \(\mu(z)v = z^{-p}v\) for \(v \in V^{pq}\).
It is a composition \(\Gm \to \DelS_{\CC} \stackrel{h}{\to} \GL(V_{\CC})\).

The \emph{Hodge filtration} \(F_{h}\) (or simply~\(F\)) is defined by
\(F^{p} = \bigoplus_{r \ge p} V^{rs}\).
One says that \(V\) is of type \(\mathscr{E} \subset \ZZ \times \ZZ\)
if \(V^{pq} = 0\) for \((p,q) \notin \mathscr{E}\).

More generally, if \(A\) is a subring of~\(\RR\)
such that \(A \otimes \QQ\) is a field
(in practice, \(A = \ZZ\),~\(\QQ\), or~\(\RR\)),
an \emph{\(A\)-Hodge structure}
is an \(A\)-module~\(V\) of finite type,
endowed with a Hodge structure on \(V \otimes_{A} \RR\).

\paragraph{(\S1.1.10 of~\cite{Del_ShimVar})}
A \emph{polarisation} of a real Hodge structure~\(V\) of weight~\(n\)
is a morphism \(\Psi \colon V \otimes V \to \RR(-n)\)
such that the form \((2\pi i)^{n} \Psi(x, h(i)y)\)
is symmetric and positive definite.
The same for \(\ZZ\)-Hodge structures,
upon replacing \(\RR(-n)\) with \(\ZZ(-n)\), \dots.
Since \(\Psi(h(i)x, y) = \Psi(x, h(-i)y)\)
(indeed, \(h(i)\) is trivial on~\(\RR(-n)\)),
and \(h(-i)y = (-1)^{n}h(i)y\),
the condition of symmetry is equivalent to:
\(\Psi\) is symmetric if \(n\) is even,
alternating if \(n\) is odd.

Hodge structure that come from algebraic geometry
are \(\ZZ\)-Hodge structures that are pure and polarisable.
Fundamental example:
the Hodge positivity theorems assure that \(\HH^{n}(X, \ZZ)\),
for \(X\) a smooth projective variety, is polarisable.

\paragraph{(From the proof of proposition~1.1.14 of~\cite{Del_ShimVar})}
Recall that a \emph{Cartan involution}
of a (not necessarily connected) real linear algebraic group~\(G\)
is an involution~\(\sigma\) of~\(G\)
such that the real form~\(G^{\sigma}\) of~\(G\)
(with complex conjugation \(g \mapsto \sigma(\bar{g})\))
is compact:
\(G^{\sigma}(\RR)\) is compact
and intersects all the connected components of \(G^{\sigma}(\CC) = G(\CC)\).

\paragraph{Classification (parts of~\S1.2 of~\cite{Del_ShimVar})} \label{3cond}
Consider systems \((G,X)\) of an adjoint simple real algebraic group~\(G\)
and a \(G(\RR)\)-conjugacy class~\(X\) of
real algebraic morphisms \(h \colon \DelS \to G\) satisfying:
\begin{enumerate*}[label=(\kern-.1pt\textit{\roman*}\kern.6pt)]
 \item the adjoint representation~\(\Lie(G)\) is of type
  \(\{(-1,1),(0,0),(1,-1)\}\)
  (in particular, \(h\) is trivial on \(\Gm \subset \DelS\));
 \item conjugation by \(h(i)\) is a Cartan involution; and
 \item \(h\) is non-trivial.
\end{enumerate*}
By \S1.1.17 of \cite{Del_ShimVar}, the connected components of the spaces~\(X\)
are irreducible hermitian symmetric domains.
Hypothesis (\textit{ii}) assures that the Cartan involutions of~\(G\)
are inner automorphisms, and hence \(G\) is an inner form of its compact form.
In particular, \(G\) being simple, is absolutely simple.

The \(G(\CC\)-conjugacy class of \(\mu_{h} \colon \Gm \to G_{\CC}\)
does not depend on the choice of \(h \in X\).
We denote it with~\(M_{X}\).

\begin{proposition}[1.2.2 of~\cite{Del_ShimVar}] \label{Del122}
 Let \(G_{\CC}\) be a simple adjoint complex algebraic group.
 To each system \((G,X)\) consisting of a
 real form~\(G\) of~\(G_{\CC}\)
 and an~\(X\) as in~\cref{3cond}
 we attach~\(M_{X}\).
 This gives a bijection
 between \(G_{\CC}(\CC)\)-conjugacy classes of pairs \((G,X)\),
 and \(G_{\CC}(\CC)\)-conjugacy classes of
 non-trivial morphisms \(\mu \colon \Gm \to G_{\CC}\)
 that satisfy the following condition:

 \((\star)\)
 only the characters \(z^{-1}\),~\(1\), and~\(z\)
 occur in the representation
 \(\ad \circ \mu\) of~\(\Gm\) on~\(\Lie(G_{\CC})\).
\end{proposition}
\begin{proof}
 \pf\ Omitted; see~\cite{Del_ShimVar}. \qed
\end{proof}

\paragraph{\S1.2.5 of~\cite{Del_ShimVar}}
Let \(G\) be an adjoint simple complex algebraic group.
We enumerate the conjugacy classes of non-trivial morphisms
\(\mu \colon \Gm \to G\) that satisfy condition~\((\star)\) of \cref{Del122},
in termes of the Dynkin diagram~\(\Delta\) of~\(G\).
Recall that the latter is canonically attached to~\(G\)---in particular,
the automorphisms of~\(G\) act on~\(\Delta\)---one can identify nodes
of~\(\Delta\) with conjugacy classes of maximal parabolic subgroups of~\(G\).

Let \(T\) be a maximal torus,
\(X(T) = \Hom(T,\Gm)\),
\(Y(T) = \Hom(\Gm,T)\)
(the dual of~\(X(T)\) under the pairing
\(X(T) \times Y(T) \stackrel{\circ}{\to} \Hom(\Gm,\Gm) = \ZZ\)),
\(R \subset X(T)\) the set of roots,
\(B\) a system of simple roots,
\(\alpha_{0}\) the opposite of the largest root,
and \(B^{+} = B \cup \{\alpha_{0}i\}\).
The nodes of~\(\Delta\) are parameterised by~\(B\),
and those of the extended Dynkin diagram~\(\Delta^{+}\) by~\(B^{+}\).

A conjugacy class of morphisms from \(\Gm\) to \(G\)
has a unique representative \(\mu \in Y(T)\)
in the fundamental Weyl chamber
\(\langle \alpha, \mu \rangle \ge 0\) for \(\alpha \in B\).
It is uniquely determined by the positive integers
\(\langle \alpha, \mu \rangle_{\alpha \in B}\) and,
\(G\) being adjoint,
the may be prescribed arbitrarily.
The condition~\((\star)\) of \cref{Del122}, for \(\mu\) non-trivial,
translates to
\[
 (\star)' \qquad \langle \alpha_{0}, \mu \rangle = -1.
\]

Write the largest root as linear combination of the simple roots,
\(\sum_{\alpha \in B^{+}} n(\alpha)\alpha = 0\), with \(n(\alpha_{0}) = 1\),
and call a node of~\(\Delta^{+}\) \emph{special}
if one has \(n(\alpha) = 1\) for the corresponding root \(\alpha \in B^{+}\).
The quotient of the lattice of coweights by that of the coroots
acts on \(\Delta^{+}\),
and this action is simply transitive on the set of special nodes.
The special nodes are the conjugates under \(\Aut(\Delta^{+})\)
of the node corresponding to~\(\alpha_{0}\).

The condition \((\star)'\) translates to

\((\star)''\) For a simple root \(\alpha \in B\) corresponding to
a special node of~\(\Delta\), one has \(\langle \alpha, \mu \rangle = 1\).
For the other simple roots, \(\langle \alpha, \mu \rangle = 0\).

\paragraph{Intermezzo: the opposition involution}
We continue the notation of the previous paragraph.
Let \(w_{0}\) be the longest element of the Weyl group (with respect to~\(B\)).
Then \(w_{0}(B) = -B\)
and \(-w_{0}\) defines an element of \(\Aut(\Delta) = \Out(G)\):
the \emph{opposition involution}.
This involution is non-trivial if and only if \(\Delta\)
has type \(A_{k}\) with \(k \ne 1\), \(D_{k}\) with \(k\)~odd, or \(E_{6}\).

\paragraph{(\S1.2.6 of~\cite{Del_ShimVar})} \label{specialnode}
In total, the \(G_{\CC}(\CC)\)-conjugacy classes \((G,X)\) as in \cref{Del122}
are parameterised by the special nodes
of the Dynkin diagram~\(\Delta\) of~\(G_{\CC}\).
In particular, for~\(G\) a given real form of~\(G_{\CC}\),
\(X\) is determined by the corresponding special node~\(s(X)\).
The node corresponding to \(X^{-1} = \{h^{-1} \mid h \in X\}\)
is the transform of~\(s(X)\) under the opposition involution.

\begin{proposition}[1.2.7 of~\cite{Del_ShimVar}]
 Let \(G\) be an adjoint simple real algebraic group,
 an suppose that there exist morphisms \(h \colon \CC^{*}/\RR^{*} \to G\)
 satisfying the conditions of~\cref{3cond}.
 Then the set of such morphisms has two connected components,
 exchanged by \(h \mapsto h^{-1}\).
 Either component has the neutral component~\(G(\RR)^{+}\) of~\(G\)
 as its stabiliser.
\end{proposition}
\begin{proof}
 \pf\ Omitted; see~\cite{Del_ShimVar}. \qed
\end{proof}

\begin{corollary}[1.2.8 of~\cite{Del_ShimVar}]
 Let \((G,X)\) be as in~\cref{3cond},
 and \(s\) the corresponding node of the Dynkin diagram of~\(G_{\CC}\).
\begin{enumerate*}[label=(\kern-.1pt\textit{\roman*}\kern.6pt)]
\item If \(s\) is not fixed by the opposition involution,
 then \(G(\RR)\) and~\(X\) are connected.
\item If \(s\) is fixed by the opposition involution,
 then \(G(\RR)\) and~\(X\) have two connected components;
 the components of~\(X\) are exchanged by \(h \mapsto h^{-1}\),
 and by \(g \in G(\RR) - G(\RR)^{+}\).
\end{enumerate*}
\end{corollary}

\paragraph{Symplectic embeddings (\S1.3 of~\cite{Del_ShimVar})}
Let \(V\) be a real vector space,
endowed with a non-degenerate alternating form~\(\Psi\).
The corresponding \emph{Siegel half space}~\(\mathfrak{H}^{+}\)
admits the following description:
it is the space of complex structures~\(h\) on~\(V\),
such that~\(\Psi\) is of type~\(1,1\)
(under the identification of complex structures
and Hodge structures of type \(\{(-1,0),(0,-1)\}\),
\S1.1.3~of~\cite{Del_ShimVar})
and such that the form \(\Psi(x,h(i)y)\) is symmetric and positive definite.
If one replaces ``positive definite'' by ``definite'',
the obtained \emph{double Siegle half space}~\(\Sds\)
is a conjugation class of morphisms \(h \colon \DelS \to \CSp(V)\).

\paragraph{(\S1.3.2 of~\cite{Del_ShimVar})} \label{Del132}
Let \(G\) be an adjoint real algebraic group
and let \(X\) be a conjugacy class of morphisms \(h \colon \DelS \to G\).
Assume that \((G,X)\) satisfies conditions~(\textit{i}) and~(\textit{ii})
of~\cref{3cond}, and replace (\textit{iii}) by

(\textit{iii})' \(G\) does not have a compact factor.

The system \((G,X)\) is then a product of systems
\((G_{\iota},X_{\iota})\) as in~\cref{3cond},
and \(X_{\iota}\) corresponds with a special node
of the Dynkin diagram of~\(G_{\iota\CC}\) (\cref{specialnode}).

Consider diagrams
\[
 (G,X) \longleftarrow (G_{1},X_{1}) \rightarrow (\CSp(V),\Sds),
\]
where \(G\) is the adjoint group of the reductive group~\(G_{1}\),
and where \(X_{1}\) is a \(G_{1}(\RR)\)-conjugacy class of
morphisms from \(\DelS\) to~\(G_{1}\).
There is a section \(\tilde{G} \to G_{1}\),
so that \(V\) is a representation of~\(\tilde{G}\).
The goal is to determine the
non-trivial irreducible complex representations of~\(\tilde{G}\),
which is essentially equivalent to occuring in the complexification of
one of the representations just obtained.

Replacing \(G_{1}\) by the subgroup generated
by the derived group~\(G_{1}^{\der}\) and the image of~\(h\),
one can assume that conjugation by \(h(i)\)
is a Cartan involution of~\(G_{1}/w(\Gm)\).
Hence there exists a polarisation~\(\Psi\) on~\(V\)
(\S1.18(a)~of~\cite{Del_ShimVar}),
such that \(\rho\) is a morphism from \((G_{1},X_{1})\) to \((\CSp(V),\Sds)\).

\paragraph{(\S1.3.4 of~\cite{Del_ShimVar})}
Consider the following projective system~\(H_{n})_{n \in \NN}\):
order~\(\NN\) by divisibility, \(H_{n} = \Gm\),
and the transition morphism from \(H_{nd}\) to \(H_{n}\) is \(x \mapsto x^{d}\).
(Then \(\lim H_{n}\) is
the universal covering---in the algebraic sense---of~\(\Gm\).)
A \emph{fractional morphism} of~\(\Gm\) to a group~\(H\)
is an element of \(\colim \Hom(H_{n}, H)\).
The same for the group~\(\DelS\).
For a fractional morphism \(\mu \colon \Gm \to H\),
defined by \(\mu_{n} \colon H_{n} = \Gm \to H\),
every linear representation~\(V\) of~\(H\)
is the sum of subspaces \(V_{a}\) (\(a \in (1/n)\ZZ\))
such that \(\Gm\) acts on~\(V_{a}\) via~\(\mu_{n}\)
as multiplication by~\(x^{na}\).
The \(a\) such that \(V_{a} \ne 0\) are the \emph{weights} of~\(\mu\) on~\(V\).
Similarly, a fractional morphism \(h \colon \DelS \to H\)
determines a fractional Hodge decomposition~\(V_{r,s}\) of~\(V\)
(\(r,s \in \QQ\)).

\begin{lemma}[1.3.5 of~\cite{Del_ShimVar}] \label{Del135}
 For \(h \in X\), let \(\tilde{\mu}_{h}\) be
 the fractional lift of~\(\mu_{h}\) to~\(\tilde{G}_{\CC}\).
 The representations~\(W\) of~\cref{Del132} are those
 for which \(\tilde{\mu}_{h}\) has only two weights \(a\)~and~\(a+1\).
\end{lemma}
\begin{proof}
 \pf\ Omitted; see~\cite{Del_ShimVar}. \qed
\end{proof}

\paragraph{(\S1.3.6 of~\cite{Del_ShimVar})}
Let us translate the condition of~\cref{Del135} in terms of roots.
Let \(T\) be a maximal torus of~\(G_{\CC}\),
\(\tilde{T}\) its inverse image in \(\tilde{G}_{\CC}\),
\(B\) a system of simple roots of~\(T\),
and \(\mu \in Y(T)\) the representative in the fundamental Weyl chamber
of the conjugacy class of~\(\mu_{h}\)~(\(h \in X\)).
If \(\alpha\) is the dominant weight of~\(W\),
the smallest weight is \(-\tau(\alpha)\),
where \(\tau\) is the opposition involution.
For \(\beta\) a weight of~\(W\), the \(\langle \mu, \beta \rangle\)
may only take the two values \(a\)~and~\(a + 1\).
The weights are all of the form
\((\alpha + \text{a \(\ZZ\)-linear combination of roots})\),
and the \(\langle \mu, r \rangle\), with \(r\) a root, are integers.
The condition is therefore
\(\langle \mu, -\tau(\alpha) \rangle = \langle \mu, \alpha \rangle - 1\),
that is to say
\begin{equation} \label{Del1361}
 \langle \mu, \alpha + \tau(\alpha) \rangle = 1.
\end{equation}

Let us determine the solutions to this condition.
For all dominant weights~\(\alpha\)
we have \(\langle \mu, \alpha + \tau(\alpha) \rangle \in \ZZ\),
because \(\alpha + \tau(\alpha)\) is a \(\ZZ\)-linear combination of roots.
If \(\alpha \ne 0\), then \(\alpha > 0\),
for otherwise \(\mu\) kills all the weights of the corresponding representation.
A dominant weight~\(\alpha\) that satisfies~\cref{Del1361}
can not be the sum of two weights.

\begin{lemma}[1.3.7 of~\cite{Del_ShimVar}] \label{Del137}
 Only the fundamental weights can satisfy~\cref{Del1361}.
\end{lemma}

\paragraph{(\S1.3.8 of~\cite{Del_ShimVar})}
By \cref{Del137}, the desired representations~\(W\)
factor via a simple factor~\(G_{\iota}\) of~\(G\),
and their dominant weight is a fundamental weight; it corresponds
to a node of the Dynkin diagram~\(\Delta_{\iota}\) of~\(G_{\iota\CC}\).
The necessary and sufficient condition~\cref{Del1361}
only depends on the projection of~\(\mu\) to~\(G_{\iota\CC}\);
it corresponds with a special node~\(s\) of~\(\Delta_{\iota}\),
and \(s\) with a simple root~\(\alpha_{s}\).
The number \(\langle \mu, \varpi \rangle\), for \(\varpi\) a weight,
is the coefficient of~\(\alpha_{s}\)
in the expression of~\(\varpi\)
as \(\QQ\)-linear combination of the simple roots.

\paragraph{Table (1.3.9 of~\cite{Del_ShimVar})} \label[table]{DelTable}
We reproduce the diagrams depicted in~\cite{Del_ShimVar};
and we add one column of information.

\begin{tabular}{lm{2.5cm}m{2cm}m{3cm}l}
 D.t. & Diagram & \(h^{-1,1}\) & \(G_{\iota}\) & Note \\
 \(A_{n}\) &
 % \begin{tikzpicture}
 %  \node (pi1) at (0,0) {};
 %  \node (pi2) [right=of pi1] {};
 % \end{tikzpicture}
 \begin{dynkin}
  \dynkinedge{1}{0}{2}{0};
  \dynkinnodes{2}{0}{3}{0};
  \dynkinedge{3}{0}{5}{0};
  \dynkinnodes{5}{0}{6}{0};
  \dynkinedge{6}{0}{7}{0};
  \foreach \x in {1,...,7}
  {
   \ifnum \x=4 {\dynkinnodespecial{\x}{0}}
   \else {\dynkinnode{\x}{0}}
   \fi
  }
 \end{dynkin}
 & \(pq\) & \(\PU(p,q)\)? & \(n = p+q-1\) \\
 \(B_{n}\) &
 \begin{dynkin}
  \dynkinedge{1}{0}{2}{0};
  \dynkinnodes{2}{0}{3}{0};
  \dynkinedge{3}{0}{4}{0};
  \dynkindoubleedge{4}{0}{5}{0};
  \dynkinnodespecial{1}{0};
  \foreach \x in {2,...,5}
  {
   \dynkinnode{\x}{0}
  }
 \end{dynkin}
 & TODO & \(\PSpin(2,n-2)\)? \\
 \(C_{n}\)
 &
 \begin{dynkin}
  \dynkinedge{1}{0}{2}{0};
  \dynkinnodes{2}{0}{3}{0};
  \dynkinedge{3}{0}{4}{0};
  \dynkindoubleedge{5}{0}{4}{0};
  \dynkinnodespecial{5}{0};
  \foreach \x in {1,...,4}
  {
   \dynkinnode{\x}{0}
  }
 \end{dynkin}
 & TODO \\
 \(D_{n}\)
 &
 \begin{dynkin}
  \foreach \x in {2,...,4}
  {
   \dynkinnode{\x}{0}
  }
  \dynkinnode{4.5}{.9}
  \dynkinnode{4.5}{-.9}
  \dynkinnodespecial{1}{0}
  \dynkinedge{1}{0}{2}{0}
  \dynkinnodes{2}{0}{3}{0}
  \dynkinedge{3}{0}{4}{0}
  \dynkinedge{4}{0}{4.5}{.9}
  \dynkinedge{4}{0}{4.5}{-.9}
 \end{dynkin}
 & TODO \\
 \(D_{n}\)
 &
 \begin{dynkin}
  \foreach \x in {1,...,4}
  {
   \dynkinnode{\x}{0}
  }
  \dynkinnodespecial{4.5}{.9}
  \dynkinnode{4.5}{-.9}
  \dynkinedge{1}{0}{2}{0}
  \dynkinnodes{2}{0}{3}{0}
  \dynkinedge{3}{0}{4}{0}
  \dynkinedge{4}{0}{4.5}{.9}
  \dynkinedge{4}{0}{4.5}{-.9}
 \end{dynkin}
 & TODO \\
 \(E_{6}\)
 &
 \begin{dynkin}
  \foreach \x in {2,...,5}
  {
   \dynkinnode{\x}{0}
  }
  \dynkinnodespecial{1}{0}
  \dynkinnode{3}{1}
  \dynkinedge{1}{0}{5}{0}
  \dynkinedge{3}{0}{3}{1}
 \end{dynkin}
 & TODO \\
 \(E_{7}\)
 &
 \begin{dynkin}
  \foreach \x in {1,...,5}
  {
   \dynkinnode{\x}{0}
  }
  \dynkinnodespecial{6}{0}
  \dynkinnode{3}{1}
  \dynkinedge{1}{0}{6}{0}
  \dynkinedge{3}{0}{3}{1}
 \end{dynkin}
 &
 TODO
\end{tabular}


\section{Hyperadjoint motives}

TODO %TODO

\section{The Hodge numbers of a motive with endomorphisms}

\paragraph{Setup}
Let \(K \subset \CC\) be a finitely generated field.
Let \(\bar{K}\) be the algebraic closure of~\(K\) in~\(\CC\).
Let \(M\) be a motive over~\(K\).
Let \(E\) be subfield of~\(\End(M)\).
Let \(\tilde{E}\) be the normal closure of~\(E\) in~\(\CC\).
After passing to a suitable finite extension of~\(K\)
we may and do assume that \(\tilde{E} \subset K\).

Let \(p\) be a prime number.
Fix an embedding \(K \into K_{v}\) into some \(p\)-adic field~\(K_{v}\).
Let \(\BdR{K_{v}}\) be the ring of \(p\)-adic periods associated with~\(K_{v}\),
in the sense of Fontaine~\cite{Fo94}.
Fix an embedding \(\bar{K} \into \BdR{K_{v}}\)
that extends the embedding \(K \into K_{v}\).
Write \(\bar{K}_{v}\) for the algebraic closure of~\(K_{v}\) in~\(\BdR{K_{v}}\).

Let \(\Sigma = \Sigma(E)\) be the set of embeddings \(E \into \tilde{E}\).
Depending on the context,
we may also view an element \(\sigma \in \Sigma\)
as embedding \(E \into K\) or \(E \into \CC\) or \(E \into \BdR{K_{v}}\).

\paragraph{}
Observe that \(\HB(M) \otimes_{\QQ} \CC\) is a module over
\(E \otimes_{\QQ} \CC = \prod_{\sigma \in \Sigma} \CC\).
Thus we have a decomposition
\(\HB(M) \otimes_{\QQ} \CC \cong
 \bigoplus_{\sigma \in \Sigma} \HB(M) \otimes_{E,\sigma} \CC\).
Since the comparison isomorphism
\(\HB(M) \otimes_{\QQ} \CC \cong \HdR(M) \otimes_{K} \CC\)
is compatible with the action of~\(E\),
we find that \(\HdR(M)\) is a filtered module over
\(E \otimes_{\QQ} K \cong \prod_{\sigma \in \Sigma} K\).
This gives a decomposition
\(\HdR(M) \cong \bigoplus_{\sigma \in \Sigma} \HdR(M)^{(\sigma)}\)
in such a way that
\[
 \HdR(M)^{\sigma} \otimes_{K} \CC \cong \HB(M) \otimes_{E,\sigma} \CC
\]
is a natural isomorphism of filtered vector spaces.

\paragraph{}
We have a natural isomorphism of vector spaces
\(\HB(M) \otimes \QQp \cong \Hp(M)\).
This isomorphism is compatible with the action of~\(E\),
and \(\Hp(M)\) is a module over \(E \otimes \QQp\).

We have a natural isomorphism
\(\HdR(M) \otimes_{K} \BdR{K_{v}} \cong \Hp(M) \otimes_{\QQp} \BdR{K_{v}}\)
of filtered modules with an action by \(\Gal(\bar{K}_{v}/K_{v})\).
This isomorphism is compatible with the action of~\(E\),
so that \(\Hp(M) \otimes_{\QQp} \BdR{K_{v}}\)
is a module over
\(E \otimes_{\QQ} \BdR{K_{v}} = \prod_{\sigma \in \Sigma} \BdR{K_{v}}\).
Thus we get a natural isomorphism
\[
 \HdR(M)^{(\sigma)} \otimes_{K} \BdR{K_{v}} \cong
 \Hp(M) \otimes_{E \otimes \QQp, \sigma \otimes \id} \BdR{K_{v}}
\]
of filtered modules with an action by \(\Gal(\bar{K}_{v}/K_{v})\),
for each \(\sigma \in \Sigma\).

\paragraph{}
Assume that \(p\) is totally split in~\(E\).
Then we get a simplified description of
\(\Hp(M) \otimes_{E \otimes \QQp, \sigma \otimes \id} \BdR{K_{v}}\),
as follows.
For each \(\sigma \in \Sigma\)
the composite embedding \(\sigma \colon E \into K \into K_{v}\)
factors via \(\QQp \subset K_{v}\).
This gives a bijection between \(\Sigma\)~and
the places \(\pi\) of~\(E\) that lie above~\(p\).

Since \(\Hp(M)\) is a module over \(E \otimes \QQp = \prod_{\pi|p} E_{\pi}\)
we get a decomposition \(\Hp(M) \cong \bigoplus_{\pi|p} \HH_{\pi}(M)\)
of Galois representations.
The above bijection between \(\Sigma\) and the places \(\pi|p\)
allows us to unambigously write \(\Hp(M)^{(\sigma)}\) for~\(\HH_{\pi}(M)\).
Then
\(\Hp(M) \otimes_{E \otimes \QQp, \sigma \otimes \id} \BdR{K_{v}}
 = \Hp(M)^{(\sigma)} \otimes_{\QQp} \BdR{K_{v}}\).

\paragraph{}
The upshot of all these comparison isomorphisms
and their decompositions induced by the action of~\(E\)
is that we get a refined view on the Hodge numbers of~\(M\).
It is classical that the comparison isomorphisms
equate the Hodge numbers of~\(\HB(M)\)
with the Hodge--Tate numbers of the Galois representation~\(\Hp(M)\).
Now we also have the following result.

\begin{theorem}
 Let \(M\) be a motive over a finitely generated field \(K \subset \CC\).
 Let \(E\) be subfield of~\(\End(M)\), and
 let \(\tilde{E}\) be the normal closure of~\(E\) in~\(\CC\).
 Assume that \(\tilde{E} \subset K\).
 Let \(\Sigma = \Sigma(E)\) be the set of embeddings \(E \into \tilde{E}\).
 For each \(\sigma \in \Sigma\) we then have
 \[
  \bigl\{\,
  \text{Hodge numbers of \(\HB(M) \otimes_{E,\sigma} \CC\)}
  \,\bigr\}
  =
  \bigl\{\,
  \text{Hodge--Tate numbers of
   \(\Hp(M) \otimes_{E \otimes \QQp, \sigma \otimes \id} \BdR{K_{v}}\)}
  \,\bigr\}.
 \]
\end{theorem}
\begin{proof}
 \pf\ By the above computations we see that both the left-hand side
 and the right-hand side are equal to
 \(
  \bigl\{\,
  \text{Hodge numbers of \(\HdR(M)^{(\sigma)}\)}
  \,\bigr\}
 \). \qed
\end{proof}

\section{Compatible systems}
TODO %TODO
Give a summary of the other article.
First finish that article,
then copy/paste the relevant definitions and the main theorem.

\section{Geometric lifts of Galois representations}

Give a summary of the relevant result of~\cite{No06}.

\section{Structure of hyperadjoint abelian motives}

\begin{lemma} \label{EGstruct}
 Let \(K \subset \CC\) be a finitely generated field.
 Let \(M\) be an irreducible hyperadjoint abelian motive over~\(K\).
 Let \(E\) be the endomorphism algebra of~\(M\).
 Assume that \(E = \End(M_{\CC})\).
 Then \(E\) is a totally real field,
 and \(\GB(M) = \Res_{E/\QQ} \mathcal{G}\)
 for some absolutely simple adjoint algebraic group \(\mathcal{G}/E\).
\end{lemma}

\paragraph{}
We follow the conventions of table~2.3.8 of~\cite{Del_ShimVar}.
Let \(K \subset \CC\) be a finitely generated field.
Let \(M\) be an irreducible hyperadjoint abelian motive over~\(K\).
\begin{enumerate}
 \item All the simple factors of~\(\GB(M)_{\CC}\)
  have the same Dynkin type;
  the only types that can occur are the classical types
  \(A_{k}\),~\(B_{k}\), \(C_{k}\), and~\(D_{k}\).
 \item For the non-compact simple factors of~\(\GB(M)_{\RR}\)
  there is a refined type:
  \(A_{k}\),~\(B_{k}\), \(C_{k}\), \(D_{k}^{\RR}\), and~\(D_{k}^{\HQ}\).
  All the non-compact simple factors have the same type.
  For \(k = 4\) the distinction between 
  \(D_{k}^{\RR}\), and~\(D_{k}^{\HQ}\)
  is subtle, and depends on \(\GB(M)\) over~\(\QQ\).
  We refer to the final paragraphs of table~2.3.8 of~\cite{Del_ShimVar}
  for more details.
\end{enumerate}

\paragraph{}
Let \(K \subset \CC\) be a finitely generated field.
Let \(\bar{K}\) be the algebraic closure of~\(K\) in~\(\CC\).
Let \(M\) be a motive over~\(K\).
Let \(E\) be subfield of~\(\End(M)\).
Let \(\tilde{E}\) be the normal closure of~\(E\) in~\(\CC\).
After passing to a suitable finite extension of~\(K\)
we may and do assume that \(\tilde{E} \subset K\).



\begin{proposition}
 Let \(M\) be an irreducible hyperadjoint abelian motive of type \(D_{k}\).
 One can recognize from \(\HH_{\Lambda}(M)\)
 whether \(M\) is of type \(D_{k}^{\RR}\) or of type \(D_{k}^{\HQ}\).
\end{proposition}
\begin{proof}
 \pf\ Use theorem~6.2 till remark~6.6 of~\cite{No06}. \qed
\end{proof}

\paragraph{}
Recall that \(\HH^{1}(E,\iAut_{G})\) classifies forms of~\(G\) over~\(E\);
reference?


\section{An intermediate result}

\begin{theorem} \label{intermed}
 Let \(M_{1}\) and~\(M_{2}\) be two hyperadjoint abelian motives
 over a finitely generated field \(K \subset \CC\).
 Assume that \(\MTC(M_{1})\) and \(\MTC(M_{2})\) are true.
 Then
 \[
  M_{1} \cong M_{2} \iff \exists \ell :
  \Gl(M_{1} \oplus M_{2}) \subsetneq \Gl(M_{1}) \times \Gl(M_{2}).
 \]
\end{theorem}

\begin{corollary} \label{intermed_sums}
 Let \(M_{i}\), with \(i \in I\),
 be a finite collection of hyperadjoint abelian motives.
 Write \(M = \bigoplus M_{i}\).
 If \(\MTC(M_{i}\) is true for all \(i \in I\),
 then \(\MTC(M)\) is true.
\end{corollary}
\begin{proof}
 \pf\ We may and do assume that for \(i,j \in I\)
 we have \(M_{i} \cong M_{j} \iff i = j\).
 By \cref{intermed},
 we know that if \(i,j \in I\) are two different indices,
 then \(\Gl(M_{i} \oplus M_{j}) \cong \Gl(M_{i}) \times \Gl(M_{j})\).
 Recall that \(\Gl(M) \into \prod_{i \in I} \Gl(M_{i})\),
 with surjective projections on to the factors \(\Gl(M_{i})\).
 By the lemma in step~3 on pages~790--791 of~\cite{Ribet}
 we conclude that
 \(\Gl(\bigoplus_{i \in I} M_{i}) \cong \prod_{i \in I} \Gl(M_{i})\).
 Since we know \(\MTC(M_{i}\) for all \(i \in I\)
 we conclude that \(\MTC(M)\) is true. \qed
\end{proof}

\paragraph{Strategy}
The proof of \cref{intermed} will consist of several distinct steps.
\begin{enumerate}
 \item We first show that \(\End(M_{1}) = E = \End(M_{2})\).
 \item We then show that \(\GB(M_{1}) = G = \GB(M_{2})\).
 \item The next step shows that
  \(\mu_{1} \colon \Gm[\CC] \to \GB(M_{1})\)
  and
  \(\mu_{2} \colon \Gm[\CC] \to \GB(M_{2})\)
  are in the same \(G(\CC)\)-conjugacy class.
 \item The final step deduces \(M_{1} \cong M_{2}\).
\end{enumerate}

\paragraph{}
Let \(K \subset \CC\) be a finitely generated field.
Let \(M\) be a hyperadjoint abelian motive over~\(K\).
Let \(E\) be the endomorphism algebra of~\(M\).
Let \(\tilde{E}\) be the normal closure of~\(E\) in~\(\CC\).

After passing to a suitable finitely generated extension of~\(K\)
we may and do assume that
\begin{enumerate*}[label=(\textit{\roman*})]
 \item \(E = \End(M_{\CC})\);
 \item \(\Gl(M)\) is connected for all prime numbers~\(\ell\); and
 \item \(\tilde{E} \subset K\).
\end{enumerate*}
(Actually assumption~(\textit{ii}) implies assumption~(\textit{i}),
but that is not important right now.)
Let \(\bar{K}\) be the algebraic closure of~\(K\) in~\(\CC\).

Let \(\Sigma = \Sigma(E)\) be the set of embeddings \(E \into \tilde{E}\).
Depending on the context,
we may also view an element \(\sigma \in \Sigma\)
as embedding \(E \into K\) or \(E \into \CC\).

\begin{proposition} \label{twofib}
 Let \(K \subset \CC\) be a finitely generated field.
 Let \(S\) be a scheme of finite type over~\(K\).
 Let \(f \colon \mathcal{A} \to S\) be an abelian scheme.
 Let \(x\) and~\(y\) be two points in~\(S(K)\).
 Let \(\ell\) be a prime number.
 If \(\Lie(\Gl(\mathcal{A}_{x}))\) and \(\Lie(\Gl(\mathcal{A}_{y}))\)
 are isomorphic as Galois representations,
 then the abelian varieties \(\mathcal{A}_{x}\) and~\(\mathcal{A}_{y}\)
 are isogenous over a finite extension of~\(K\).
\end{proposition}
\begin{proof}
 \pf\ View the lisse \(\ell\)-adic sheaf \(\mathrm{R}^{1}f_{*}\QQl\)
 as representation
 \(\pi(S,\bar{x}) \to \GL(\Hl^{1}(\mathcal{A}_{x})\).
 Let \(\rho_{x}\) denote the representation
 \(\Gal(\bar{K}/K) \to \GL(\Hl^{1}(\mathcal{A}_{x})\).
 Via parallel transport,
 the Galois representation on~\(\Hl^{1}(\mathcal{A}_{y})\)
 corresponds with a representation
 \(\rho_{y} \colon \Gal(\bar{K}/K) \to \GL(\Hl^{1}(\mathcal{A}_{x})\).
 Let \(G \subset \GL(\Hl^{1}(\mathcal{A}_{x})\) be the smallest
 subgroup such that \(G(\QQl)\) contains the image
 of both \(\rho_{x}\) and~\(\rho_{y}\).

 Consider the maps \(\ab \colon G \to G^{\ab}\)
 and \(\ad \colon G \to G^{\ad}\).
 It follows from \cref{MTCcentre} that
 the map \(\ab \circ \rho_{s}\) does not depend on~\(s \in \{x,y\}\).
 By assumption
 \(\Lie(\Gl(\mathcal{A}_{x}))\) and \(\Lie(\Gl(\mathcal{A}_{y}))\)
 are isomorphic as Galois representations,
 and therefore \(\ad \circ \rho_{s}\) does not depend on~\(s \in \{x,y\}\).

 Write \(Z\) for the finite intersection \(\ker(\ab) \cap \ker(\ad)\).
 Define the morphism \(\phi \colon \Gal(\bar{K}/K) \to G(\QQl)\)
 via \(g \mapsto \rho_{x}(g)\rho_{y}(g)^{-1}\).
 We claim that \(\phi\) is a homomorphism.
 Note that \(\ab \circ \phi\) and \(\ad \circ \phi\)
 are trivial, and therefore \(\phi\) factors via~\(Z(\QQl)\).
 We need to show that
 \[
  \rho_{x}(g_{1})\rho_{y}(g_{1})^{-1} \cdot \rho_{x}(g_{2})\rho_{y}(g_{2})^{-1}
  \stackrel{?}{=}
  \rho_{x}(g_{1}g_{2}) \cdot \rho_{y}(g_{1}g_{2})^{-1}
  =
  \rho_{x}(g_{1})\rho_{x}(g_{2}) \cdot
  \rho_{y}(g_{2})^{-1}\rho_{y}(g_{1})^{-1}.
 \]
 This amounts to showing that
 \(\rho_{y}(g_{1})^{-1} \cdot \phi(g_{2}) =
  \phi(g_{2}) \cdot \rho_{y}(g_{1})^{-1}\).
 Since \(Z\) is contained in the centre of~\(G\),
 we know that \(\phi(g_{2})\) commutes with \(\rho_{y}(g_{1})^{-1}\).
 This proves the claim that \(\phi\) is a homomorphism.

 Because \(Z\) is finite,
 the homomorphism \(\phi\) becomes trivial
 after replacing \(K\) by suitable a finite extension;
 which means \(\rho_{x} = \rho_{y}\).
 The \namecref{twofib} now follows from Faltings' results
 on the Tate conjecture for endomorphisms of abelian varieties
 (Korollar~1 of Satz~4 of~\cite{Fal83}, see also~\cite{Fal84}). \qed
\end{proof}

\section{Main theorem}

\paragraph{}
Let \(K \subset \CC\) be a finitely generated field.
Let \(\mathcal{M}_{K} \subset \Mot_{K}\) be the subcategory
of \emph{abelian} motives for which the Mumford--Tate conjecture is true.

\begin{theorem}
 Let \(K \subset \CC\) be a finitely generated field.
 The category~\(M_{K}\) is a Tannakian subcategory of \(\Mot_{K}\).
\end{theorem}
\begin{proof}
 \pf\
 The category~\(\Mot_{K}\) is semisimple,
 so subquotients are direct summands.
 It is clear that the subcategory~\(\mathcal{M}_{K}\)
 is closed under duals, tensor powers, and direct summands.
 Let \(M_{1}\) and~\(M_{2}\) be two objects in~\(\mathcal{M}_{K}\).
 We need to show that \(M_{1} \oplus M_{2}\)
 and \(M_{1} \otimes M_{2}\) are objects in \(\mathcal{M}_{K}\).
 Observe that \(M_{1} \otimes M_{2}\) is a direct summand of
 \((M_{1} \oplus M_{2})^{\otimes 2}\).
 Thus we are done if we show that the Mumford--Tate conjecture is true for
 \(M = M_{1} \oplus M_{2}\).
 By foobar?? we may assume that \(M\) is hyperadjoint.
 Write \(M = \bigoplus_{i \in I} M'_{i}\),
 with the \(M'_{i}\) hyperadjoint.
 Now the result follows from \cref{intermed_sums}.
 \qed
\end{proof}

\begin{corollary}
 Let \(A_{1}\),~\(A_{2}\), \dots, \(A_{n}\) be abelian varieties over~\(K\).
 If the Mumford--Tate conjecture is true for \(A_{i}\), for \(i = 1,\dots,n\),
 then the Mumford--Tate conjecture is true for
 \(A_{1} \times A_{2} \times \cdots \times A_{n}\).
\end{corollary}


\printbibliography

\end{document}
