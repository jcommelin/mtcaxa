\documentclass[10pt,twoside,leqno]{article}

\usepackage[utf8]{inputenc}
\usepackage[T1]{fontenc}
\usepackage[full]{textcomp}

\usepackage{csquotes}
\usepackage[english]{babel}

% \usepackage[urw-garamond,expert,
% uppercase=upright,greeklowercase=upright]{mathdesign}
% \usepackage[osf,swashQ]{garamondx}
% \def\kappa{\varkappa}
\usepackage{mathtools}
\mathtoolsset{mathic} % Italic correction before mathmode, works with ~'s.
% \def\mathds{\mathbb}

\usepackage{cfr-lm}
\usepackage{dsfont}  % disable this when loading mathdesign

\usepackage{microtype}
\linespread{1.25}  % = 1.500 * fontheight
% \linespread{1.388} % = 1.666 * fontheight
\usepackage[
% paper=b5paper,
nohead,nomarginpar,
% bindingoffset=.3cm,
paper=a4paper,
]{geometry}

% \raggedbottom

\usepackage{lastpage}
\usepackage{fancyhdr}
\pagestyle{fancy}
\fancyhf{}
\renewcommand{\headrulewidth}{0pt}
\fancyfoot[LE,RO]{\thepage/\pageref{LastPage}}

\usepackage{longtable}
\usepackage{booktabs}
\usepackage{tabu}

\usepackage[inline]{enumitem}
\setlist{noitemsep,nosep,listparindent=\parindent}
\setlist[itemize]{label=\guillemotright}
\setlist[enumerate,1]{ref=\thesubsection.\arabic*}
\setlist[enumerate,2]{label=\alph*.,ref=\theenumi.\alph*}
% \setlist[enumerate*]{label=(\textit{\roman*}\thinspace)}

\usepackage{cjhebrew}

\usepackage[backend=biber,doi=false,url=false,isbn=false,%
sorting=nyt,safeinputenc]{biblatex}
\bibliography{\jobname.bib}
\defbibheading{bibliography}[\bibname]{\sectionstar{#1}}

%%% SECTION HEADINGS
\usepackage{ifthen}
\makeatletter
\renewcommand{\part}[1]{%
 \cleardoublepage%
 \vbox{\null\vskip90pt%
 \normalfont\fontsize{20pt}{30pt}\selectfont%
 \baselineskip=30pt%
 \scshape\noindent\textls*{#1}\par}%
 \addcontentsline{toc}{part}{#1}%
 \@afterindentfalse%
 \@afterheading%
}
\renewcommand{\section}[1]{%
 \refstepcounter{section}%
 \bigskip\penalty-250
 \vbox{\normalfont\fontsize{12pt}{15pt}\selectfont%
  \centering\scshape\noindent\textls*{\thesection\quad#1}%
  \par}
 \nobreak
 \addcontentsline{toc}{section}{\protect\numberline{\thesection} #1}%
 \@afterindentfalse%
 \@afterheading%
} 
\newcommand{\sectionstar}[1]{%
 \bigskip\penalty-250
 \vbox{\normalfont\fontsize{12pt}{15pt}\selectfont%
  \centering\scshape\noindent\textls*{#1}%
  \par}
 \nobreak\vskip15pt
 \@afterindentfalse%
 \@afterheading%
} 
\renewcommand{\paragraph}[1]{\par\bigskip\refstepcounter{subsection}%
 {\normalfont\normalsize\scshape\noindent\thesubsection%
 \ifthenelse{\equal{#1}{}}%
 {}%
 {\ \textls{#1.}}%
 \ ---}%
}
\newcommand{\readme}{\par\vskip\baselineskip%
 {\normalfont\normalsize\scshape\noindent%
  \textls{Readme.}\ ---}
}
\renewcommand\tableofcontents{%
 \sectionstar{\contentsname}%
 \@starttoc{toc}%
}
\renewcommand*\l@part[2]{%
 \addvspace{15pt \@plus\p@}%
 \noindent{\leavevmode%
  \scshape\textls{#1\qquad#2}%
 }\par\nobreak%
}
\renewcommand*\l@section[2]{%
 \setlength\@tempdima{\parindent}%
 \noindent
 {\leavevmode%
  \hskip\parindent#1\qquad#2%
 }\par\nobreak%
}
\makeatother

%%% MATH PACKAGES
\usepackage{amsmath,amssymb}  % disable when using mathdesign
\usepackage{mathrsfs}         % disable when using mathdesign
\usepackage{mathabx}
\usepackage[all]{xy}

\usepackage[thmmarks,amsmath]{ntheorem}
\usepackage{thmtools}

\numberwithin{equation}{subsection}

\declaretheoremstyle[headformat=swapnumber,headpunct={.\ ---},%
headfont=\normalfont\scshape\lsstyle,bodyfont=\itshape,%
spaceabove=0pt,spacebelow=0pt,%
preheadhook={\bigskip}]{theorem}
\declaretheorem[style=theorem,sibling=subsection]{theorem}
\declaretheorem[style=theorem,sibling=subsection]{proposition}
\declaretheorem[style=theorem,sibling=subsection]{lemma}
\declaretheorem[style=theorem,sibling=subsection]{corollary}
\declaretheorem[style=theorem,sibling=subsection]{conjecture}

\declaretheoremstyle[headformat=swapnumber,headpunct={.\ ---},%
headfont=\normalfont\scshape\lsstyle,bodyfont=\normalfont,%
spaceabove=0pt,spacebelow=0pt,%
preheadhook={\bigskip}]{definition}
\declaretheorem[style=definition,sibling=subsection]{definition}
\declaretheorem[style=definition,sibling=subsection]{exercise}
\declaretheorem[style=definition,sibling=subsection]{example}
\declaretheorem[style=definition,sibling=subsection]{remark}
\declaretheorem[style=definition,sibling=subsection]{construction}

\declaretheoremstyle[headpunct={\!.},headfont=\itshape,bodyfont=\normalfont,%
qed=\ensuremath{\square},spaceabove=0pt,spacebelow=0pt]{proof}
\declaretheoremstyle[headpunct={\!.},headfont=\itshape,bodyfont=\normalfont,%
qed=\ensuremath{\square},spaceabove=0pt,spacebelow=0pt]{nonumberproof}
\declaretheorem[style=proof,numbered=no]{proof}

\declaretheoremstyle[headformat=swapnumber,headpunct={.\ ---},%
headfont=\itshape,bodyfont=\normalfont,qed=\ensuremath{\square},%
spaceabove=0pt,spacebelow=0pt,%
preheadhook={\bigskip}]{nproof}
\declaretheorem[style=nproof,sibling=subsection,name=Proof]{nproof}

\usepackage{cleveref}
\crefname{condition}{condition}{conditions}
\crefname{conjecture}{conjecture}{conjectures}
\crefname{construction}{construction}{constructions}
\crefname{corollary}{corollary}{corollaries}
\crefname{diagram}{diagram}{diagrams}
\crefformat{subsection}{\S#2#1#3}
\crefformat{enumi}{\S#2#1#3}
\crefformat{nproof}{\S#2#1#3}
\creflabelformat{equation}{#2#1#3}

%%% MATH MACROS
\newcommand{\id}{\textnormal{id}}

\newcommand{\into}{\hookrightarrow}
\newcommand{\onto}{\twoheadrightarrow}
\newcommand{\longto}{\longrightarrow}
\newcommand{\longinto}{\lhook\joinrel\longrightarrow}

\renewcommand{\Im}{\textnormal{Im}}

\newcommand{\Hom}{\textnormal{Hom}}
\newcommand{\End}{\textnormal{End}}
\newcommand{\Isom}{\textnormal{Isom}}
\newcommand{\Inn}{\textnormal{Inn}}
\newcommand{\Aut}{\textnormal{Aut}}
\newcommand{\Out}{\textnormal{Out}}
\newcommand{\iHom}{\underline{\Hom}}
\newcommand{\iEnd}{\underline{\End}}
\newcommand{\iIsom}{\underline{\Isom}}
\newcommand{\iInn}{\underline{\Inn}}
\newcommand{\iAut}{\underline{\Aut}}
\newcommand{\iOut}{\underline{\Out}}

\newcommand{\Mat}{\textnormal{Mat}}
\newcommand{\Sym}{\textnormal{Sym}}

\newcommand{\Fil}{\textnormal{Fil}}

\newcommand{\dual}[1]{\check{#1}}

\newcommand{\ZZ}{\mathbb{Z}}
\newcommand{\QQ}{\mathbb{Q}}
\newcommand{\QQl}{\QQ_{\ell}}
\newcommand{\QQlbar}{\bar{\QQ}_{\ell}}
\newcommand{\QQp}{\QQ_{p}}
\newcommand{\QQpbar}{\bar{\QQ}_{p}}
\newcommand{\RR}{\mathbb{R}}
\newcommand{\CC}{\mathbb{C}}
\newcommand{\HQ}{\mathbb{H}}
\newcommand{\FF}{\mathbb{F}}
\newcommand{\FFp}{\FF_{p}}
\newcommand{\FFq}{\FF_{q}}
\newcommand{\FFqbar}{\bar{\FF}_{q}}
\newcommand{\Adele}{\mathbb{A}}
\newcommand{\fin}{\textnormal{f}}

\newcommand{\primes}{\mathscr{L}}

\newcommand{\Spec}{\textnormal{Spec}}

\newcommand{\DelS}{\mathbb{S}}
\newcommand{\Sh}{\textnormal{Sh}}
\newcommand{\mSh}{\mathscr{S}}
\newcommand{\DD}{\mathbb{D}}
\newcommand{\mcG}{\mathcal{G}}
\newcommand{\Kmpt}{\mathcal{K}}
\newcommand{\Sds}{\mathfrak{H}^{\pm}}
\newcommand{\AV}{\mathscr{A}}
\newcommand{\Ag}{\AV_{g}}

\newcommand{\Gal}{\textnormal{Gal}}

% \newcommand{\HH}{\textnormal{H}}
% \newcommand{\Hhom}{\HH_{\textnormal{hom}}}
\newcommand{\HdR}{\HH_{\dR}}
% \newcommand{\Hl}{\HH_{\ell}}
% \newcommand{\Hp}{\HH_{p}}
% \newcommand{\Hlambda}{\HH_{\lambda}}
% \newcommand{\HB}{\HH_{\textnormal{B}}}
% \newcommand{\Hsigma}{\HH_{\sigma}}

% \newcommand{\GG}{\textnormal{G}}
% \newcommand{\Gl}{\GG_{\ell}}
% \newcommand{\Glc}{\Gl^{\circ}}
% \newcommand{\GB}{\GG_{\textnormal{B}}}
% \newcommand{\Gsigma}{\GG_{\sigma}}
% \newcommand{\Gmot}[1]{\GG_{\textnormal{mot},#1}}
% \newcommand{\Gmots}{\Gmot{\sigma}}
% \newcommand{\Gmotl}{\Gmot{\ell}}
% \newcommand{\GmotB}{\Gmot{\textnormal{B}}}
% \newcommand{\Gp}{\GG_{p}}

% \newcommand{\Zl}{\textnormal{Z}_{\ell}}
% \newcommand{\Zlc}{\Zl^{\circ}}
% \newcommand{\ZB}{\textnormal{Z}_{\textnormal{B}}}
% \newcommand{\Zsigma}{\textnormal{Z}_{\sigma}}
% \newcommand{\Zmot}[1]{\textnormal{Z}_{\textnormal{mot},#1}}
% \newcommand{\Zmots}{\Zmot{\sigma}}
% \newcommand{\Zmotl}{\Zmot{\ell}}

\newcommand{\BdR}[1]{\textnormal{B}_{\dR,#1}}
\newcommand{\BHT}[1]{\textnormal{B}_{\textnormal{HT},#1}}
\newcommand{\gr}{\textnormal{gr}}

\newcommand{\Vect}{\textnormal{Vect}}
\newcommand{\Filt}{\textnormal{Filt}}
\newcommand{\Rep}{\textnormal{Rep}}
\newcommand{\QHS}{\QQ\textnormal{HS}}

\makeatletter
\def\cpwith[#1]#2{\textnormal{c.p.}_{#1}(#2)}
\def\cpwithout#1{\textnormal{c.p.}(#1)}
\def\cp{\@ifnextchar[{\cpwith}{\cpwithout}}
\makeatother
\makeatletter
\def\Gmwith[#1]{\mathbb{G}_{\textnormal{m},#1}}
\def\Gmwithout{\mathbb{G}_{\textnormal{m}}}
\def\Gm{\@ifnextchar[{\Gmwith}{\Gmwithout}}
\makeatother
\newcommand{\GL}{\textnormal{GL}}
\newcommand{\SL}{\textnormal{SL}}
\newcommand{\PSL}{\textnormal{PSL}}
\newcommand{\UU}{\textnormal{U}}
\newcommand{\SU}{\textnormal{SU}}
\newcommand{\OO}{\textnormal{O}}
\newcommand{\SO}{\textnormal{SO}}
\newcommand{\Spin}{\textnormal{Spin}}
\newcommand{\CSpin}{\textnormal{CSpin}}
\newcommand{\GSp}{\textnormal{GSp}}
\newcommand{\Lie}{\textnormal{Lie}}

% \newcommand{\mfsl}{\mathfrak{sl}}
% \newcommand{\mfso}{\mathfrak{so}}

% \newcommand{\Cliff}{\textnormal{Cl}}
% \newcommand{\spin}{\textnormal{spin}}

% \newcommand{\St}{\textnormal{St}}

\newcommand{\ab}{\textnormal{ab}}
\newcommand{\der}{\textnormal{der}}
\newcommand{\ad}{\textnormal{ad}}

\newcommand{\SmPr}{\textnormal{SmPr}}

\newcommand{\an}{\textnormal{an}}
\newcommand{\cl}{\textnormal{cl}}

\newcommand{\dR}{\textnormal{dR}}
\newcommand{\et}{\textnormal{\'{e}t}}
\newcommand{\sing}{\textnormal{sing}}

\newcommand{\HH}{\textnormal{H}}
\newcommand{\Hl}{\HH_{\ell}}
\newcommand{\Hp}{\HH_{p}}
\newcommand{\Hlambda}{\HH_{\lambda}}
\newcommand{\HB}{\HH_{\textnormal{B}}}

\newcommand{\Zar}{\textnormal{Zar}}

\newcommand{\Mot}{\textnormal{Mot}}

\newcommand{\GG}{\textnormal{G}}
\newcommand{\GB}{\GG_{\textnormal{B}}}
\newcommand{\Gp}{\GG_{p}}
\newcommand{\Gl}{\GG_{\ell}}

\newcommand{\alg}{\textnormal{alg}}
\newcommand{\tra}{\textnormal{tra}}

\newcommand{\Res}{\textnormal{Res}}
\newcommand{\Nm}{\textnormal{Nm}}
\newcommand{\trace}{\textnormal{tr}}
\newcommand{\rk}{\textnormal{rk}}
\renewcommand{\det}{\textnormal{det}}
\newcommand{\res}{\textnormal{res}}

\newcommand{\chrc}{\textnormal{char}}

\newcommand{\Tangen}[1]{\langle #1 \rangle^{\otimes}}
\newcommand{\Val}{\textnormal{Val}}

\newcommand{\tr}{\textsc{tr}}
\newcommand{\cm}{\textsc{cm}}

\newcommand{\MTC}{\textnormal{MTC}}

\newcommand{\rotatesim}{\rotatebox{90}{$\sim$}}

\usepackage{tikz}
\usetikzlibrary{cd,positioning}
\usetikzlibrary{decorations.pathmorphing}

\def\title{The Mumford--Tate conjecture for products of abelian varieties}
\def\author{Johan Commelin}

\usepackage{datetime}
\def\date{\dayofweekname{\day}{\month}{\year},
 the \ordinaldate{\day} of \monthname, \number\year}

\begin{document}
\begin{center}\Large\scshape
\textls*{\title}
\end{center}

\medskip

\noindent\textit{by} \quad \author \hfill \date

\vskip3\baselineskip

\tableofcontents

\section{Introduction}

TODO %TODO
\hfill\cjRL{hw/s`nh brwK hb' b/sM yhwh}

\paragraph{Conventions}
We say that a reductive group~\(G\) over a field~\(K\) of characteristic~\(0\)
is \emph{classical} if all its simple factors
are of type \(A_{n}\),~\(B_{n}\), \(C_{n}\), or~\(D_{n}\).
(In other words, we exclude the exceptional groups.)

\section{A local-global principle for forms of adjoint algebraic groups}

\begin{theorem} \label{locglob}
 Let \(E\) be a number field.
 Let \(G\) be an absolutely simple classical group over~\(E\).
 Assume that \(G\) is either simply connected or adjoint.
 Then the natural morphism
 \[
  \HH^{1}(E, \iAut_{G}) \stackrel{\phi_{G}}{\longto}
  \prod_{v} \HH^{1}(E_{v}, \iAut_{G})
 \]
 is injective.
 (Here \(v\) runs over all the places of~\(E\), both finite and infinite.)
\end{theorem}

\paragraph{First reduction} \label{firstred}
The proof of \cref{locglob} takes the rest of this section.
We first show that we may assume that \(G\) is split and simply connected.
Let \(G'\) be a form of~\(G\) over~\(E\),
corresponding to a class \([G'] \in \HH^{1}(E, \iAut_{G})\).
Then we obtain a natural commutative diagram
\[
 \begin{tikzcd}
  \HH^{1}(E, \iAut_{G}) \ar[r,"\phi_{G}"] \ar[d,<->,"\rotatesim"]
  & \prod_{v} \HH^{1}(E_{v}, \iAut_{G}) \ar[d,<->,"\rotatesim"] \\
  \HH^{1}(E, \iAut_{G'}) \ar[r,"\phi_{G'}"]
  & \prod_{v} \HH^{1}(E_{v}, \iAut_{G'}) \\
 \end{tikzcd}
\]
where the downward map on the left sends \([G']\)
to the distinguished element \(*\) of \(\HH^{1}(E, \iAut_{G})\);
and the downward map on the right sends \(\phi_{G}([G'])\)
to \(* \in \prod_{v} \HH^{1}(E_{v}, \iAut_{G})\).
It is therefore sufficient to prove \cref{locglob}
in the case that \(G\) is a split group.
If \(G\) is simply connected, then \(\iAut_{G} \cong \Aut_{G^{\ad}}\),
and thus the adjoint case follows from the simply connected case.

\paragraph{} \label{exactEpts}
For the moment, let \(E\) be any field of characteristic~\(0\)
(for example a \(p\)-adic local field); and
let \(G\) be a classical group over~\(E\) that is split and simply connected.
Let \(\Delta\) denote the Dynkin diagram of~\(G_{\bar{E}}\),
and let \(\Delta_{\bar{E}}\) denote the Dynkin diagram of~\(G_{\bar{E}}\).
Recall that \(\iOut_{G}(\bar{E}) \cong \Aut(\Delta)\).
The assumption that \(G\) is split and simply connected
implies that
\begin{enumerate*}[label=(\textit{\roman*})]
 \item \(\iOut_{G}(E) = \iOut_{G}(\bar{E})\); and
 \item the sequence \(0 \to \iInn_{G}(E) \to \iAut_{G}(E) \to \iOut_{G}(E) \to 0\)
  is exact.
\end{enumerate*}
See for example proposition~1.5.1 and example~1.5.2 of~\cite{Conrad_RedGrp}.

\paragraph{}
For ease of mind we recall that
\[
 \Aut(\Delta) =
 \begin{cases}
  1 & \text{if \(G\) is of type \(A_{1}\)} \\
  \ZZ/2\ZZ & \text{if \(G\) is of type \(A_{n}\), with \(n > 1\)} \\
  1 & \text{if \(G\) is of type \(B_{n}\) or~\(C_{n}\)} \\
  \mathfrak{S}_{3} & \text{if \(G\) is of type \(D_{4}\)} \\
  \ZZ/2\ZZ & \text{if \(G\) is of type \(D_{n}\), with \(n > 4\).}
 \end{cases}
\]

\paragraph{} \label{phiDelta}
We now return to the setting of \cref{locglob} and~\cref{firstred}:
\(E\) is a number field,
and \(G\) is a classical group over~\(E\) that is split and simply connected.
Recall from \cref{exactEpts} that
the action of \(\Gal(\bar{E}/E)\) on~\(\Aut(\Delta)\) is trivial.
Therefore \(\HH^{1}(E, \Aut(\Delta)) = \Hom(\Gal(\bar{E}/E), \Aut(\Delta))\),
which classifies Galois extensions of~\(E\)
whose Galois group is a subgroup of~\(\Aut(\Delta)\).
Let \(F/E\) be a Galois extension, and let~\(v\) be any place of~\(E\).
Recall that \(F \otimes E_{v} \cong \prod_{w|v} F_{w}\),
and the Galois group acts transitively on the factors~\(F_{w}\).
Consider the map
\(\phi_{\Delta} \colon
 \HH^{1}(E, \Aut(\Delta)) \longto \prod_{v} \HH^{1}(E_{v}, \Aut(\Delta))\),
which maps a Galois extension \(F/E\) to \((F_{w}/E_{v})_{v}\).
Hence the map \(\phi_{\Delta}\) is injective,
since a Galois extension \(F/E\) is trivial
if and only if it is totally split at all places of~\(E\).

\paragraph{}
Recall the short exact sequence
\( 0 \to \iInn_{G} \to \iAut_{G} \to \iOut_{G} \to 0 \).
Now consider the following commutative diagram of exact sequences
\[
 \begin{tikzcd}[column sep=1em]
    \HH^{0}(E,\iOut_{G}) \ar[r] \ar[d]
  & \HH^{1}(E,\iInn_{G}) \ar[r,"\beta"]  \ar[d,"\psi"]
  & \HH^{1}(E,\iAut_{G}) \ar[r,"\gamma"] \ar[d,"\phi_{G}"]
  & \HH^{1}(E,\iOut_{G}) \ar[d,"\phi_{\Delta}",hook] \\
    \prod_{v} \HH^{0}(E_{v},\iOut_{G}) \ar[r,"\delta"]
  & \prod_{v} \HH^{1}(E_{v},\iInn_{G}) \ar[r,"\varepsilon"]
  & \prod_{v} \HH^{1}(E_{v},\iAut_{G}) \ar[r]
  & \prod_{v} \HH^{1}(E_{v},\iOut_{G}) \\
 \end{tikzcd}
\]
Recall that \cref{phiDelta} shows that the map \(\phi_{\Delta}\) is injective.
We claim that
the map \(\delta\) is trivial; and
the map \(\psi\) is injective.
Recall from \cref{exactEpts} that the sequences
\( 0 \to \iInn_{G}(E_{v}) \to \iAut_{G}(E_{v}) \to \iOut_{G}(E_{v}) \to 0 \)
are exact. This shows that the map \(\delta\) is trivial.
The map \(\psi\) is injective by theorem~6.22 on page~336 of~\cite{PlaRap},
since \(\iInn_{G} = G^{\ad}\).

We can now finish the proof of \cref{locglob} by a diagram chase.
Let \(x \in \HH^{1}(E, \iAut_{G})\) be a class such that
\(\phi_{G}(x) = *\).
Then \(\phi_{\Delta} \circ \gamma(x) = *\),
and since \(\phi_{\Delta}\) is injective we conclude that \(\gamma(x) = *\).
Therefore \(x = \beta(y)\) for some class \(y \in \HH^{1}(E, \iInn_{G})\).
Since \(\delta\) is trivial and \(\psi\) is injective,
we find that \(\varepsilon \circ \psi\) is injecitve.
Because \(\varepsilon \circ \psi(y) = \phi_{G} \circ \beta(y) = *\),
we conclude that \(y = *\), and hence \(x = *\).
This completes the proof of \cref{locglob}.

\section{Recap: classification of real adjoint Shimura data}

\readme
Deligne completely describes all pairs~\((G,X)\), where
\(G\) is a simple adjoint real algebraic group, and
\(X\) is a conjugacy class of homomorphisms
\(h \colon \DelS \to G\) satisfying the following conditions:
\begin{enumerate*}[label=(\kern-.1pt\textit{\roman*}\kern.6pt)]
 \item the adjoint representation~\(\Lie(G)\) is of type
  \(\{(-1,1),(0,0),(1,-1)\}\);
 \item conjugation by \(h(i)\) is a Cartan involution; and
  \item \(h\) is non-trivial.
\end{enumerate*}
What follows is a partial translation of~\S0 and~\S1 of~\cite{Del_ShimVar}.
Each paragraph indicates to which paragraph of~\cite{Del_ShimVar} it corresponds
and intermezzos are clearly specified as such.

\paragraph{Terminology and notation (parts of~\S0 of~\cite{Del_ShimVar})}
In this section \emph{reductive group}
always means \emph{connected reductive group}.
A \emph{covering} of a reductive group is a \emph{connected} covering.
\emph{Adjoint} group means \emph{reductive adjoint} group.
If \(G\) is a reductive group, we denote with \(G^{\ad}\) its adjoint group,
with \(G^{\der}\) its derived subgroup, and with
\(\rho \colon \tilde{G} \to G^{\der}\) the universal covering of~\(G^{\der}\).
We always denote with~\(Z\) (or~\(Z(G)\)) the centre of~\(G\),
and (conflict of notation) with \(\tilde{Z}\) that of~\(\tilde{G}\).

With a superscript~\({}^{\circ}\) we denote a
\emph{algebraic connected component}
(for example: \(Z^{\circ}\) is
the connected component of the identity of the centre~\(Z\) of~\(G\)).
The superscript~\({}^{+}\) denotes a
\emph{topological connected component}
(for example: \(G(\RR)^{+}\) is the topological connected component of the
identity of the real points of a group~\(G\)).
We also denote with \(G(\QQ)^{+}\) the trace of~\(G(\RR)^{+}\) on~\(G(\QQ)\).
For \(G\) real reductive, we denote with a subscript \({}_{+}\)
the inverse image of \(G^{\ad}(\RR)^{+}\) in~\(G(\RR)\).
The same notation \({}_{+}\) for the trace on the rational points of a group.

If \(X\) is a topological space,
then we denote with \(\pi_{0}(X)\) the set of its connected components,
endowed with the quotient topology of that on~\(X\).

A \emph{hermitian symmetric domain} is a hermitian symmetric space
with curvature \(< 0\) (that is, without euclidean or compact factors).

If not expressly mentioned otherwise,
a \emph{vector space} is supposed to be finite-dimensional,
and a \emph{number field} is supposed to be of finite degree over~\(\QQ\).

\paragraph{Moduli spaces of Hodge structures (\S1.1 of~\cite{Del_ShimVar})}
Recall that a Hodge structure on a real vector space~\(V\)
is a bigrading \(V_{\CC} = \bigoplus V^{pq}\) of the complexification of~\(V\),
such that \(V^{pq}\) is the complex conjugate of~\(V^{qp}\).

Define an action \(h\) of~\(\CC^{*}\) on~\(V_{\CC}\) by the formula
\[
 h(z)v = z^{-p}\bar{z}^{-q}v \qquad \text{for \(v \in V^{pq}\).}
\]
The \(h(z)\) commute with complex conjugation on~\(V_{\CC}\),
and thus are deduced from the extension of scalars of an action,
again denoted~\(h\), of~\(\CC^{*}\) on~\(V\).
Regard \(\CC\) as an extension of~\(\RR\),
and let~\(\DelS\) be its multiplicative group,
regarded as real algebraic group
(in other words, \(\DelS = \Res_{\CC/\RR}\Gm\)
(restriction of scalars in the sense of Weil)).
One has \(\DelS(\RR) = \CC^{*}\),
and \(h\) is an action of the algebraic group~\(\DelS\).
One verifies that this construction defines an equivalence of categories:
\(\{\)real vector spaces endowed with a Hodge structure
\(\} \to \{\)
real vector spaces endowed with an action of the algebraic group~\(\DelS\}\).

The inclusion \(\RR^{*} \subset \CC^{*}\) corresponds to an inclusion
of real algebraic groups \(\Gm \subset \DelS\).
We denote with~\(w_{h}\) (or simply~\(w\))
the restriction of~\(h^{-1}\) to~\(\Gm\),
and call \(w \colon \Gm \to \GL(V)\) the \emph{weights}.
One says that~\(V\) is pure of weight~\(n\)
if \(V^{pq} = 0\) for \(p + q \ne n\),
that is, \(w(\lambda)\) is multiplication by the scalar~\(\lambda^{n}\).

We denote with \(\mu_{h}\) (or simply~\(\mu\))
the action of~\(\Gm\) on~\(V_{\CC}\)
defined by \(\mu(z)v = z^{-p}v\) for \(v \in V^{pq}\).
It is a composition \(\Gm \to \DelS_{\CC} \stackrel{h}{\to} \GL(V_{\CC})\).

The \emph{Hodge filtration} \(F_{h}\) (or simply~\(F\)) is defined by
\(F^{p} = \bigoplus_{r \ge p} V^{rs}\).
One says that \(V\) is of type \(\mathscr{E} \subset \ZZ \times \ZZ\)
if \(V^{pq} = 0\) for \((p,q) \notin \mathscr{E}\).

More generally, if \(A\) is a subring of~\(\RR\)
such that \(A \otimes \QQ\) is a field
(in practice, \(A = \ZZ\),~\(\QQ\), or~\(\RR\)),
an \emph{\(A\)-Hodge structure}
is an \(A\)-module~\(V\) of finite type,
endowed with a Hodge structure on \(V \otimes_{A} \RR\).

\paragraph{Intermezzo}
The reader who is unfamiliar with
(variation of) Hodge structures may want to read the
instructive paragraphs \S\S~1.1.2--1.1.9 of~\cite{Del_ShimVar}.

\paragraph{(\S1.1.10 of~\cite{Del_ShimVar})}
A \emph{polarisation} of a real Hodge structure~\(V\) of weight~\(n\)
is a morphism \(\Psi \colon V \otimes V \to \RR(-n)\)
such that the form \((2\pi i)^{n} \Psi(x, h(i)y)\)
is symmetric and positive definite.
The same for \(\ZZ\)-Hodge structures,
upon replacing \(\RR(-n)\) with \(\ZZ(-n)\), \dots.
Since \(\Psi(h(i)x, y) = \Psi(x, h(-i)y)\)
(indeed, \(h(i)\) is trivial on~\(\RR(-n)\)),
and \(h(-i)y = (-1)^{n}h(i)y\),
the condition of symmetry is equivalent to:
\(\Psi\) is symmetric if \(n\) is even,
alternating if \(n\) is odd.

Hodge structure that come from algebraic geometry
are \(\ZZ\)-Hodge structures that are pure and polarisable.
Fundamental example:
the Hodge positivity theorems assure that \(\HH^{n}(X, \ZZ\),
for \(X\) a smooth projective variety, is polarisable.



\paragraph{}
Recall that \(X^{*}(\DelS) = \Hom(\DelS_{\CC}, \Gm)\)
is generated by two characters \(z\) and~\(\bar{z}\)
that are permuted by the action of~\(\Gal(\CC/\RR)\).
Let \(G\) be a real algebraic group,
and let \(h \colon \DelS \to G\) be a morphism.


\begin{proposition}[1.2.2 of~\cite{Del_ShimVar}] \label{Del122}
 Let \(G_{\CC}\) be a simple adjoint complex algebraic group.
 There is a bijection
 between \(G_{\CC}(\CC)\)-conjugacy classes
 of pairs \((G,X)\) as above, with \(G\) a real form of~\(G_{\CC}\),
 and \(G_{\CC}(\CC)\)-conjugacy classes of
 non-trivial morphisms \(\mu \colon \Gm[\CC] \to G_{\CC}\)
 that satisfy the following condition:

 \((\star)\)
 only the characters \(z^{-1}\),~\(1\), and~\(z\)
 occur in the representation
 \(\ad \circ \mu\) of~\(\Gm\) on~\(\Lie(G_{\CC})\).
\end{proposition}

\paragraph{\S1.2.5 of~\cite{Del_ShimVar}}
Let \(G\) be a simple adjoint complex algebraic group.
We enumerate the conjugacy classes of non-trivial morphisms
\(\mu \colon \Gm \to G\) that satisfy condition \((\star)\) of \cref{Del122}.



\paragraph{TODO} %TODO

\begin{lemma}
 Let \(M\) be an irreducible hyperadjoint abelian motive.
 If on non-compact factor of \(\GB(M)\) is of type \(D_{k}^{\HQ}\)
 then all non-compact factors are. This is also true if \(k = 4\).
 \begin{proof}
  See \cite{No06}; \S\S2.5--2.7.
 \end{proof}
\end{lemma}

\begin{proposition}
 Let \(M\) be an irreducible hyperadjoint abelian motive of type \(D_{k}\).
 One can recognize from \(\HH_{\Lambda}(M)\)
 whether \(M\) is of type \(D_{k}^{\RR}\) or of type \(D_{k}^{\HQ}\).
 \begin{proof}
  Use theorem~6.2 till remark~6.6 of~\cite{No06}.
 \end{proof}
\end{proposition}

\section{Compatible systems}
TODO %TODO
Give a summary of the other article.
First finish that article,
then copy/paste the relevant definitions and the main theorem.

\section{Geometric lifts of Galois representations}

Give a summary of the relevant result of~\cite{No06}.

\section{The Hodge numbers of a motive with endomorphisms}

\paragraph{Setup}
Let \(K \subset \CC\) be a finitely generated field.
Let \(\bar{K}\) be the algebraic closure of~\(K\) in~\(\CC\).
Let \(M\) be a motive over~\(K\).
Let \(E\) be subfield of~\(\End(M)\).
Let \(\tilde{E}\) be the normal closure of~\(E\) in~\(\CC\).
After passing to a suitable finite extension of~\(K\)
we may and do assume that \(\tilde{E} \subset K\).

Let \(p\) be a prime number.
Fix an embedding \(K \into K_{v}\) into some \(p\)-adic field~\(K_{v}\).
Let \(\BdR{K_{v}}\) be the ring of \(p\)-adic periods associated with~\(K_{v}\),
in the sense of Fontaine~\cite{Fo94}.
Fix an embedding \(\bar{K} \into \BdR{K_{v}}\)
that extends the embedding \(K \into K_{v}\).
Write \(\bar{K}_{v}\) for the algebraic closure of~\(K_{v}\) in~\(\BdR{K_{v}}\).

Let \(\Sigma = \Sigma(E)\) be the set of embeddings \(E \into \tilde{E}\).
Depending on the context,
we may also view an element \(\sigma \in \Sigma\)
as embedding \(E \into K\) or \(E \into \CC\) or \(E \into \BdR{K_{v}}\).

\paragraph{}
Observe that \(\HB(M) \otimes_{\QQ} \CC\) is a module over
\(E \otimes_{\QQ} \CC = \prod_{\sigma \in \Sigma} \CC\).
Thus we have a decomposition
\(\HB(M) \otimes_{\QQ} \CC \cong
 \bigoplus_{\sigma \in \Sigma} \HB(M) \otimes_{E,\sigma} \CC\).
Since the comparison isomorphism
\(\HB(M) \otimes_{\QQ} \CC \cong \HdR(M) \otimes_{K} \CC\)
is compatible with the action of~\(E\),
we find that \(\HdR(M)\) is a filtered module over
\(E \otimes_{\QQ} K \cong \prod_{\sigma \in \Sigma} K\).
This gives a decomposition
\(\HdR(M) \cong \bigoplus_{\sigma \in \Sigma} \HdR(M)^{(\sigma)}\)
in such a way that
\[
 \HdR(M)^{\sigma} \otimes_{K} \CC \cong \HB(M) \otimes_{E,\sigma} \CC
\]
is a natural isomorphism of filtered vector spaces.

\paragraph{}
We have a natural isomorphism of vector spaces
\(\HB(M) \otimes \QQp \cong \Hp(M)\).
This isomorphism is compatible with the action of~\(E\),
and \(\Hp(M)\) is a module over \(E \otimes \QQp\).

We have a natural isomorphism
\(\HdR(M) \otimes_{K} \BdR{K_{v}} \cong \Hp(M) \otimes_{\QQp} \BdR{K_{v}}\)
of filtered modules with an action by \(\Gal(\bar{K}_{v}/K_{v})\).
This isomorphism is compatible with the action of~\(E\),
so that \(\Hp(M) \otimes_{\QQp} \BdR{K_{v}}\)
is a module over
\(E \otimes_{\QQ} \BdR{K_{v}} = \prod_{\sigma \in \Sigma} \BdR{K_{v}}\).
Thus we get a natural isomorphism
\[
 \HdR(M)^{(\sigma)} \otimes_{K} \BdR{K_{v}} \cong
 \Hp(M) \otimes_{E \otimes \QQp, \sigma \otimes \id} \BdR{K_{v}}
\]
of filtered modules with an action by \(\Gal(\bar{K}_{v}/K_{v})\),
for each \(\sigma \in \Sigma\).

\paragraph{}
Assume that \(p\) is totally split in~\(E\).
Then we get a simplified description of
\(\Hp(M) \otimes_{E \otimes \QQp, \sigma \otimes \id} \BdR{K_{v}}\),
as follows.
For each \(\sigma \in \Sigma\)
the composite embedding \(\sigma \colon E \into K \into K_{v}\)
factors via \(\QQp \subset K_{v}\).
This gives a bijection between \(\Sigma\)~and
the places \(\pi\) of~\(E\) that lie above~\(p\).

Since \(\Hp(M)\) is a module over \(E \otimes \QQp = \prod_{\pi|p} E_{\pi}\)
we get a decomposition \(\Hp(M) \cong \bigoplus_{\pi|p} \HH_{\pi}(M)\)
of Galois representations.
The above bijection between \(\Sigma\) and the places \(\pi|p\)
allows us to unambigously write \(\Hp(M)^{(\sigma)}\) for~\(\HH_{\pi}(M)\).
Then
\(\Hp(M) \otimes_{E \otimes \QQp, \sigma \otimes \id} \BdR{K_{v}}
 = \Hp(M)^{(\sigma)} \otimes_{\QQp} \BdR{K_{v}}\).

\paragraph{}
The upshot of all these comparison isomorphisms
and their decompositions induced by the action of~\(E\)
is that we get a refined view on the Hodge numbers of~\(M\).
It is classical that the comparison isomorphisms
equate the Hodge numbers of~\(\HB(M)\)
with the Hodge--Tate numbers of the Galois representation~\(\Hp(M)\).
Now we also have the following result.

\begin{theorem}
 Let \(M\) be a motive over a finitely generated field \(K \subset \CC\).
 Let \(E\) be subfield of~\(\End(M)\), and
 let \(\tilde{E}\) be the normal closure of~\(E\) in~\(\CC\).
 Assume that \(\tilde{E} \subset K\).
 Let \(\Sigma = \Sigma(E)\) be the set of embeddings \(E \into \tilde{E}\).
 For each \(\sigma \in \Sigma\) we then have
 \[
  \bigl\{\,
  \text{Hodge numbers of \(\HB(M) \otimes_{E,\sigma} \CC\)}
  \,\bigr\}
  =
  \bigl\{\,
  \text{Hodge--Tate numbers of
   \(\Hp(M) \otimes_{E \otimes \QQp, \sigma \otimes \id} \BdR{K_{v}}\)}
  \,\bigr\}.
 \]
 \begin{proof}
  By the above computations we see that both the left-hand side
  and the right-hand side are equal to
  \(
   \bigl\{\,
   \text{Hodge numbers of \(\HdR(M)^{(\sigma)}\)}
   \,\bigr\}
  \).
 \end{proof}
\end{theorem}

\section{Foobar}

\paragraph{}
Recall that \(\HH^{1}(E,\iAut_{G})\) classifies forms of~\(G\) over~\(E\);
reference?

\paragraph{}
Let \(K \subset \CC\) be a finitely generated field.
Let \(M\) be a hyperadjoint abelian motive over~\(K\).
Let \(E\) be the endomorphism algebra of~\(M\).
Let \(\tilde{E}\) be the normal closure of~\(E\) in~\(\CC\).

After passing to a suitable finitely generated extension of~\(K\)
we may and do assume that
\begin{enumerate*}[label=(\textit{\roman*})]
 \item \(E = \End(M_{\CC})\);
 \item \(\Gl(M)\) is connected for all prime numbers~\(\ell\); and
 \item \(\tilde{E} \subset K\).
\end{enumerate*}
(Actually assumption~(\textit{ii}) implies assumption~(\textit{i}),
but that is not important right now.)
Let \(\bar{K}\) be the algebraic closure of~\(K\) in~\(\CC\).

Let \(\Sigma = \Sigma(E)\) be the set of embeddings \(E \into \tilde{E}\).
Depending on the context,
we may also view an element \(\sigma \in \Sigma\)
as embedding \(E \into K\) or \(E \into \CC\).


\section{An intermediate result}

\begin{theorem}
 Let \(M_{1}\) and~\(M_{2}\) be two hyperadjoint abelian motives
 over a finitely generated field \(K \subset \CC\).
 Assume that \(\MTC(M_{1})\) and \(\MTC(M_{2})\) are true.
 Then
 \[
  M_{1} \cong M_{2} \iff \exists \ell :
  \Gl(M_{1} \oplus M_{2}) \subsetneq \Gl(M_{1}) \times \Gl(M_{2}).
 \]
\end{theorem}

\begin{corollary}
 Let \(M_{1}\) and~\(M_{2}\) be two hyperadjoint abelian motives
 over a finitely generated subfield of~\(\CC\).
 If \(\MTC(M_{1})\) and \(\MTC(M_{2})\) are true,
 then \(\MTC(M_{1} \oplus M_{2})\) is true.
\end{corollary}

\begin{proposition} \label{twofib}
 Let \(K \subset \CC\) be a finitely generated field.
 Let \(S\) be a scheme of finite type over~\(K\).
 Let \(f \colon \mathcal{A} \to S\) be an abelian scheme.
 Let \(x\) and~\(y\) be two points in~\(S(K)\).
 Let \(\ell\) be a prime number.
 If \(\Lie(\Gl(\mathcal{A}_{x}))\) and \(\Lie(\Gl(\mathcal{A}_{y}))\)
 are isomorphic as Galois representations,
 then the abelian varieties \(\mathcal{A}_{x}\) and~\(\mathcal{A}_{y}\)
 are isogenous over a finite extension of~\(K\).
 \begin{proof}
  View the lisse \(\ell\)-adic sheaf \(\mathrm{R}^{1}f_{*}\QQl\)
  as representation
  \(\pi(S,\bar{x}) \to \GL(\Hl^{1}(\mathcal{A}_{x})\).
  Let \(\rho_{x}\) be denote the representation
  \(\Gal(\bar{K}/K) \to \GL(\Hl^{1}(\mathcal{A}_{x})\).
  Via parallel transport,
  the Galois representation on~\(\Hl^{1}(\mathcal{A}_{y})\)
  corresponds with a representation
  \(\rho_{y} \colon \Gal(\bar{K}/K) \to \GL(\Hl^{1}(\mathcal{A}_{x})\).
  Let \(G \subset \GL(\Hl^{1}(\mathcal{A}_{x})\) be the smallest
  subgroup such that \(G(\QQl)\) contains the image
  of both \(\rho_{x}\) and~\(\rho_{y}\).

  Consider the maps \(\ab \colon G \to G^{\ab}\)
  and \(\ad \colon G \to G^{\ad}\).
  It follows from \cref{MTCcentre} that
  the map \(\ab \circ \rho_{s}\) does not depend on~\(s \in \{x,y\}\).
  By assumption
  \(\Lie(\Gl(\mathcal{A}_{x}))\) and \(\Lie(\Gl(\mathcal{A}_{y}))\)
  are isomorphic as Galois representations,
  and therefore \(\ad \circ \rho_{s}\) does not depend on~\(s \in \{x,y\}\).

  Write \(Z\) for the finite intersection \(\ker(\ab) \cap \ker(\ad)\).
  Define the morphism \(\phi \colon \Gal(\bar{K}/K) \to G(\QQl)\)
  via \(g \mapsto \rho_{x}(g)\rho_{y}(g)^{-1}\).
  We claim that \(\phi\) is a homomorphism.
  Note that \(\ab \circ \phi\) and \(\ad \circ \phi\)
  are trivial, and therefore \(\phi\) factors via~\(Z(\QQl)\).
  We need to show that
  \[
   \rho_{x}(g_{1})\rho_{y}(g_{1})^{-1} \cdot \rho_{x}(g_{2})\rho_{y}(g_{2})^{-1}
   \stackrel{?}{=}
   \rho_{x}(g_{1}g_{2}) \cdot \rho_{y}(g_{1}g_{2})^{-1}
   =
   \rho_{x}(g_{1})\rho_{x}(g_{2}) \cdot
   \rho_{y}(g_{2})^{-1}\rho_{y}(g_{1})^{-1}.
  \]
  This amounts to showing that
  \(\rho_{y}(g_{1})^{-1} \cdot \phi(g_{2}) =
   \phi(g_{2}) \cdot \rho_{y}(g_{1})^{-1}\).
  Since \(Z\) is contained in the centre of~\(G\),
  we know that \(\phi(g_{2})\) commutes with \(\rho_{y}(g_{1})^{-1}\).
  This proves the claim that \(\phi\) is a homomorphism.

  Because \(Z\) is finite,
  the homomorphism \(\phi\) becomes trivial
  after replacing \(K\) by suitable a finite extension;
  which means \(\rho_{x} = \rho_{y}\).
  The \namecref{twofib} now follows from Faltings' results
  on the Tate conjecture for endomorphisms of abelian varieties
  (Korollar~1 of Satz~4 of~\cite{Fal83}, see also~\cite{Fal84}).
 \end{proof}
\end{proposition}

\section{Main theorem}

\paragraph{}
Let \(K \subset \CC\) be a finitely generated field.
Let \(\mathcal{M}_{K} \subset \Mot_{K}\) be the subcategory
of \emph{abelian} motives for which the Mumford--Tate conjecture is true.

\begin{theorem}
 Let \(K \subset \CC\) be a finitely generated field.
 The category~\(M_{K}\) is a Tannakian subcategory of \(\Mot_{K}\).
\end{theorem}

\begin{corollary}
 Let \(A_{1}\),~\(A_{2}\), \dots, \(A_{n}\) be abelian varieties over~\(K\).
 If the Mumford--Tate conjecture is true for \(A_{i}\), for \(i = 1,\dots,n\),
 then the Mumford--Tate conjecture is true for
 \(A_{1} \times A_{2} \times \cdots \times A_{n}\).
\end{corollary}


\printbibliography

\end{document}
